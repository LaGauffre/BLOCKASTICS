\documentclass{beamer}
\usepackage[utf8]{inputenc}
\usepackage[T1]{fontenc}
% \usepackage{amscd, amsfonts, amsmath, amssymb, amstext, amsthm, caption, epsfig, fancyhdr, float, graphicx, latexsym, mathtools, multicol, multirow, algorithm, chngcntr}
\usepackage[english, french]{babel}
\usepackage{booktabs}

\usepackage{amsmath,amssymb}
\usepackage{graphicx}
\usepackage{caption}
\usepackage{subfig}
\usepackage{xspace}
\usepackage{fourier}


\usepackage{tikz}
\usetikzlibrary{shapes,arrows}
\usepackage{tkz-graph}
\usetikzlibrary{automata,arrows,positioning,calc}
\usetikzlibrary{positioning}
\usetikzlibrary{fit}
\usetikzlibrary{backgrounds}
\usetikzlibrary{calc}
\usetikzlibrary{shapes}
\usetikzlibrary{mindmap}
\usetikzlibrary{decorations.text}
\usetikzlibrary{snakes}

\usepackage{algorithm}
% \usepackage{algorithmic}
\usepackage{algorithmicx}
\usepackage{listings}
\usepackage{hyperref}
\hypersetup{
    bookmarks=true,         % show bookmarks bar?
    unicode=false,          % non-Latin characters in Acrobat’s bookmarks
    pdftoolbar=true,        % show Acrobat’s toolbar?
    pdfmenubar=true,        % show Acrobat’s menu?
    pdffitwindow=false,     % window fit to page when opened
    pdfstartview={FitH},    % fits the width of the page to the window
    pdftitle={My title},    % title
    pdfauthor={Author},     % author
    pdfsubject={Subject},   % subject of the document
    pdfcreator={Creator},   % creator of the document
    pdfproducer={Producer}, % producer of the document
    pdfkeywords={keyword1, key2, key3}, % list of keywords
    pdfnewwindow=true,      % links in new PDF window
    colorlinks=false,       % false: boxed links; true: colored links
    linkcolor=red,          % color of internal links (change box color with linkbordercolor)
    citecolor=green,        % color of links to bibliography
    filecolor=cyan,         % color of file links
    urlcolor=magenta        % color of external links
}

% \theoremstyle{definition} % insert bellow all blocks you want in normal text
% \newtheorem{definition}{Definition}



% tikzmark command, for shading over items
\newcommand{\tikzmark}[1]{\tikz[overlay,remember picture] \node (#1) {};}
% Define block styles
\tikzstyle{decision} = [diamond, draw, fill=blue!20,
    text width=4.5em, text badly centered, node distance=3cm, inner sep=0pt]
\tikzstyle{block} = [rectangle, draw, fill=blue!20,
    text width=5em, text centered, rounded corners]
\tikzstyle{line} = [draw]
\tikzstyle{cloud} = [draw, ellipse,fill=red!20, node distance=3cm,
    minimum height=2em]

\usepackage[most]{tcolorbox}

\setbeamertemplate{blocks}[rounded][shadow=true] % use rounded blocks with standard beamer shadow


% Distributions.
\newcommand*{\UnifDist}{\mathsf{Unif}}
\newcommand*{\ExpDist}{\mathsf{Exp}}
\newcommand*{\DepExpDist}{\mathsf{DepExp}}
\newcommand*{\GammaDist}{\mathsf{Gamma}}
\newcommand*{\LognormalDist}{\mathsf{LogNorm}}
\newcommand*{\WeibullDist}{\mathsf{Weib}}
\newcommand*{\ParetoDist}{\mathsf{Par}}
\newcommand*{\NormalDist}{\mathsf{Norm}}

\newcommand*{\GeometricDist}{\mathsf{Geom}}
\newcommand*{\NegBinomialDist}{\mathsf{NegBin}}
\newcommand*{\PoissonDist}{\mathsf{Poisson}}
\newcommand*{\BivariatePoissonDist}{\mathsf{BPoisson}}
\newcommand*{\CyclicalPoissonDist}{\mathsf{CPoisson}}

\newcommand*{\iid}{\textbf{iid}\@\xspace}
\newcommand*{\pdf}{\textbf{pdf}\@\xspace}
\newcommand*{\cdf}{\textbf{cdf}\@\xspace}
\newcommand*{\pmf}{\textbf{pmf}\@\xspace}
\newcommand*{\abc}{{\textbf{abc}}\@\xspace}
\newcommand*{\smc}{\textbf{smc}\@\xspace}
\newcommand*{\mcmc}{\textbf{mcmc}\@\xspace}
\newcommand*{\ess}{\textbf{ess}\@\xspace}
\newcommand*{\mle}{\textbf{mle}\@\xspace}
\newcommand*{\bic}{\textbf{bic}\@\xspace}
\newcommand*{\kde}{\textbf{kde}\@\xspace}
\newcommand*{\glm}{\textbf{glm}\@\xspace}
\newcommand*{\xol}{\textbf{xol}\@\xspace}
\newcommand*{\cpu}{\textbf{cpu}\@\xspace}
\newcommand*{\gpu}{\textbf{gpu}\@\xspace}
\newcommand*{\arm}{\textbf{arm}\@\xspace}

\def \si {\sigma}
\def \la {\lambda}
\def \al {\alpha}
% \def\e*{\end{eqnarray*}}
\def \di{\displaystyle}

\def \E{\mathbb E}
\def \N{\mathbb N}
\def \Z{\mathbb Z}
\def \NZ{\mathbb{N}_0}
\def \I{\mathbb I}
\def \w{\widehat}
\def \P {\mathbb P}
\def \V{\mathbb V}


\newcommand{\CL}{\mathbb{C}}
\newcommand{\RL}{\mathbb{R}}
\newcommand{\nat}{{\mathbb N}}
\newcommand{\Laplace}{\mathscr{L}}
\newcommand{\e}{\mathrm{e}}
\newcommand{\ve}{\bm{\mathrm{e}}} % vector e

\renewcommand{\L}{\mathcal{L}} % e.g. L^2 loss.

\newcommand{\ih}{\mathrm{i}}
\newcommand{\oh}{{\mathrm{o}}}
\newcommand{\Oh}{{\mathcal{O}}}
\newcommand{\Exp}{\mathbb{E}}

\newcommand{\Norm}{\mathcal{N}}
\newcommand{\LN}{\mathcal{LN}}
\newcommand{\SLN}{\mathcal{SLN}}

\renewcommand{\Pr}{\mathbb{P}}
\newcommand{\Ind}{\mathbb I}
\newcommand\bfsigma{\bm{\sigma}}
\newcommand\bfSigma{\bm{\Sigma}}
\newcommand\bfLambda{\bm{\Lambda}}
\newcommand{\stimes}{{\times}}
\def \limsup{\underset{n\rightarrow+\infty}{\overline{\lim}}}
\def \liminf{\underset{n\rightarrow+\infty}{\underline{\lim}}}


\AtBeginSection[]
  {
     \begin{frame}<beamer>
     \frametitle{Agenda}
     \tableofcontents[currentsection]
     \end{frame}
  }



% vertical separator macro
\newcommand{\vsep}{
  \column{0.0\textwidth}
    \begin{tikzpicture}
      \draw[very thick,black!10] (0,0) -- (0,7.3);
    \end{tikzpicture}
}
\newcommand\blfootnote[1]{%
  \begingroup
  \renewcommand\thefootnote{}\footnote{#1}%
  \addtocounter{footnote}{-1}%
  \endgroup
}

% More space between lines in align
% \setlength{\mathindent}{0pt}

% Beamer theme
\usetheme{ZMBZFMK}
\usefonttheme[onlysmall]{structurebold}
\mode<presentation>
\setbeamercovered{transparent=10}

% align spacing
\setlength{\jot}{0pt}

\setbeamertemplate{navigation symbols}{}%remove navigation symbols

\title[UNSW Workshop]{Stochastic Models for Blockchain analysis}

\author{Pierre-O. Goffard}
\institute[UNISTRA]{Université de Strasbourg\\
 \texttt{goffard@unistra.fr}
}
\date{\today}
% \titlegraphic{\includegraphics[width=2.5c../../Figures/bfs_logo.png}} 
% \titlegraphic{
% \includegraphics[width=0.25\textwidt../../Figures/DALL-E/WS_UNSW_DEFI.png}
% }
\begin{document}
% \begin{frame}[plain]
%   \includegraphics[width=\textwidt../../Figures/DALL-E/WS_UNSW_DEFI.png}
% \end{frame}
\begin{frame}
  \titlepage
\end{frame}
\begin{frame}{Agenda}
  \tableofcontents
\end{frame}


\section{Security of PoW blockchains}
\subsection{Double spending attack}
\begin{frame}{Double spending attack}
\scriptsize
\underline{I. Security of PoW blockchains}\\
\underline{I.1 Double spending attack}
\begin{enumerate}
\item Mary transfers 10 BTCs to John
\item The transaction is recorded in the public branch of the blockchain and John ships the good.
\item Mary transfers to herself the exact same BTCs
\item The malicious transaction is recorded into a private branch of the blockchain
\begin{itemize}
  \scriptsize
\item Mary has friends among the miners to help her out
\item The two chains are copycat up to the one transaction
\end{itemize}
\end{enumerate}
\begin{tcolorbox}[enhanced,drop shadow, title=Fact (Bitcoin has only one rule)]
The longest chain is to be trusted
\end{tcolorbox}
\end{frame}
\begin{frame}{Double Spending attack}
\scriptsize
Vendor are advised to wait for $\alpha\in\mathbb{N}$ of confirmations so that the honest chain is ahead of the dishonest one.
\begin{center}
\begin{tikzpicture}[-, >=stealth', auto, semithick, node distance=1cm]
% \tikzstyle{block} = [rectangle, draw, fill=blue!20,
%     text width=5em, text centered, rounded corners]
\tikzstyle{block}=[rectangle, fill=black,draw=black,thick,text=black,scale=0.6]
\tikzstyle{block}=[rectangle, fill=white,draw=black,thick,text=black,scale=0.8]
\tikzstyle{confirmed block}=[rectangle, fill=white,draw=blue,thick,text=black,scale=0.8]
\tikzstyle{bad block}=[rectangle, fill=white,draw=red,thick,text=black,scale=0.8]
\node[block]    (1)                     {\tiny $\text{M}\rightarrow \text{J}$};
\node[block]    (2)[right of=1]                     {};
\node[block]    (3)[right of=2]                     {};
\node[block]    (4)[right of=3]                     {};
\node[confirmed block]    (5)[right of=4]                     {};

\node[bad block]    (6)[below of=1]         {\tiny $\text{M}\rightarrow \text{M}$};
\node[block]    (7)[right of=6]         {};
\node[block]    (8)[right of=7]         {};
\path
(1) edge[ left]     node{}     (2)
(2) edge[ left]     node{}     (3)
(3) edge[ left]     node{}     (4)
(4) edge[ left]     node{}     (5)
(6) edge[ left]     node{}     (7)
(7) edge[ left]     node{}     (8);

\end{tikzpicture}
\end{center}
In the example, vendor awaits $\alpha = 4$ confirmations, the honest chain is ahead of the dishonest one by $z = 2$ blocks.
\begin{tcolorbox}[enhanced,drop shadow, title=Fact (PoW is resistant to double spending)]
\begin{itemize}
\item Attacker does not own the majority of computing power 
\item Suitable $\alpha$ 
\end{itemize}
Double spending is unlikely to succeed.
\end{tcolorbox}
\tiny
\begin{thebibliography}{1}
\bibitem{Na08}
S.~Nakamoto, ``Bitcoin: A peer-to-peer electronic cash system.'' Available at
  \href{https://bitcoin.org/bitcoin.pdf}{https://bitcoin.org/bitcoin.pdf},
  2008.
  
\end{thebibliography}

\end{frame}
\begin{frame}{Double Spending Attack}
\footnotesize
Assume that
\begin{itemize}
\item $R_0=z\geq1$ (the honest chain is z blocks ahead)
\item at each time unit a block is created
\begin{itemize}
  \footnotesize
\item[$\hookrightarrow$] in the honest chain with probability $p$
\item[$\hookrightarrow$] in the dishonest chain with probability $q=1-p$
\end{itemize}
\end{itemize}
The process $(R_n)_{n\geq0}$ is a random walk on $\mathbb{Z}$ with
$$R_n=z+Y_1+\ldots+Y_n,$$
where $Y_1,\ldots,Y_n$ are the \textbf{i.i.d.} steps of the random walk. 

\end{frame}
\begin{frame}{Double Spending Attack}
\scriptsize
Double spending occurs at time
$$
\tau_z=\inf\{n\in \mathbb{N}\text{ ; }R_n=0\}.
$$
\begin{tikzpicture}
  %Origin and axis
  \coordinate (O) at (0,0);
  \draw[->] (-1,0) -- (9,0) coordinate[label = {below:$n$}] (xmax);
  \draw[->] (0,-0.5) -- (0,3) coordinate[label = {left:$R_n$}] (ymax);
  %Lower linear boundary

 
  %Stochastic process trajectory
  
  \draw (0,0) node[tublue,left] {} node{};
  \draw[very thick,tublue,-] (0,1) -- (1,1) node[pos=0.5, above] {} ;
  \draw[very thick,dashed,tublue] (1,1) -- (1,1.5) node[pos=0.5, right] {};
  \draw[very thick,tublue,-] (1,1.5) -- (2,1.5) node[pos=0.5, above] {};
  \draw[very thick,dashed,tublue] (2,1.5) -- (2,2) node[pos=0.5, right] {};
  \draw[very thick,tublue,-] (2,2) -- (3,2) node[pos=0.5, above] {};
  \draw[very thick,dashed,tublue] (3,2) -- (3,1.5) node[pos=0.5, right] {};
  \draw[very thick,tublue,-] (3,1.5) -- (4,1.5)node[pos=0.5, above] {};
  \draw[very thick,dashed,tublue] (4,1.5) -- (4,1) node[pos=0.5, right] {};  
  \draw[very thick,tublue,-] (4,1) -- (5,1) node[pos=0.5, above] {};
  \draw[very thick,dashed,tublue] (5,1) -- (5,0.5) node[pos=0.5, right] {};  
  \draw[very thick,tublue,-] (5,0.5) -- (6,0.5) node[pos=0.5, above] {};
  \draw[very thick,dashed,tublue,-] (6,0.5) -- (6,1) node[pos=0.5, above] {};
   \draw[very thick,tublue,-] (6,1) -- (7,1) node[pos=0.5, above] {};
    \draw[very thick,dashed,tublue,-] (7,1) -- (7,0.5) node[pos=0.5, above] {};
     \draw[very thick,tublue,-] (7,0.5) -- (8,0.5) node[pos=0.5, above] {};
     \draw[very thick,dashed,tublue,-] (8,0.5) -- (8,0) node[pos=0.5, above] {};
  %Jump Times
  \draw (1,0) node[black,below] {$1$} node{ \color{black}$\bullet$};
  \draw (2,0) node[black,below] {$2$} node{ \color{black}$\bullet$};
  \draw (3,0) node[black,below] {$3$} node{ \color{black}$\bullet$};
  \draw (4,0) node[black,below] {$4$} node{ \color{black}$\bullet$};
  \draw (5,0) node[black,below] {$5$} node{ \color{black}$\bullet$};
  \draw (6,0) node[black,below] {$6$} node{ \color{black}$\bullet$};
  \draw (7,0) node[black,below] {$7$} node{ \color{black}$\bullet$};
  \draw (8,0) node[black,below] {$8$} node{ \color{black}$\bullet$};
  %Level of the counting process
   \draw (0,0) node[black,below left] {$0$} node{ \color{black}$\bullet$};
   \draw (0,0.5) node[black,left] {$1$} node{ \color{black}$\bullet$};
   \draw (0,1) node[black,left] {$z=2$} node{ \color{black}$\bullet$};
   \draw (0,1.5) node[black,left] {$3$} node{ \color{black}$\bullet$};
   \draw (0,2) node[black,left] {$4$} node{ \color{black}$\bullet$};
   \draw (0,2.5) node[black,left] {$5$} node{ \color{black}$\bullet$};

  % %Aggregated Capital gains
%  \draw (0,1.5) node[blue,below right] {$\mu_1$} node{ \color{blue}$-$};
%  \draw (0,2.25) node[blue,left] {$\mu_2$} node{ \color{blue}$-$};
%  \draw (0,3.75) node[blue,left] {$\mu_3$} node{ \color{blue}$-$};
  %Ruin time = First-crossing time time
%  \draw (5,0) node[black,above right] {$\tau_u$} node{ \color{black}$\times$};
%  \draw[dotted,black] (0,3.28) -- (5,3.28);
%  \draw[dotted,black] (5,0) -- (5,3.28);
\end{tikzpicture}\\
\underline{Goal:} Find 
$$
\psi(z) = \mathbb{P}(\tau_z<\infty)
$$


\end{frame}
% \begin{frame}{Refinements of the double spending problem}
% \scriptsize
% The number of blocks $M$ found by the attacker until the honest miners find $\alpha$ blocks is a negative binomial random variable with \pmf
% $$
% \mathbb{P}(M = m) = \binom{\alpha+m-1}{m}p^\alpha q^m,\text{ }m\geq0.
% $$
% The number of block that the honest chain is ahead of the dishonest one is given by 
% $$
% Z= (\alpha-M)_+.
% $$
% Applying the law of total probability yields the probability of successful double spending with
% $$
% \mathbb{P}(\text{Double Spending}) = \mathbb{P}(M\geq \alpha) + \sum_{m = 0}^{\alpha - 1}\binom{\alpha+m-1}{m}q^{\alpha} p^{m}.
% $$ 
% \tiny


% \begin{thebibliography}{1}
%   \bibitem{rosenfeld2014analysis}
% M.~Rosenfeld, ``Analysis of hashrate-based double spending,'' {\em arXiv
%   preprint arXiv:1402.2009}, 2014.
%   \bibitem{GRUNSPAN2018}
% C.~Grunspan and R.~Perez-Marco, ``Double spend race,'' {\em
%   International Journal of Theoretical and Applied Finance}, vol.~21,
%   p.~1850053, dec 2018.
% \end{thebibliography}

% \end{frame}
% \begin{frame}{Refinements of the double spending problem}
% \scriptsize
% Let the length of honest and dishonest chain be driven by counting processes
% \begin{itemize}
% \item Honest chain $\Rightarrow$ $z+N_t\text{ , }t\geq0$, where $z\geq1$.
% \item Malicious chain $\Rightarrow$ $M_t\text{ , }t\geq0$
% \item Study the distribution of the first-\textit{rendez-vous} time
% $$
% \tau_z=\inf\{t\geq0\text{ , } M_t=z+N_t\}.
% $$
% \end{itemize}
% If $N_t\sim\text{Pois}(\lambda t)$ and $M_t\sim\text{Pois}(\mu t)$ such that $\lambda>\mu$ then 
% $$
% \psi(z) = \left(\frac{\mu}{\lambda}\right)^z,\text{ }z\geq 0.
% $$
% \tiny
% \begin{thebibliography}{1}

% \bibitem{Goffard2019}
% P.-O. Goffard, ``Fraud risk assessment within blockchain transactions,'' {\em
%   Advances in Applied Probability}, vol.~51, pp.~443--467, jun 2019.
% \newblock \url{https://hal.archives-ouvertes.fr/hal-01716687v2}.

% \bibitem{Bowden2020}
% R.~Bowden, H.~P. Keeler, A.~E. Krzesinski, and P.~G. Taylor, ``Modeling and
%   analysis of block arrival times in the bitcoin blockchain,'' {\em Stochastic
%   Models}, vol.~36, pp.~602--637, jul 2020.
% \end{thebibliography}

% \end{frame}

\subsection{Insurance risk theory}
\begin{frame}{Insurance risk theory}
\scriptsize
\underline{I.2 Insurance risk theory}
\begin{columns}
\begin{column}{0.5\textwidth}

The financial reserves of an insurance company over time have the following dynamic
\begin{equation*}
R_t = z +ct - \sum_{i = 1}^{N_t}U_i\text{, }t\geq0,
\end{equation*}
where 
\begin{itemize}
  \item $z>0$ denotes the initial reserves
  \item $c$ is the premium rate
  \item $(N_t)_{t\geq0}$ is a counting process that models the claim arrival 
  \begin{itemize}
    \scriptsize
    \item[$\hookrightarrow$]  Poisson process with intensity $\lambda$
  \end{itemize}
  \item The $U_i$'s are the randomly sized compensations
  \begin{itemize}
    \scriptsize
    \item[$\hookrightarrow$] non-negative, \textbf{i.i.d.}
  \end{itemize}
\end{itemize}
\end{column}
\begin{column}{0.5\textwidth}
\begin{tikzpicture}
  %Origin and axis
  \coordinate (O) at (0,0);
  \draw[->] (-0.5,0) -- (5.5,0) coordinate[label = {below:\scriptsize$t$}] (xmax);
  \draw[->] (0,-0.5) -- (0,4) coordinate[label = {right:\scriptsize$R_t$}] (ymax);
   %Initial reserves
  \draw (0,2) node[black,left] {\scriptsize$z$} node{};
 % % %Length of the honest chain
  \draw[thick, tublue,-] (0,2) -- (2,3) node[pos=0.5, above] {};
  \draw[thick, dashed, tublue] (2,3) -- (2,1) node[pos=0.5, left] {\scriptsize\color{black}$U_1$};
  \draw[thick, tublue] (2,1) -- (3,1.5) node[pos=0.5, above] {};
  \draw[thick, dashed, tublue] (3,1.5) -- (3, 0.5) node[pos=0.5, left] {\scriptsize\color{black}$U_2$};
  \draw[thick, tublue] (3,0.5) -- (5, 1.5) node[pos=0.5, above] {};
   \draw[thick, dashed, tublue] (5,1.5) -- (5, -0.5) node[pos=0.5,above left] {\scriptsize\color{black}$U_3$};

  %Block finding Times 
  \draw (2,0) node[black,below] {\scriptsize$T_1$} node{ \color{black}$\bullet$};
  \draw (3,0) node[black,below] {\scriptsize$T_2$} node{ \color{black}$\bullet$};
  \draw (5,0) node[black,below left] {\scriptsize$\tau_z$} node{ \color{black}$\bullet$};
\end{tikzpicture}
\end{column}
\end{columns}

\end{frame}
\begin{frame}{Insurance risk theory}
\scriptsize
Define the ruin time as 
$$
\tau_z = \inf\{t\geq0\text{ ; }R_t <0\}
$$
and the ruin probabilities as 
$$
\psi(z,t) = \mathbb{P}(\tau_z < t)\text{ and }\psi(z) = \mathbb{P}(\tau_z < \infty)
$$
We look for $z$ such that 
$$
\mathbb{P}(\text{Ruin}) = \alpha\text{ (0.05)},
$$
given that 
$$
c=(1+\eta)\lambda\mathbb{E}(U),
$$
with 
$$\eta>0\text{ (net profit condition)}$$  
otherwise 
$$\psi(z)=1.$$

\tiny
\begin{thebibliography}{1}

\bibitem{Asmussen_2010}
S.~Asmussen and H.~Albrecher, {\em Ruin Probabilities}.
\newblock {WORLD} {SCIENTIFIC}, sep 2010.

\end{thebibliography}

\end{frame}

\begin{frame}{Insurance risk theory}
\scriptsize
Let 
$$
S_t = z - R_t,\text{ }t\geq0
$$
\begin{tcolorbox}[enhanced,drop shadow, title=Theorem (Wald exponential martingale)]
If $(S_t)_{t\geq0}$ is a L\'evy process or a random walk then
$$
\{\exp\left[\theta S_t-t\kappa(\theta)\right]\text{ , }t\geq0\},\text{ is a martingale,}
$$
where $\kappa(\theta)=\log\mathbb{E}\left(e^{\theta S_1}\right)$.
\end{tcolorbox}
\begin{tcolorbox}[enhanced,drop shadow, title=Theorem (Representation of the ruin probability)]

If $S_t\overset{\textbf{a.s.}}{\rightarrow} -\infty$, and there exists $\gamma>0$ such that $\{e^{\gamma S_t}\text{ , }t\geq0\}$ is a martingale then
$$
\mathbb{P}(\tau_z<\infty)=\frac{e^{-\gamma z}}{\mathbb{E}\left[e^{\gamma \xi(z)}|\tau_z<\infty\right]},
$$
where $\xi(z)=S_{\tau_z}-z\text{ denotes the deficit at ruin.}$
\end{tcolorbox}
\end{frame}
\begin{frame}{Sketch of Proof}
\scriptsize
\begin{itemize}
\item Because of the net profit condition $S_t = \sum_{i=1}^{N_t}U_i-ct\rightarrow -\infty$ as $t\rightarrow\infty$
\item $(S_t)_{t\geq0}$ is a Lévy process, let $\gamma$ be the (unique, positive) solution to
$$
\kappa(\theta) = 0\text{ (Cramer-Lundberg equation)}.
$$
\item  $(e^{\gamma S_t})_{t\geq0}$ is a Martingale then apply the Optional stopping theorem at $\tau_z$.

\end{itemize}
\end{frame}

\subsection{Double spending in Satoshi's framework}
\begin{frame}{Double spending in Satoshi's framework}

\scriptsize
\underline{I.3 Double spending in Satoshi's framework}
\begin{tcolorbox}[enhanced,drop shadow, title=Double spending theorem]
If $p>q$ then the double-spending probability is given by
$$
\psi(z) = \mathbb{P}(\tau_z<\infty)=\left(\frac{q}{p}\right)^{z}.
$$
\end{tcolorbox}
\begin{itemize}
\item The risk reserve process is $R_t=z+Y_1+\ldots+Y_t.$
\item The claim surplus process is $S_t=-(Y_1+\ldots+Y_t).$
\item $\kappa(\theta)=0$ is equivalent to
$$pe^{-\theta}+qe^{\theta}=1.$$
\begin{itemize}
\item[$\hookrightarrow$]\scriptsize $\gamma=\log(p/q).$
\end{itemize}
\item If $p>q$ then $S(t)\rightarrow - \infty$.
\item  $\xi(z)=S_{\tau_z}-z=0$ \textbf{a.s}.
\end{itemize}
Thus,
$$\mathbb{P}(\tau_z<\infty)=\left(\frac{q}{p}\right)^{z}.$$
\end{frame}
\subsection{Double spending with Poisson processes}
\begin{frame}{Double spending with Poisson processes}
\scriptsize
\underline{I.4. Double spending with Poisson processes}\\\vspace{0.5cm}
Let the length of honest and dishonest chain be driven by counting processes
\begin{itemize}
\item Honest chain $\Rightarrow$ $z+N_t\text{ , }t\geq0$, where $z\geq1$.
\item Malicious chain $\Rightarrow$ $M_t\text{ , }t\geq0$
\item Study the distribution of the first-\textit{rendez-vous} time
$$
\tau_z=\inf\{t\geq0\text{ , } M_t=z+N_t\}.
$$
\end{itemize}
% If $N_t\sim\text{Pois}(\lambda t)$ and $M_t\sim\text{Pois}(\mu t)$ such that $\lambda>\mu$ then 
% $$
% \psi(z) = \left(\frac{\mu}{\lambda}\right)^z,\text{ }z\geq 0.
% $$
\tiny
\begin{thebibliography}{1}

\bibitem{Goffard2019}
P.-O. Goffard, ``Fraud risk assessment within blockchain transactions,'' {\em
  Advances in Applied Probability}, vol.~51, pp.~443--467, jun 2019.
\newblock \url{https://hal.archives-ouvertes.fr/hal-01716687v2}.

\bibitem{Bowden2020}
R.~Bowden, H.~P. Keeler, A.~E. Krzesinski, and P.~G. Taylor, ``Modeling and
  analysis of block arrival times in the bitcoin blockchain,'' {\em Stochastic
  Models}, vol.~36, pp.~602--637, jul 2020.
\end{thebibliography}

\end{frame}
\begin{frame}{Double spending with Poisson processes}

\begin{columns}
\begin{column}{0.5\textwidth}
\scriptsize
\begin{itemize}
\item Suppose that
$$
N_t\sim\text{Pois}(\lambda t)\text{ and }M_t\sim\text{Pois}(\mu t)
$$
such that $\lambda>\mu$.
\item The risk reserve process is $R_t=z+N_t-M_t.$
\item The claim surplus process is $S_t=M_t-N_t.$
\end{itemize}
\begin{tcolorbox}[enhanced,drop shadow, title=Fact]
The difference of two Poisson processes is not a Poisson process, However it is L\'evy!
\end{tcolorbox}
\end{column}
\begin{column}{0.5\textwidth}
\begin{tikzpicture}
  %Origin and axis
  \coordinate (O) at (0,0);
  \draw[->] (-0.5,0) -- (5.5,0) coordinate[label = {above:\scriptsize$t$}] (xmax);
  \draw[->] (0,-0.5) -- (0,5) coordinate[label = {right:\scriptsize$n$}] (ymax);
 %Length of the honest chain
  \draw[thick,tublue,-] (0,3) -- (2,3) node[pos=0.5, above] {} ;
  \draw[thick,tublue] (2,3) -- (2,4) node[pos=0.5, above] {};
  \draw[thick,tublue] (2,4) -- (5.5,4) node[pos=0.5, right] {};
  % %Length of the Malicious chain
  \draw[very thick,dashed,red,-] (0,0) -- (0.75,0) node[pos=0.5, above] {} ;
  \draw[very thick,dashed,red] (0.75,0) -- (0.75,1) node[pos=0.5, right] {};
  \draw[very thick,dashed,red] (0.75,1) -- (1.25,1) node[pos=0.5, above] {};
  \draw[very thick,dashed,red] (1.25,1) -- (1.25,2) node[pos=0.5, right] {};
  \draw[very thick,dashed,red] (1.25,2) -- (2.5,2) node[pos=0.5, above] {};
  \draw[very thick,dashed,red] (2.5,2) -- (2.5,3) node[pos=0.5, right] {};
  \draw[very thick,dashed,red] (2.5,3) -- (5,3) node[pos=0.5, right] {};
  \draw[very thick,dashed,red] (5,3) -- (5,4) node[pos=0.5, above] {};
  \draw[very thick,dashed,red] (5,4) -- (5.5,4) node[pos=0.5, above] {};
  %Jump Times of the malicious chain
  \draw (0.75,0) node[red,below] {\scriptsize$S_1$} node{ \color{red}$\bullet$};
  \draw (1.25,0) node[red,below] {\scriptsize$S_2$} node{ \color{red}$\bullet$};
  \draw (2.5,0) node[red,below] {\scriptsize$S_3$} node{ \color{red}$\bullet$};
  \draw (5,0) node[black,below] {\scriptsize$S_4=\tau_z$} node{ \color{black}$\bullet$};
  % %Jump Times of the honest chain
  \draw (2,0) node[tublue,below] {\scriptsize$T_1$} node{ \color{tublue}$\bullet$};
  % %Aggregated Capital gains
  \draw (0,1) node[black,left] {\scriptsize$1$} node{ \color{black}$-$};
  \draw (0,2) node[black,left] {\scriptsize$2$} node{ \color{black}$-$};
  \draw (0,3) node[black,left] {\scriptsize$z$} node{};
  \draw (0,4) node[black,left] {\scriptsize$4$} node{ \color{black}$-$};
  % %Ruin time = First-meeting time
  % \draw (7,0) node[black,below] {$\tau_z$} node{ \color{black}$\times$};
  % \draw[dotted,black] (7,3) -- (7,0);
\end{tikzpicture}
\end{column}
\end{columns}
\end{frame}

\begin{frame}{Double spending with Poisson processes}
\scriptsize
\begin{tcolorbox}[enhanced,drop shadow, title=Double spending theorem]
If $\lambda>\mu$ then the double-spending probability is given by
$$
\psi(z) = \mathbb{P}(\tau_z<\infty)=\left(\frac{\mu}{\lambda}\right)^{z}.
$$
\end{tcolorbox}
\begin{itemize}
\item $\kappa(\theta)=0$ is equivalent to
$$
\mu e^{\theta}+\lambda e^{-\theta}-(\lambda+\mu)=0.
$$

\begin{itemize}
\item[$\hookrightarrow$] $\gamma=\log(\lambda/\mu).$
\end{itemize}
\item If $\lambda>\mu$ then $S_t\rightarrow - \infty$.
\item $\xi(z)=S_{\tau_z}-z=0$ \textbf{a.s}.
\end{itemize}
Thus
$$\mathbb{P}(\tau_z<\infty)=\left(\frac{\mu}{\lambda}\right)^{z}.$$
\begin{thebibliography}{1}

\bibitem{Goffard2019}
P.-O. Goffard, ``Fraud risk assessment within blockchain transactions,'' {\em
  Advances in Applied Probability}, vol.~51, pp.~443--467, jun 2019.
\newblock \url{https://hal.archives-ouvertes.fr/hal-01716687v2}.
\end{thebibliography}
\end{frame}
\section{Decentralization in PoS blockchain}

% \begin{frame}{Nothing-at-Stake}
% \scriptsize
% If given the opportunity a node will always append a new block
% \begin{itemize}
%   \item Everlasting fork if any
% \end{itemize}
% Perpetuating disagreement prevent users to exchange which lower the coin value.
% \begin{tcolorbox}[enhanced,drop shadow, title=Theorem]
% To get consensus faster and almost surely 
% \begin{itemize}
%   \item Set a minimum stake to outweight the benefit of the reward
%   \item Set up a modest reward schedule $\sum_{t = 1}^\infty \text{r}_t<\infty$
% \end{itemize}
% \end{tcolorbox}
% \tiny
% \begin{thebibliography}{1}

% \bibitem{Saleh2020}
% F.~Saleh, ``Blockchain without waste: Proof-of-stake,'' {\em The Review of
%   Financial Studies}, vol.~34, pp.~1156--1190, jul 2020.
% \end{thebibliography}
% \end{frame}
\begin{frame}{Risk of centralization ?}
\scriptsize
\underline{II. Decentralization in PoS blockchains}\\
\underline{II.1 Average stake owned by each peers}\\
\vspace{0.5cm}
\begin{columns}
\begin{column}{0.5\textwidth}
PoS algorithm
\begin{itemize}
  \item Draw a coin at random
  \item The owner of the coin append a block and collect the reward
  \item The block appender is more likely to get selected during the next round
\end{itemize}
\end{column}
\begin{column}{0.5\textwidth}
Similar to Polya's urn \includegraphics[scale=0.1]{../../Figures/poly_urn.png}
\begin{itemize}
  \item Consider an urn of $N$ balls of color in $E=\{1,\ldots, p\}$
  \item Draw a ball of color $x\in E$
  \item Replace the ball together with $r$ balls of color $x$ 
\end{itemize}
$p$ is the number of peers and $r$ is the size of the block reward. 
\end{column}
\end{columns}
\begin{tcolorbox}[enhanced,drop shadow, title=Theorem]
The proportion of coins owned by each peer is stable on average over the long run
\end{tcolorbox}

\tiny
\begin{thebibliography}{1}

\bibitem{Rosu2021}
I.~Ro{\c{s}}u and F.~Saleh, ``Evolution of shares in a proof-of-stake
  cryptocurrency,'' {\em Management Science}, vol.~67, pp.~661--672, feb 2021.
  \end{thebibliography}
\end{frame}
\begin{frame}{Mathematical framework}
\scriptsize
Let's consider a network $E = \{1,\ldots, p\}$ of size $p$ and $r$ be the block reward. At time $n=0$
\begin{itemize}
  \item Peer $x$ has $Z_0^{(x)}$ coins
  \item The total number of coins is given by 
  $$
Z_0 = \sum_{x\in E}Z_0^{(x)}
  $$
\end{itemize}
The number of coins owned by each peer evolves over time as 
$$
Z_n^{(x)} = Z_0^{(x)}+r\sum_{k=1}^n\mathbb{I}_{A_k^{(x)}},
$$
where 
$$
A_n^{(x)} = \text{ A coin of peer $x\in E$ is drawn during round $n\geq 1$}.
$$
The total number of coins is given by  
$$
Z_n = Z_0+nr.
$$
Let $(W_n^{(x)})_{n\geq 0}$ be the proportion of coins owned by $x\in E$, we have 
$$
W_{n}^{(x)} = \frac{Z_n^{(x)}}{Z_n}
$$
and $\mathcal{F}_n = \sigma(\{Z_{k}^{(x)},x\in E,k\leq n\})$. Note that $\mathbb{P}(A_n^{(x)}|\mathcal{F}_{n-1}) = W_{n-1}^{(x)}.$


% Consider the coins of peer $x\in E$, and denote by 
% \begin{itemize}
% \item $N_x$ the number of coins of peer $x$ initially and denote by $N =\sum_x N_x$ the total number of coins
% \item $Y_n$ the number of coins of peer $x$ after $n$ rounds
% \item $Z_n$ the corresponding proportion of coins of color $x$.
% \end{itemize}   
% We show that $(Z_n)_{n\geq0}$ is a $\mathcal{F}_n-$Martingale where $\mathcal{F}_n=\sigma(Y_1,\ldots, Y_n)$. We have 
% \begin{eqnarray*}
% \mathbb{E}(Z_{n+1}|\mathcal{F}_n) = Z_n\frac{Y_n+r}{N+r(n+1)} +(1-Z_n)\frac{Y_n}{N+r(n+1)} = Z_n
% \end{eqnarray*}
% It follows that 
% $$
% \mathbb{E}(Z_n) = \mathbb{E}(Z_0) = \frac{N_x}{N}, \text{ for }n\geq0.
% $$
% hence the stability. Furthermore, because $|Z_n|<1$, then $\underset{n\rightarrow\infty}{\lim} Z_n = Z_\infty$ exists and it holds that $\mathbb{E}(Z_\infty) = \mathbb{E}(Z_0)$.\\
\end{frame}
\begin{frame}{Proof}
\scriptsize
\begin{tcolorbox}[enhanced,drop shadow, title=Theorem]
$$
\mathbb{E}\left(W_n^{(x)}\right) = \frac{Z_0^{(x)}}{Z_0},\text{ }x\in E\text{ }, n\geq0.
$$
\end{tcolorbox}
We show that $(W_n^{(x)})_{n\geq0}$ is a martingale. We have that 
\begin{eqnarray*}
\mathbb{E}\left[W_n^{(x)}|\mathcal{F}_{n-1}\right]&=& \mathbb{E}\left[\frac{Z^{(x)}_{n-1} + r\mathbb{I}_{A_n^{(x)}}}{Z_0 + rn}\Big \rvert\mathcal{F}_{n-1}\right]\\
&=& \frac{Z^{(x)}_{n-1} }{Z_0 + rn}+\frac{rW_{n-1}^{(x)}}{Z_0 + rn}\\\
&=& \frac{W_{n-1}^{(x)}[Z_0 + r(n-1)]}{Z_0 + rn}+\frac{rW_{n-1}^{(x)}}{Z_0 + rn}\\
&=&W_{n-1}^{(x)}.
\end{eqnarray*}
It then follows that 
$\mathbb{E}\left(W_n^{(x)}\right) =W_0^{(x)}= \frac{Z_0^{(x)}}{Z_0},\text{ }x\in E\text{ }, n\geq0.$ The long term average of the stake of each peer is stable

\end{frame}
\begin{frame}{What is the limiting distributions of the shares?}
\scriptsize
\underline{II.2 Asymptotic distribution of stakes owned by each peers}\\
\begin{tcolorbox}[enhanced,drop shadow, title=Theorem (Convergence toward a Dirichlet distribution)]
Suppose that $r=1$, we have that 
\[
\left(W_\infty^{(1)},\ldots, W_\infty^{(p)}\right)\sim\text{Dir}\left(Z_0^{(1)},\ldots, Z_0^{(p)}\right).
\]
\end{tcolorbox}
\begin{tcolorbox}[enhanced,drop shadow, title=Fact]
The most desirable situation corresponds to all the peers being equally likely to be selected. 
\end{tcolorbox}
Decentrality maybe measure by Shannon's entropy
$$
H(\mu^\ast) = -\mathbb{E}\left\{\sum_x \mu^\ast(x)\ln[\mu^\ast(x)]\right\} = -\sum_x\frac{N}{N_x}\left[\psi(N_x+1)-\psi(N+1)\right],
$$
where $\psi(x) = \frac{\text{d}}{\text{d}x}\ln[\Gamma(x)]$ is the digamma function, to be compared to $\ln(p)$.
\end{frame}
% \begin{frame}{Measuring decentrality}
% \scriptsize

% \tiny
% \begin{thebibliography}{1}

% \bibitem{Gochhayat2020}
% S.~P. Gochhayat, S.~Shetty, R.~Mukkamala, P.~Foytik, G.~A. Kamhoua, and
%   L.~Njilla, ``Measuring decentrality in blockchain based systems,'' {\em
%   {IEEE} Access}, vol.~8, pp.~178372--178390, 2020.

% \end{thebibliography}

% \end{frame}
\begin{frame}{Extensions and perspectives}
\begin{itemize}
  \item How to include more peers along the way?
  \item What if the peers are not simply buy and hold investors?
  \item Find ways to monitor decentralization and take action if necessary
\end{itemize}
\tiny
\begin{thebibliography}{1}

\bibitem{Rosu2021}
I.~Ro{\c{s}}u and F.~Saleh, ``Evolution of shares in a proof-of-stake
  cryptocurrency,'' {\em Management Science}, vol.~67, pp.~661--672, feb 2021.
  \end{thebibliography} 

\end{frame}
\section{Blockchain efficiency}

\begin{frame}{Efficiency}
\scriptsize
\underline{III. Efficiency of PoW blockchains}\\
\vspace{0.5cm}
Efficiency is characterized by 
\begin{itemize}
  \item Throughputs: Number of transaction being processed per time unit
  \item Latency: Average transaction confirmation time
\end{itemize}
\end{frame}
\begin{frame}{Efficiency}
\underline{III.1 A queueing model with bulk service}\\
\begin{center}
\begin{tikzpicture}[-, >=stealth', auto, semithick, node distance=1cm]

\tikzstyle{phantom block}=[rectangle, fill=white,draw=white, thick,text=black,scale=2]
\tikzstyle{block}=[rectangle, fill=white,draw=black,thick,text=black,scale=4]
\tikzstyle{Intensity}=[circle, fill=white,draw=tublue,very thick, text=black,scale=1.2]
\tikzstyle{transaction pending}=[circle, fill=white,draw=tublue,very thick, text=black,scale=1]
\tikzstyle{transaction considered}=[circle, fill=tublue, text=black, scale=1]
\node[Intensity]    (1){$\lambda$};
\node[phantom block]  (2)[right of=1] {};
\node[transaction pending] (3)[right of=2] {};
\node[transaction pending] (4)[above of=3] {};
\node[transaction pending] (5)[above of=4] {};
\node[transaction pending] (6)[above of=5] {};
\node[transaction pending] (7)[below of=3] {};
\node[transaction pending] (8)[below of=7] {};
\path
(1) edge[->,bend left]     node{} (4)
    edge[->, bend right]     node{}          (7);

\node[Intensity]  (14)[above of=6]{$\mu$};
% \path
% (1) edge[->,bend left]     node{$\text{Exp}(\lambda)$}        (4)
%     edge[->, bend right]     node{}          (7);

\node[transaction considered] (3)[right of=2] {};
\node[transaction considered] (4)[above of=3] {};
\node[transaction considered] (5)[above of=4] {};
\node[transaction considered] (6)[above of=5] {};


\node[phantom block]  (10)[right of=3] {};
\node[block]  (11)[right of=10] {};
\path
(6) edge[->]   node{} (11)
(5) edge[->]   node{} (11)
(4) edge[->]   node{} (11)
(3) edge[->]   node{} (11)
;
\node[Intensity]  (12)[above of=11] {$\mu$};
\node[phantom block]  (13)[above of=12] {};

\path
(11) edge[-]     node{}        (12);
\path
(12) edge[->]     node{}        (13);

\end{tikzpicture}
\end{center}
\end{frame}
\begin{frame}{Queueing setting}

\begin{itemize}
\item Poisson arrival with rate $\lambda>0$ for the transactions
\item Poisson arrival with rate $\mu>0$ for the blocks 
\item Block size $b\in\mathbb{N}^\ast$
$\Rightarrow $Batch service
\item[\warning] The server is always busy
\end{itemize}
This is somekind of $M/M^b/1$ queue.
\tiny
\begin{thebibliography}{1}
\bibitem{Kawase2017}
Y.~Kawase and S.~Kasahara, ``Transaction-confirmation time for bitcoin:
  A~queueing analytical approach to~blockchain~mechanism,'' in {\em Queueing
  Theory and Network Applications}, pp.~75--88, Springer International
  Publishing, 2017.
\bibitem{Bailey1954}
N.~T.~J. Bailey, ``On queueing processes with bulk service,'' {\em Journal of
  the Royal Statistical Society: Series B (Methodological)}, vol.~16,
  pp.~80--87, jan 1954.

\bibitem{Cox1955}
D.~R. Cox, ``The analysis of non-markovian stochastic processes by the
  inclusion of supplementary variables,'' {\em Mathematical Proceedings of the
  Cambridge Philosophical Society}, vol.~51, pp.~433--441, jul 1955.

\end{thebibliography}
\end{frame}

\begin{frame}{Queue length distribution}
\scriptsize
The queueuing process eventually reaches stationarity if 
\begin{equation}\label{eq:stationarity_cond}
\mu\cdot b > \lambda.
\end{equation}
We denote by $N^q$ the length of the queue upon stationarity. 
\begin{tcolorbox}[enhanced,drop shadow, title=The blockchain efficiency theorem]
Assume that \eqref{eq:stationarity_cond} holds then $N^q$ is geometrically distributed 
$$
\mathbb{P}(N^q = n) = (1-p)\cdot p^n,
$$
where $p = 1/z^\ast$ and $z^\ast$ is the only root of 
$$
-\frac{\lambda}{\mu}z^{b+1}+z^b\left(\frac{\lambda}{\mu}+1\right) - 1,
$$
such that $|z^\ast$|>1.  
\end{tcolorbox}
\end{frame}
\begin{frame}{Latency and throughputs}
\scriptsize
\underline{III.2 Throughputs and latency computation}\\
\begin{tcolorbox}[enhanced,drop shadow, title=Little's law]
Consider a stationary queueing system and denote by 
\begin{itemize}
  \item $1/\lambda$ the mean of the unit inter-arrival times
  \item $L$ be the mean number of units in the system
  \item $W$ be the mean time spent by units in the system
\end{itemize}
We have
$$
L = \lambda \cdot W
$$
\end{tcolorbox}
\tiny
\begin{thebibliography}{1}

\bibitem{Little1961}
J.~D.~C. Little, ``A proof for the queuing formula:{L}= $\lambda${W},'' {\em
  Operations Research}, vol.~9, pp.~383--387, jun 1961.

\end{thebibliography}
\scriptsize
\begin{itemize}
  \item Latency is the confirmation time of a transaction 
    $$
    \text{Latency} = W + \frac{1}{\mu} = \frac{\mathbb{E}(N^q)}{\lambda} + \frac{1}{\mu} =  \frac{p}{(1-p)\lambda} + \frac{1}{\mu}.
    $$
  \item Throughput is the number of transaction confirmed per time unit
  $$
    \text{Throughput} = \mu\mathbb{E}(N^q\mathbb{I}_{N^q\leq b}+b\mathbb{I}_{N^q> b}) = \mu\sum_{n = 0}^bn(1-p)p^n + bp^{b+1}.
  $$
\end{itemize}
\end{frame}
\begin{frame}{Perspective}
\begin{enumerate}
  \item  Include some priority consideration to account for the transaction fees 
  \tiny 
  \begin{thebibliography}{1}

\bibitem{Kawase2020}
Y.~Kawase, , and S.~Kasahara, ``Priority queueing analysis of
  transaction-confirmation time for bitcoin,'' {\em Journal of Industrial {\&}
  Management Optimization}, vol.~16, no.~3, pp.~1077--1098, 2020.

\end{thebibliography}

  \item \normalsize Go beyond the Poisson process framework
  \tiny
  \begin{thebibliography}{1}

\bibitem{Li2018}
Q.-L. Li, J.-Y. Ma, and Y.-X. Chang, ``Blockchain queue theory,'' in {\em
  Computational Data and Social Networks}, pp.~25--40, Springer International
  Publishing, 2018.

\bibitem{Li2019}
Q.-L. Li, J.-Y. Ma, Y.-X. Chang, F.-Q. Ma, and H.-B. Yu, ``Markov processes in
  blockchain systems,'' {\em Computational Social Networks}, vol.~6, jul 2019.

\end{thebibliography}
\end{enumerate}

\end{frame}
\appendix
\begin{frame}{Refinements of the double spending problem}
\scriptsize
Let the length of honest and dishonest chain be driven by counting processes
\begin{itemize}
\item Honest chain $\Rightarrow$ $z+N_t\text{ , }t\geq0$, where $z\geq1$.
\item Malicious chain $\Rightarrow$ $M_t\text{ , }t\geq0$
\item Study the distribution of the first-\textit{rendez-vous} time
$$
\tau_z=\inf\{t\geq0\text{ , } M_t=z+N_t\}.
$$
\end{itemize}
If $N_t\sim\text{Pois}(\lambda t)$ and $M_t\sim\text{Pois}(\mu t)$ such that $\lambda>\mu$ then 
$$
\psi(z) = \left(\frac{\mu}{\lambda}\right)^z,\text{ }z\geq 0.
$$
\tiny
\begin{thebibliography}{1}

\bibitem{Goffard2019}
P.-O. Goffard, ``Fraud risk assessment within blockchain transactions,'' {\em
  Advances in Applied Probability}, vol.~51, pp.~443--467, jun 2019.
\newblock \url{https://hal.archives-ouvertes.fr/hal-01716687v2}.

\bibitem{Bowden2020}
R.~Bowden, H.~P. Keeler, A.~E. Krzesinski, and P.~G. Taylor, ``Modeling and
  analysis of block arrival times in the bitcoin blockchain,'' {\em Stochastic
  Models}, vol.~36, pp.~602--637, jul 2020.
\end{thebibliography}

\end{frame}
\begin{frame}{Dirichlet distribution}
\scriptsize
\begin{tcolorbox}[enhanced,drop shadow, title=Dirichlet distribution]
A random vector $(Z_1,\ldots, Z_p)$ has a Dirichlet distribution $\text{Dir}(\alpha_1,\ldots, \alpha_p)$ with \pdf
$$
f(z_1,\ldots, z_p;\alpha_1,\ldots, \alpha_p) = \frac{1}{B(\alpha)}\prod_{i=1}^p z_i^{\alpha_i-1}, 
$$
for $\alpha_1,\ldots, \alpha_p>0$, $0< z_1,\ldots, z_p <1$ and $\sum_{i=1}^pz_i=1$, where 
$$
B(\alpha) = \frac{\prod_{i = 1}^p \Gamma(\alpha_i)}{\Gamma(\sum_{i=1}^p \alpha_i)}.
$$
\end{tcolorbox}
\end{frame}
\begin{frame}[allowframebreaks]{Proof}
\scriptsize
We have that 
\begin{equation}\label{eq:polya_sequence_1}
\mathbb{P}(X_1=x) = \frac{N_x}{N}
\end{equation}
and 
\begin{equation}\label{eq:polya_sequence_2}
\mathbb{P}(X_{n+1}=x) = \frac{N_x + \sum_{i=1}^n\delta_{X_i}(x)}{N+n} = \frac{N_x + \lambda_n(x)}{N+n} = m_n(x)
\end{equation}
where $\delta_{X_i}$ denotes the Dirac measure at $X_i$.\\
A sequence that satisfies \eqref{eq:polya_sequence_1} and \eqref{eq:polya_sequence_2} is said to be a Polya sequence with parameter $N_x\text{, }x\in E$.

\begin{lemma}
There is an equivalence between the two following statements
\begin{itemize}
\item[(i)] $X_1,X_2,\ldots,$ is a Polya sequence
\item[(ii)] $\mu^{\ast}\sim \text{Dir}(N_x,x\in E)$ and $X_1,X_2,\ldots$ given $\mu^\ast$ are \iid as $\mu^\ast$
\end{itemize}
\end{lemma}
Consider the event $A_n = \{X_1 = x_1,\ldots, X_n = x_n\}$. Induction on $n$ allows us to show that (i) is equivalent to 
\begin{equation}\label{eq:P_A_polya_i}
\mathbb{P}(A_n) = \frac{\prod_{x\in E} N_x^{[\lambda_n(x)]}}{N^{[n]}},
\end{equation}
where $\lambda_n(x)$ is the number of $i$'s in $1,\ldots, n$ for which $x_i = x$ and $a^{[k]} = a(a+1)\ldots(a+k-1)$.  Now assume that $(ii)$ holds true, then 
$$
\mathbb{P}(A_n|\mu^\ast) = \prod_{x\in E}\mu^\ast(x)^{\lambda_n(x)},
$$
recall that $\mu^\ast$ is a random vector, indexed on $E$, We denote by $\mu^\ast(x)$ the component associated with $x\in E$. The law of total probability then yields
\begin{equation}\label{eq:P_A_polya_ii}
\mathbb{P}(A_n) = \E\left[\prod_{x\in E}\mu^\ast(x)^{\lambda_n(x)}\right],
\end{equation}
which is the same as \eqref{eq:P_A_polya_i}. Applying the lemma together with the law of large number yields 
$$
n^{-1}\sum_{i=1}^n\delta_{X_i}(x) \rightarrow \mu^{\ast}(x)\text{ as } n\rightarrow\infty.
$$
and then $m_n(x)\rightarrow\mu^{\ast}(x)$.
\tiny
\begin{thebibliography}{1}
\bibitem{Blackwell1973}
D.~Blackwell and J.~B. MacQueen, ``Ferguson distributions via polya urn
  schemes,'' {\em The Annals of Statistics}, vol.~1, mar 1973.
  \end{thebibliography}
\end{frame}
\begin{frame}[allowframebreaks]{Proof of the efficiency theorem}
\scriptsize
Let $N^q_t$ be the number of transactions in the queue at time $t\geq0$ and $X_t$ the time elapsed since the last block was found. Further define
\[
P_{n}(x,t)\text{d}x  =\mathbb{P}[N_t^q = n, X_t \in(x, x + \text{dx})] 
\]
If $\lambda < \mu\cdot b$ holds then the process admits a limiting distribution given by 
\[
\underset{t\rightarrow\infty}{\lim}P_{n}(x,t) = P_{n}(x).
\]
We aim at finding the distribution of the queue length upon stationarity
\begin{equation}\label{eq:alpha_n}
\mathbb{P}(N^q=n):=\alpha_n =\int_{0}^\infty P_{n}(x)\text{d}x.
\end{equation}
Consider the possible transitions over a small time lapse \text{h} during which no block is being generated. Over this time interval, either 
\begin{itemize}
  \item no transactions arrives
  \item one transaction arrives
\end{itemize}
We have for $n\geq1$
\[
P_{n}(x+h) = e^{-\mu h}\left[e^{-\lambda h}P_{n}(x)+\lambda h e^{-\lambda h}P_{n-1}(x)\right]
\]
Differentiating with respect to $h$ and letting $h\rightarrow0$ leads to 
\begin{equation}\label{eq:diff_eq_n_geq_1}
P_{n}'(x) = -(\lambda+\mu)P_{n}(x)+\lambda P_{n-1}(x),\text{ }n \geq1.
\end{equation}
Similarly for $n = 0$, we have 
\begin{equation}\label{eq:diff_eq_n_eq_0}
P_{0}'(x) = -(\lambda+\mu)P_{0}(x).
\end{equation}
We denote by 
$$
\xi(x)\text{d}x =\mathbb{P}(x\leq X< x+\text{d}x|X\geq x)= \mu\text{d}x
$$
the hazard function of the block arrival time (constant as it is exponentially distributed). The system of differential equations \eqref{eq:diff_eq_n_geq_1}, \eqref{eq:diff_eq_n_eq_0} admits boundary conditions at $x = 0$ with 
\begin{equation}\label{eq:boundary_cond_1}
\begin{cases}
P_{n}(0) = \int_0^{+\infty} P_{n+b}(x)\xi(x)\text{d}x = \mu\alpha_{n+b},&n \geq1,\\
P_{0}(0) = \mu\sum_{n=0}^{b}\alpha_n,&n = 0,\ldots,b\\
\end{cases}
\end{equation}
Define the probability generating function of $N^q$ at some elapsed service time $x\geq 0$ as 
$$
G(z;x) = \sum_{n=0}^\infty P_{n}(x)z^n.
$$
By differentiating with respect to $x$, we get (using \eqref{eq:diff_eq_n_geq_1} and \eqref{eq:diff_eq_n_eq_0})
$$
\frac{\partial}{\partial x}G(z;x) = -\left[\lambda(1-z)+\mu\right]G(z;x)
$$
and therefore
$$
G(z;x) = G(z;0)\exp\left\{-\left[\lambda(1-z)+\mu\right]x\right\}
$$
We get the probability generating function of $N^q$ by integrating over $x$ as 
\begin{equation}\label{eq:G_z_solve_ODE}
G(z) = \frac{G(z;0)}{\lambda(1-z)+\mu}
\end{equation}
Using the boundary conditions \eqref{eq:boundary_cond_1}, we write 
\begin{eqnarray}
G(z;0) &= &\sum_{n = 0}^\infty P_{n}(0)z^n \nonumber\\
&= &P_{0}(0)+\sum_{n=1}^{+\infty}P_{n}(0)z^n\nonumber\\
&=& \mu\sum_{n = 0}^{b}\alpha_n  + \mu\sum_{n=1}^{+\infty}\alpha_{n+b} z^n\nonumber\\
&=& \mu\sum_{n = 0}^{b}\alpha_n + \mu z^{-b}\left[G(z)-\sum_{n = 0}^{b}\alpha_n z^n\right]\label{eq:G_z_0}
\end{eqnarray}
Replacing the left hand side of \eqref{eq:G_z_0} by \eqref{eq:G_z_solve_ODE}, multiplying on both side by $z^b$ and rearranging yields 
\begin{equation}\label{eq:G_z_as_rational_function}
\frac{G(z)}{M(z)}[z^b - M(z)] =\sum_{n=0}^{b-1}\alpha_n(z^b - z^n), 
\end{equation}
where $M(z) = \mu/(\lambda(1-z)+\mu)$. Using Rouche's theorem, we find that both side of the equation shares $b$ zeros inside the circle $\mathcal{C} = \{z\in\mathbb{C}\text{ ; }|z| <1+\epsilon\}$ for some epsilon. 
\begin{tcolorbox}[enhanced,drop shadow, title=Rouche's theorem]
Let $\mathcal{C}\in \mathbb{C}$ and $f$ and $g$ two holomorphic functions on $\mathcal{C}$. Let $\partial\mathcal{C}$ be the contour of $\partial\mathcal{C}$. If 
$$
|f(z)-g(z)|<|g(z)|\text{, }\forall z\in\partial\mathcal{C}
$$ 
then $Z_f-P_f = Z_g-P_g$, where $Z_f$, $P_f$, $Z_g$, and $P_g$ are the number of zeros and poles of $f$ and $g$ respectively.

\end{tcolorbox}
We have $\partial\mathcal{C} =\{z\in\mathbb{C}\text{ ; }|z| =1+\epsilon\}$. The left hand side can be rewritten as
$$
G(z)\left[-\frac{\lambda}{\mu}z^{b+1} + \left(1 + \frac{\lambda}{\mu}\right)z^b -1\right].
$$
Define $f(z) = -\frac{\lambda}{\mu}z^{b+1} + \left(1 + \frac{\lambda}{\mu}\right)z^b -1$ and $g(z)=\left(1 + \frac{\lambda}{\mu}\right)z^b$. We have 
$$
|f(z) - g(z)| = |-\frac{\lambda}{\mu}z^{b+1}-1|< \frac{\lambda}{\mu}(1+\epsilon)^{b+1}+1\rightarrow \frac{\lambda}{\mu}+1,\text{ as }\epsilon \rightarrow 0. 
$$
and 
$$
|g(z)| = \left(1 + \frac{\lambda}{\mu}\right)(1+\epsilon)^b\rightarrow \frac{\lambda}{\mu}+1,\text{ as }\epsilon \rightarrow 0. 
$$
Regarding the right hand side, define $f(z) = \sum_{n=0}^{b-1}\alpha_n(z^b - z^n)$ and $g(z) =\sum_{n=0}^{b-1}\alpha_nz^b $. We have 
$$
|f(z) - g(z)| < |\sum_{n=0}^{b-1}\alpha_n z^n| < \sum_{n=0}^{b-1}\alpha_n (1+\epsilon)^n\rightarrow  \sum_{n=0}^{b-1}\alpha_n,\text{ as }\epsilon \rightarrow 0.
$$
and 
$$
|g(z)| = (1+\epsilon)^b\sum_{n=0}^{b-1}\alpha_n\rightarrow \sum_{n=0}^{b-1}\alpha_n,\text{ as }\epsilon \rightarrow 0.
$$
One of them is $1$, and we denote by $z_k$, $k = 1,\ldots, b-1$ the remaining $b-1$ zeros. Given the polynomial form of the right hand side of \eqref{eq:G_z_as_rational_function}, the fundamental theorem of algebra indicates that the number of zero is $b$. Given the left hand side 
$$
G(z)\left[-\frac{\lambda}{\mu}z^{b+1} + \left(1 + \frac{\lambda}{\mu}\right)z^b -1\right].
$$
we deduce that there is one zeros outside $\mathcal{C}$, we can further show that it is a real number $z^\ast$. Multiplying both side of \eqref{eq:G_z_as_rational_function} by $(z-1)\prod_{k =1}^{b-1}(z-z_k)$, and using $G(1)=1$ yields
$$
G(z) = \frac{1-z^\ast}{z-z^{\ast}}.
$$
$N^q$ is then a geometric random variable with parameter $p = \frac{1}{z^\ast}.$
\end{frame}


\end{document}
