\documentclass{beamer}
\usepackage[utf8]{inputenc}
\usepackage[T1]{fontenc}
% \usepackage{amscd, amsfonts, amsmath, amssymb, amstext, amsthm, caption, epsfig, fancyhdr, float, graphicx, latexsym, mathtools, multicol, multirow, algorithm, chngcntr}
\usepackage[english, french]{babel}
\usepackage{booktabs}

\usepackage{amsmath,amssymb}
\usepackage{graphicx}
\usepackage{caption}
\usepackage{subfig}
\usepackage{xspace}
\usepackage{fourier}

\usepackage{tikz}
\usetikzlibrary{shapes,arrows}
\usepackage{tkz-graph}
\usetikzlibrary{automata,arrows,positioning,calc}
\usetikzlibrary{positioning}
\usetikzlibrary{fit}
\usetikzlibrary{backgrounds}
\usetikzlibrary{calc}
\usetikzlibrary{shapes}
\usetikzlibrary{mindmap}
\usetikzlibrary{decorations.text}

% \theoremstyle{definition} % insert bellow all blocks you want in normal text
% \newtheorem{definition}{Definition}



% tikzmark command, for shading over items
\newcommand{\tikzmark}[1]{\tikz[overlay,remember picture] \node (#1) {};}
% Define block styles
\tikzstyle{decision} = [diamond, draw, fill=blue!20,
    text width=4.5em, text badly centered, node distance=3cm, inner sep=0pt]
\tikzstyle{block} = [rectangle, draw, fill=blue!20,
    text width=5em, text centered, rounded corners]
\tikzstyle{line} = [draw]
\tikzstyle{cloud} = [draw, ellipse,fill=red!20, node distance=3cm,
    minimum height=2em]

\usepackage[most]{tcolorbox}

\setbeamertemplate{blocks}[rounded][shadow=true] % use rounded blocks with standard beamer shadow

\newcommand*{\warning}{\fontencoding{U}\fontfamily{futs}\selectfont\char 66\relax}

% Distributions.
\newcommand*{\UnifDist}{\mathsf{Unif}}
\newcommand*{\ExpDist}{\mathsf{Exp}}
\newcommand*{\DepExpDist}{\mathsf{DepExp}}
\newcommand*{\GammaDist}{\mathsf{Gamma}}
\newcommand*{\LognormalDist}{\mathsf{LogNorm}}
\newcommand*{\WeibullDist}{\mathsf{Weib}}
\newcommand*{\ParetoDist}{\mathsf{Par}}
\newcommand*{\NormalDist}{\mathsf{Norm}}

\newcommand*{\GeometricDist}{\mathsf{Geom}}
\newcommand*{\NegBinomialDist}{\mathsf{NegBin}}
\newcommand*{\PoissonDist}{\mathsf{Poisson}}
\newcommand*{\BivariatePoissonDist}{\mathsf{BPoisson}}
\newcommand*{\CyclicalPoissonDist}{\mathsf{CPoisson}}

\newcommand*{\iid}{\textbf{iid}\@\xspace}
\newcommand*{\pdf}{\textbf{pdf}\@\xspace}
\newcommand*{\cdf}{\textbf{cdf}\@\xspace}
\newcommand*{\pmf}{\textbf{pmf}\@\xspace}
\newcommand*{\abc}{{\textbf{abc}}\@\xspace}
\newcommand*{\smc}{\textbf{smc}\@\xspace}
\newcommand*{\mcmc}{\textbf{mcmc}\@\xspace}
\newcommand*{\ess}{\textbf{ess}\@\xspace}
\newcommand*{\mle}{\textbf{mle}\@\xspace}
\newcommand*{\bic}{\textbf{bic}\@\xspace}
\newcommand*{\kde}{\textbf{kde}\@\xspace}
\newcommand*{\glm}{\textbf{glm}\@\xspace}
\newcommand*{\xol}{\textbf{xol}\@\xspace}
\newcommand*{\cpu}{\textbf{cpu}\@\xspace}
\newcommand*{\gpu}{\textbf{gpu}\@\xspace}
\newcommand*{\arm}{\textbf{arm}\@\xspace}

\def \si {\sigma}
\def \la {\lambda}
\def \al {\alpha}
% \def\e*{\end{eqnarray*}}
\def \di{\displaystyle}

\def \E{\mathbb E}
\def \N{\mathbb N}
\def \Z{\mathbb Z}
\def \NZ{\mathbb{N}_0}
\def \I{\mathbb I}
\def \w{\widehat}
\def \P {\mathbb P}
\def \V{\mathbb V}


\newcommand{\CL}{\mathbb{C}}
\newcommand{\RL}{\mathbb{R}}
\newcommand{\nat}{{\mathbb N}}
\newcommand{\Laplace}{\mathscr{L}}
\newcommand{\e}{\mathrm{e}}
\newcommand{\ve}{\bm{\mathrm{e}}} % vector e

\renewcommand{\L}{\mathcal{L}} % e.g. L^2 loss.

\newcommand{\ih}{\mathrm{i}}
\newcommand{\oh}{{\mathrm{o}}}
\newcommand{\Oh}{{\mathcal{O}}}
\newcommand{\Exp}{\mathbb{E}}

\newcommand{\Norm}{\mathcal{N}}
\newcommand{\LN}{\mathcal{LN}}
\newcommand{\SLN}{\mathcal{SLN}}

\renewcommand{\Pr}{\mathbb{P}}
\newcommand{\Ind}{\mathbb I}
\newcommand\bfsigma{\bm{\sigma}}
\newcommand\bfSigma{\bm{\Sigma}}
\newcommand\bfLambda{\bm{\Lambda}}
\newcommand{\stimes}{{\times}}
\def \limsup{\underset{n\rightarrow+\infty}{\overline{\lim}}}
\def \liminf{\underset{n\rightarrow+\infty}{\underline{\lim}}}




% vertical separator macro
\newcommand{\vsep}{
  \column{0.0\textwidth}
    \begin{tikzpicture}
      \draw[very thick,black!10] (0,0) -- (0,7.3);
    \end{tikzpicture}
}
\newcommand\blfootnote[1]{%
  \begingroup
  \renewcommand\thefootnote{}\footnote{#1}%
  \addtocounter{footnote}{-1}%
  \endgroup
}

% More space between lines in align
% \setlength{\mathindent}{0pt}

% Beamer theme
\usetheme{ZMBZFMK}
\usefonttheme[onlysmall]{structurebold}
\mode<presentation>
\setbeamercovered{transparent=10}

% align spacing
\setlength{\jot}{0pt}

\setbeamertemplate{navigation symbols}{}%remove navigation symbols

\title[BLOCKASTICS III]{Stochastic Models for blockchain analysis}
\subtitle{Blockchain risk analysis}
\author{Pierre-O. Goffard}
\institute[ISFA]{Institut de Science Financières et d'Assurances\\
 \texttt{pierre-olivier.goffard@univ-lyon1.fr}}
\date{\today}
\titlegraphic{\includegraphics[width=2.5cm]{../../Figures/bfs_logo.png}} 

\begin{document}
\begin{frame}
  \titlepage
\end{frame}
\begin{frame}{Blockchain risk analysis}
\tableofcontents
% \begin{enumerate}
%   \item Security of PoW blockchain
%   \item Decentralization in PoS blockchain
%   \item Blockhain efficiency
% \end{enumerate}
\end{frame}
\section{Insurance risk theory}
\begin{frame}{Cramer-Lunberg model}
The financial reserves of an insurance company over time have the following dynamic
\begin{equation*}
R_t = u +ct - \sum_{i = 1}^{N_t}U_i\text{, }t\geq0,
\end{equation*}
where 
\begin{itemize}
  \item $u>0$ denotes the initial reserves
  \item $c$ is the premium rate
  \item $(N_t)_{t\geq0}$ is a counting process that models the claim arrival 
  \begin{itemize}
    \item[$\hookrightarrow$] Poisson process with intensity $\lambda$
  \end{itemize}
  \item The $U_i$'s are the randomly sized compensations
  \begin{itemize}
    \item[$\hookrightarrow$] non-negative, \textbf{i.i.d.}
  \end{itemize}
\end{itemize}
\end{frame}
\begin{frame}{Ruin probabilities}
Define the ruin time as 
$$
\tau_u = \inf\{t\geq0\text{ ; }R_t <0\}
$$
and the ruin probabilities as 
$$
\psi(u,t) = \mathbb{P}(\tau_u \leq t)\text{ and }\psi(u) = \mathbb{P}(\tau_u \leq \infty)
$$
We look for $u$ such that 
$$
\mathbb{P}(\text{Ruin}) = \alpha\text{ (0.05)},
$$
given that 
$$
c=(1+\eta)\lambda\mathbb{E}(U),
$$
with $\eta>0$.

\scriptsize
\begin{thebibliography}{1}

\bibitem{Asmussen_2010}
S.~Asmussen and H.~Albrecher, {\em Ruin Probabilities}.
\newblock {WORLD} {SCIENTIFIC}, sep 2010.

\end{thebibliography}

\end{frame}

\begin{frame}{Wald's Martingale}
Let 
$$
S_t = z - R_t,\text{ }t\geq0
$$
\begin{theorem}[Wald exponential Martingale]
If $\{S_t\text{ , }t\geq0\}$ is a L\'evy process or a random walk then
$$
\{\exp\left[\theta S_t-t\kappa(\theta)\right]\text{ , }t\geq0\},\text{ is a martingale,}
$$
where $\kappa(\theta)=\log\mathbb{E}\left(e^{\theta S_1}\right)$.
\end{theorem}
\end{frame}
\begin{frame}[allowframebreaks]{Proof}
\end{frame}
\begin{frame}{Ruin probability computation}
\begin{theorem}[Representation of the ruin probability]
If
\begin{itemize}
\item $S_t\overset{\textbf{a.s.}}{\rightarrow} -\infty$,
\item There exists $\gamma>0$ such that $\{e^{\gamma S_t}\text{ , }t\geq0\}$ is a martingale
\end{itemize}
then
$$
\mathbb{P}(\tau_z<\infty)=\frac{e^{-\gamma z}}{\mathbb{E}\left[e^{\gamma \xi(z)}|\tau_z<\infty\right]},
$$
where
$$
\xi(z)=S_{\tau_z}-z\text{ denotes the deficit at ruin.}
$$
\end{theorem}
\end{frame}
\begin{frame}[allowframebreaks]{Proof}
\end{frame}
\section{Link to double spending}
\begin{frame}{Double spending in Satoshi's framework}
\begin{itemize}
\item The risk reserve process is $R_t=z+Y_1+\ldots+Y_t.$
\item The claim surplus process is $S_t=-(Y_1+\ldots+Y_t).$
\item $\kappa(\theta)=0$ is equivalent to
$$pe^{-\theta}+qe^{\theta}=1.$$
\begin{itemize}
\item[$\hookrightarrow$] $\gamma=\log(p/q).$
\end{itemize}
\item If $p>q$ then $S(t)\rightarrow - \infty$.
\item  $\xi(z)=S_{\tau_z}-z=0$ \textbf{a.s}.
\end{itemize}
Thus,
$$\mathbb{P}(\tau_z<\infty)=\left(\frac{q}{p}\right)^{z}.$$
\end{frame}

\begin{frame}{Double spending with Poisson processes}

\begin{itemize}
\item Suppose that
$$
N_t\sim\text{Pois}(\lambda t)\text{ and }M_t\sim\text{Pois}(\mu t)
$$
such that $\lambda>\mu$.
\item The risk reserve process is $R_t=z+N_t-M_t.$
\item The claim surplus process is $S_t=M_t-N_t.$
\end{itemize}
\begin{tcolorbox}[enhanced,drop shadow, title=Fact]
The difference of two Poisson processes is not a Poisson process, However it is L\'evy!
\end{tcolorbox}
\end{frame}
\begin{frame}{Double spending with Poisson processes}
\begin{itemize}
\item $\kappa(\theta)=0$ is equivalent to
$$
\mu e^{\theta}+\lambda e^{-\theta}-(\lambda+\mu)=0.
$$

\begin{itemize}
\item[$\hookrightarrow$] $\gamma=\log(\lambda/\mu).$
\end{itemize}
\item If $\lambda>\mu$ then $S(t)\rightarrow - \infty$.
\item $\xi(z)=S_{\tau_z}-z=0$ \textbf{a.s}.
\end{itemize}
Thus
$$\mathbb{P}(\tau_z<\infty)=\left(\frac{\mu}{\lambda}\right)^{z}.$$
\end{frame}

\begin{frame}{Double spending cost}
Mining cryptocurrency in PoW equipped blockchain is energy consuming
\begin{itemize}
\item[$\hookrightarrow$] Operational cost for miners
\end{itemize}
Per time unit a miner pays
$$
c = \pi_W\cdot W\cdot p
$$
where 
\begin{itemize}
  \item $\pi_W$ is the electricty price per kWh
  \item $W$ is the consumption of the network \url{https://cbeci.org/}
  \item  $p$ is the miner's hashpower 
\end{itemize}
The cost of double spending is $c\cdot \tau_z$
\end{frame}
\begin{frame}{Finite horizon double spending}
\begin{theorem}[\textbf{p.d.f.} of the double spending time]
If $\{N_t\text{ , }t\geq0\}$ is a Poisson process then the \textbf{p.d.f.} of $\tau_z$ is given by
\begin{equation*}
f_{\tau_z}(t)=\mathbb{E}\left[\frac{z}{z+N(t)}f_{\Delta^{S}}^{\ast[N(t)+z]}(t)\right],\text{ for }t\geq0.
\end{equation*}
\end{theorem}
\end{frame}
\begin{frame}[allowframebreaks]{Proof}
\end{frame}
\section{Link to blockhain mining}
\begin{frame}{Dual risk model}
A blockchain miner with hashpower share $p\in(0,1)$ that
\begin{itemize}
  \item owns $u\geq0$ at the beginning 
  \item spend $c = \pi_W\cdot W\cdot p$ per time unit
  \item finds block at a rate $p \lambda$, where $\lambda$ is the arrival rate of blocks
\end{itemize}
The miner's surplus is given by 
$$
R_t = u - c\cdot t + N_t\cdot b,\text{ (Dual risk model)}
$$
where 
\begin{itemize}
  \item $(N_t)_{t\geq0}$ is a Poisson process with intensity $p\cdot\lambda$
  \item $b$ is the block finding reward (6.25 BTC) \url{bitcoinhalf.com}
\end{itemize}

\end{frame}
\begin{frame}{Expected profit given not ruin}
\begin{tcolorbox}[enhanced,drop shadow, title=Fact]
The steady operational cost compensated by infrequent capital gains makes mining a risky business.
\end{tcolorbox}
Define the ruin time 
$$
\tau_u  = \inf\{t\geq0\text{ ; }R_t <0\}
$$
\begin{itemize}
  \item Risk measure 
  $$
  \psi(u,t) = \mathbb{P(\tau_u \geq t)}
  $$
  \item Profitability measure
  $$
  V(u,t) = \mathbb{E}(R_t\mathbb{I}_{\tau_u > t})
  $$
\end{itemize} 
\end{frame}
\begin{frame}{Miner's dilemna} 
\begin{itemize}
  \item Joining a mining pool
  \item Deviating from the protocol (selfish mining)
\end{itemize}
\end{frame}

\begin{frame}
\bibliography{../../blockastics}
\bibliographystyle{ieeetr}
\end{frame}

\end{document}
