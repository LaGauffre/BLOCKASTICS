\documentclass{beamer}
\usepackage[utf8]{inputenc}
\usepackage[T1]{fontenc}
% \usepackage{amscd, amsfonts, amsmath, amssymb, amstext, amsthm, caption, epsfig, fancyhdr, float, graphicx, latexsym, mathtools, multicol, multirow, algorithm, chngcntr}
\usepackage[english, french]{babel}
\usepackage{booktabs}

\usepackage{amsmath,amssymb}
\usepackage{graphicx}
\usepackage{caption}
\usepackage{subfig}
\usepackage{xspace}
\usepackage{fourier}

\usepackage{tikz}
\usetikzlibrary{shapes,arrows}
\usepackage{tkz-graph}
\usetikzlibrary{automata,arrows,positioning,calc}
\usetikzlibrary{positioning}
\usetikzlibrary{fit}
\usetikzlibrary{backgrounds}
\usetikzlibrary{calc}
\usetikzlibrary{shapes}
\usetikzlibrary{mindmap}
\usetikzlibrary{decorations.text}

% \theoremstyle{definition} % insert bellow all blocks you want in normal text
% \newtheorem{definition}{Definition}



% tikzmark command, for shading over items
\newcommand{\tikzmark}[1]{\tikz[overlay,remember picture] \node (#1) {};}
% Define block styles
\tikzstyle{decision} = [diamond, draw, fill=blue!20,
    text width=4.5em, text badly centered, node distance=3cm, inner sep=0pt]
\tikzstyle{block} = [rectangle, draw, fill=blue!20,
    text width=5em, text centered, rounded corners]
\tikzstyle{line} = [draw]
\tikzstyle{cloud} = [draw, ellipse,fill=red!20, node distance=3cm,
    minimum height=2em]

\usepackage[most]{tcolorbox}

\setbeamertemplate{blocks}[rounded][shadow=true] % use rounded blocks with standard beamer shadow
\setbeamertemplate{theorems}[numbered]

\newcommand*{\warning}{\fontencoding{U}\fontfamily{futs}\selectfont\char 66\relax}

% Distributions.
\newcommand*{\UnifDist}{\mathsf{Unif}}
\newcommand*{\ExpDist}{\mathsf{Exp}}
\newcommand*{\DepExpDist}{\mathsf{DepExp}}
\newcommand*{\GammaDist}{\mathsf{Gamma}}
\newcommand*{\LognormalDist}{\mathsf{LogNorm}}
\newcommand*{\WeibullDist}{\mathsf{Weib}}
\newcommand*{\ParetoDist}{\mathsf{Par}}
\newcommand*{\NormalDist}{\mathsf{Norm}}

\newcommand*{\GeometricDist}{\mathsf{Geom}}
\newcommand*{\NegBinomialDist}{\mathsf{NegBin}}
\newcommand*{\PoissonDist}{\mathsf{Poisson}}
\newcommand*{\BivariatePoissonDist}{\mathsf{BPoisson}}
\newcommand*{\CyclicalPoissonDist}{\mathsf{CPoisson}}

\newcommand*{\iid}{\textbf{iid}\@\xspace}
\newcommand*{\pdf}{\textbf{pdf}\@\xspace}
\newcommand*{\cdf}{\textbf{cdf}\@\xspace}
\newcommand*{\pmf}{\textbf{pmf}\@\xspace}
\newcommand*{\abc}{{\textbf{abc}}\@\xspace}
\newcommand*{\smc}{\textbf{smc}\@\xspace}
\newcommand*{\mcmc}{\textbf{mcmc}\@\xspace}
\newcommand*{\ess}{\textbf{ess}\@\xspace}
\newcommand*{\mle}{\textbf{mle}\@\xspace}
\newcommand*{\bic}{\textbf{bic}\@\xspace}
\newcommand*{\kde}{\textbf{kde}\@\xspace}
\newcommand*{\glm}{\textbf{glm}\@\xspace}
\newcommand*{\xol}{\textbf{xol}\@\xspace}
\newcommand*{\cpu}{\textbf{cpu}\@\xspace}
\newcommand*{\gpu}{\textbf{gpu}\@\xspace}
\newcommand*{\arm}{\textbf{arm}\@\xspace}

\def \si {\sigma}
\def \la {\lambda}
\def \al {\alpha}
% \def\e*{\end{eqnarray*}}
\def \di{\displaystyle}

\def \E{\mathbb E}
\def \N{\mathbb N}
\def \Z{\mathbb Z}
\def \NZ{\mathbb{N}_0}
\def \I{\mathbb I}
\def \w{\widehat}
\def \P {\mathbb P}
\def \V{\mathbb V}


\newcommand{\CL}{\mathbb{C}}
\newcommand{\RL}{\mathbb{R}}
\newcommand{\nat}{{\mathbb N}}
\newcommand{\Laplace}{\mathscr{L}}
\newcommand{\e}{\mathrm{e}}
\newcommand{\ve}{\bm{\mathrm{e}}} % vector e

\renewcommand{\L}{\mathcal{L}} % e.g. L^2 loss.

\newcommand{\ih}{\mathrm{i}}
\newcommand{\oh}{{\mathrm{o}}}
\newcommand{\Oh}{{\mathcal{O}}}
\newcommand{\Exp}{\mathbb{E}}

\newcommand{\Norm}{\mathcal{N}}
\newcommand{\LN}{\mathcal{LN}}
\newcommand{\SLN}{\mathcal{SLN}}

\renewcommand{\Pr}{\mathbb{P}}
\newcommand{\Ind}{\mathbb I}
\newcommand\bfsigma{\bm{\sigma}}
\newcommand\bfSigma{\bm{\Sigma}}
\newcommand\bfLambda{\bm{\Lambda}}
\newcommand{\stimes}{{\times}}
\def \limsup{\underset{n\rightarrow+\infty}{\overline{\lim}}}
\def \liminf{\underset{n\rightarrow+\infty}{\underline{\lim}}}




% vertical separator macro
\newcommand{\vsep}{
  \column{0.0\textwidth}
    \begin{tikzpicture}
      \draw[very thick,black!10] (0,0) -- (0,7.3);
    \end{tikzpicture}
}
\newcommand\blfootnote[1]{%
  \begingroup
  \renewcommand\thefootnote{}\footnote{#1}%
  \addtocounter{footnote}{-1}%
  \endgroup
}

% More space between lines in align
% \setlength{\mathindent}{0pt}

% Beamer theme
\usetheme{ZMBZFMK}
\usefonttheme[onlysmall]{structurebold}
\mode<presentation>
\setbeamercovered{transparent=10}

% align spacing
\setlength{\jot}{0pt}

\setbeamertemplate{navigation symbols}{}%remove navigation symbols

\title[BLOCKASTICS III]{Stochastic Models for blockchain analysis}
\subtitle{Blockchain risk analysis}
\author{Pierre-O. Goffard}
\institute[ISFA]{Institut de Science Financières et d'Assurances\\
 \texttt{pierre-olivier.goffard@univ-lyon1.fr}}
\date{\today}
\titlegraphic{\includegraphics[width=2.5cm]{../../Figures/bfs_logo.png}} 

\begin{document}
\begin{frame}
  \titlepage
\end{frame}
\begin{frame}{Blockchain risk analysis}
\tableofcontents
% \begin{enumerate}
%   \item Security of PoW blockchain
%   \item Decentralization in PoS blockchain
%   \item Blockhain efficiency
% \end{enumerate}
\end{frame}
\section{Insurance risk theory}
\begin{frame}{Cramer-Lunberg model}
\begin{columns}
\begin{column}{0.5\textwidth}
\scriptsize
The financial reserves of an insurance company over time have the following dynamic
\begin{equation*}
R_t = z +ct - \sum_{i = 1}^{N_t}U_i\text{, }t\geq0,
\end{equation*}
where 
\begin{itemize}
  \item $z>0$ denotes the initial reserves
  \item $c$ is the premium rate
  \item $(N_t)_{t\geq0}$ is a counting process that models the claim arrival 
  \begin{itemize}
    \scriptsize
    \item[$\hookrightarrow$]  Poisson process with intensity $\lambda$
  \end{itemize}
  \item The $U_i$'s are the randomly sized compensations
  \begin{itemize}
    \scriptsize
    \item[$\hookrightarrow$] non-negative, \textbf{i.i.d.}
  \end{itemize}
\end{itemize}
\end{column}
\begin{column}{0.5\textwidth}
\begin{tikzpicture}
  %Origin and axis
  \coordinate (O) at (0,0);
  \draw[->] (-0.5,0) -- (5.5,0) coordinate[label = {below:\scriptsize$t$}] (xmax);
  \draw[->] (0,-0.5) -- (0,4) coordinate[label = {right:\scriptsize$R_t$}] (ymax);
   %Initial reserves
  \draw (0,2) node[black,left] {\scriptsize$z$} node{};
 % % %Length of the honest chain
  \draw[thick, tublue,-] (0,2) -- (2,3) node[pos=0.5, above] {};
  \draw[thick, dashed, tublue] (2,3) -- (2,1) node[pos=0.5, left] {\scriptsize\color{black}$U_1$};
  \draw[thick, tublue] (2,1) -- (3,1.5) node[pos=0.5, above] {};
  \draw[thick, dashed, tublue] (3,1.5) -- (3, 0.5) node[pos=0.5, left] {\scriptsize\color{black}$U_2$};
  \draw[thick, tublue] (3,0.5) -- (5, 1.5) node[pos=0.5, above] {};
   \draw[thick, dashed, tublue] (5,1.5) -- (5, -0.5) node[pos=0.5,above left] {\scriptsize\color{black}$U_3$};

  %Block finding Times 
  \draw (2,0) node[black,below] {\scriptsize$T_1$} node{ \color{black}$\bullet$};
  \draw (3,0) node[black,below] {\scriptsize$T_2$} node{ \color{black}$\bullet$};
  \draw (5,0) node[black,below left] {\scriptsize$\tau_z$} node{ \color{black}$\bullet$};
\end{tikzpicture}
\end{column}
\end{columns}

\end{frame}
\begin{frame}{Ruin probabilities}
\scriptsize
Define the ruin time as 
$$
\tau_z = \inf\{t\geq0\text{ ; }R_t <0\}
$$
and the ruin probabilities as 
$$
\psi(z,t) = \mathbb{P}(\tau_z < t)\text{ and }\psi(u) = \mathbb{P}(\tau_z < \infty)
$$
We look for $z$ such that 
$$
\mathbb{P}(\text{Ruin}) = \alpha\text{ (0.05)},
$$
given that 
$$
c=(1+\eta)\lambda\mathbb{E}(U),
$$
with 
$$\eta>0\text{ (net profit condition)}$$  
otherwise 
$$\psi(z)=1.$$

\tiny
\begin{thebibliography}{1}

\bibitem{Asmussen_2010}
S.~Asmussen and H.~Albrecher, {\em Ruin Probabilities}.
\newblock {WORLD} {SCIENTIFIC}, sep 2010.

\end{thebibliography}

\end{frame}

\begin{frame}{Ruin probability computation}
\scriptsize
Let 
$$
S_t = z - R_t,\text{ }t\geq0
$$
\begin{tcolorbox}[enhanced,drop shadow, title=Theorem (Wald exponential martingale)]
If $(S_t)_{t\geq0}$ is a L\'evy process or a random walk then
$$
\{\exp\left[\theta S_t-t\kappa(\theta)\right]\text{ , }t\geq0\},\text{ is a martingale,}
$$
where $\kappa(\theta)=\log\mathbb{E}\left(e^{\theta S_1}\right)$.
\end{tcolorbox}
\begin{tcolorbox}[enhanced,drop shadow, title=Theorem (Representation of the ruin probability)]

If $S_t\overset{\textbf{a.s.}}{\rightarrow} -\infty$, and there exists $\gamma>0$ such that $\{e^{\gamma S_t}\text{ , }t\geq0\}$ is a martingale then
$$
\mathbb{P}(\tau_z<\infty)=\frac{e^{-\gamma z}}{\mathbb{E}\left[e^{\gamma \xi(z)}|\tau_z<\infty\right]},
$$
where $\xi(z)=S_{\tau_z}-z\text{ denotes the deficit at ruin.}$
\end{tcolorbox}
\end{frame}
\begin{frame}{Sketch of Proof}
\scriptsize
\begin{itemize}
\item Because of the net profit condition $S_t = \sum_{i=1}^{N_t}U_i-ct\rightarrow -\infty$ as $t\rightarrow\infty$
\item $(S_t)_{t\geq0}$ is a Lévy process, let $\gamma$ be the (unique, positive) solution to
$$
\kappa(\theta) = 0\text{ (Cramer-Lundberg equation)}.
$$
\item  $(e^{\gamma S_t})_{t\geq0}$ is a Martingale then apply the Optional stopping theorem at $\tau_z$.

\end{itemize}
\end{frame}
\section{Link to double spending}
\begin{frame}{Double spending in Satoshi's framework}
\scriptsize
\begin{itemize}
\item The risk reserve process is $R_t=z+Y_1+\ldots+Y_t.$
\item The claim surplus process is $S_t=-(Y_1+\ldots+Y_t).$
\item $\kappa(\theta)=0$ is equivalent to
$$pe^{-\theta}+qe^{\theta}=1.$$
\begin{itemize}
\item[$\hookrightarrow$]\scriptsize $\gamma=\log(p/q).$
\end{itemize}
\item If $p>q$ then $S(t)\rightarrow - \infty$.
\item  $\xi(z)=S_{\tau_z}-z=0$ \textbf{a.s}.
\end{itemize}
Thus,
$$\mathbb{P}(\tau_z<\infty)=\left(\frac{q}{p}\right)^{z}.$$
\end{frame}

\begin{frame}{Double spending with Poisson processes}
\begin{columns}
\begin{column}{0.5\textwidth}
\scriptsize
\begin{itemize}
\item Suppose that
$$
N_t\sim\text{Pois}(\lambda t)\text{ and }M_t\sim\text{Pois}(\mu t)
$$
such that $\lambda>\mu$.
\item The risk reserve process is $R_t=z+N_t-M_t.$
\item The claim surplus process is $S_t=M_t-N_t.$
\end{itemize}
\begin{tcolorbox}[enhanced,drop shadow, title=Fact]
The difference of two Poisson processes is not a Poisson process, However it is L\'evy!
\end{tcolorbox}
\end{column}
\begin{column}{0.5\textwidth}
\begin{tikzpicture}
  %Origin and axis
  \coordinate (O) at (0,0);
  \draw[->] (-0.5,0) -- (5.5,0) coordinate[label = {above:\scriptsize$t$}] (xmax);
  \draw[->] (0,-0.5) -- (0,5) coordinate[label = {right:\scriptsize$n$}] (ymax);
 %Length of the honest chain
  \draw[thick,tublue,-] (0,3) -- (2,3) node[pos=0.5, above] {} ;
  \draw[thick,tublue] (2,3) -- (2,4) node[pos=0.5, above] {};
  \draw[thick,tublue] (2,4) -- (5.5,4) node[pos=0.5, right] {};
  % %Length of the Malicious chain
  \draw[very thick,dashed,red,-] (0,0) -- (0.75,0) node[pos=0.5, above] {} ;
  \draw[very thick,dashed,red] (0.75,0) -- (0.75,1) node[pos=0.5, right] {};
  \draw[very thick,dashed,red] (0.75,1) -- (1.25,1) node[pos=0.5, above] {};
  \draw[very thick,dashed,red] (1.25,1) -- (1.25,2) node[pos=0.5, right] {};
  \draw[very thick,dashed,red] (1.25,2) -- (2.5,2) node[pos=0.5, above] {};
  \draw[very thick,dashed,red] (2.5,2) -- (2.5,3) node[pos=0.5, right] {};
  \draw[very thick,dashed,red] (2.5,3) -- (5,3) node[pos=0.5, right] {};
  \draw[very thick,dashed,red] (5,3) -- (5,4) node[pos=0.5, above] {};
  \draw[very thick,dashed,red] (5,4) -- (5.5,4) node[pos=0.5, above] {};
  %Jump Times of the malicious chain
  \draw (0.75,0) node[red,below] {\scriptsize$S_1$} node{ \color{red}$\bullet$};
  \draw (1.25,0) node[red,below] {\scriptsize$S_2$} node{ \color{red}$\bullet$};
  \draw (2.5,0) node[red,below] {\scriptsize$S_3$} node{ \color{red}$\bullet$};
  \draw (5,0) node[black,below] {\scriptsize$S_4=\tau_z$} node{ \color{black}$\bullet$};
  % %Jump Times of the honest chain
  \draw (2,0) node[tublue,below] {\scriptsize$T_1$} node{ \color{tublue}$\bullet$};
  % %Aggregated Capital gains
  \draw (0,1) node[black,left] {\scriptsize$1$} node{ \color{black}$-$};
  \draw (0,2) node[black,left] {\scriptsize$2$} node{ \color{black}$-$};
  \draw (0,3) node[black,left] {\scriptsize$z$} node{};
  \draw (0,4) node[black,left] {\scriptsize$4$} node{ \color{black}$-$};
  % %Ruin time = First-meeting time
  % \draw (7,0) node[black,below] {$\tau_z$} node{ \color{black}$\times$};
  % \draw[dotted,black] (7,3) -- (7,0);
\end{tikzpicture}
\end{column}
\end{columns}
\end{frame}
\begin{frame}{Double spending with Poisson processes}
\scriptsize
\begin{itemize}
\item $\kappa(\theta)=0$ is equivalent to
$$
\mu e^{\theta}+\lambda e^{-\theta}-(\lambda+\mu)=0.
$$

\begin{itemize}
\item[$\hookrightarrow$] $\gamma=\log(\lambda/\mu).$
\end{itemize}
\item If $\lambda>\mu$ then $S_t\rightarrow - \infty$.
\item $\xi(z)=S_{\tau_z}-z=0$ \textbf{a.s}.
\end{itemize}
Thus
$$\mathbb{P}(\tau_z<\infty)=\left(\frac{\mu}{\lambda}\right)^{z}.$$

\end{frame}

\begin{frame}{Double spending cost}
\scriptsize
Mining cryptocurrency in PoW equipped blockchain is energy consuming
\begin{itemize}
\item[$\hookrightarrow$] Operational cost for miners
\end{itemize}
Per time unit a miner pays
$$
c = \pi_W\cdot W\cdot p,
$$
where 
\begin{itemize}
  \item $\pi_W$ is the electricty price per kWh
  \item $W$ is the consumption of the network \url{https://cbeci.org/}
  \item  $p$ is the miner's hashpower 
\end{itemize}
\begin{tcolorbox}[enhanced,drop shadow, title=Fact]
The cost of double spending is $c\cdot \tau_z$.
\end{tcolorbox}
\begin{tcolorbox}[enhanced,drop shadow, title=Theorem (\textbf{P.d.f.} of the double spending time)]
If $\{N_t\text{ , }t\geq0\}$ is a Poisson process then the \textbf{p.d.f.} of $\tau_z$ is given by
\begin{equation*}
f_{\tau_z}(t)=\mathbb{E}\left[\frac{z}{z+N_t}f_{S_{N_t+z}}(t)\right],\text{ for }t\geq0.
\end{equation*}
\end{tcolorbox}
\end{frame}
\begin{frame}{Sketch of the proof}
\scriptsize
Let's condition upon the values of $N_t$,
\begin{itemize}
  \item if $N_t=0$ then 
  $$\tau_z = S_z\text{ and }f_{\tau_z|N_t=0}(t) = f_{S_z}(t)$$
  \item If $N_t = n$ for $n\geq 1$ then 
  $$
  \{\tau_z = t\} = \bigcup_{k = 1}^n\{T_k\leq S_{z+k-1}\}\cup\{S_{n+z} = t\}
  $$
We have 
\begin{eqnarray*}
f_{\tau_z|N_t = n}(t) &=& \mathbb{P}(U_{1:n}\leq S_z/t,\ldots,U_{n:n}\leq S_{z+n-1}/t\Big\rvert S_{n+z}=t)f_{S_{n+z}}(t)\\
&=&\frac{z}{z+n}f_{S_{n+z}}(t).
\end{eqnarray*}
Thanks to the properties of the Abel-Gontcharov polynomials.
\end{itemize}
\tiny
\begin{thebibliography}{1}

\bibitem{Goffard2019}
P.-O. Goffard, ``Fraud risk assessment within blockchain transactions,'' {\em
  Advances in Applied Probability}, vol.~51, pp.~443--467, jun 2019.
\newblock \url{https://hal.archives-ouvertes.fr/hal-01716687v2}.
\bibitem{Grunspan2021}
C.~Grunspan and R.~P{\'{e}}rez-Marco, ``{ON} {PROFITABILITY} {OF} {NAKAMOTO}
  {DOUBLE} {SPEND},'' {\em Probability in the Engineering and Informational
  Sciences}, pp.~1--15, feb 2021.

\bibitem{Brown2020}
M.~Brown, E.~Peköz, and S.~Ross, ``{BLOCKCHAIN} {DOUBLE}-{SPEND} {ATTACK}
  {DURATION},'' {\em Probability in the Engineering and Informational
  Sciences}, pp.~1--9, may 2020.

\bibitem{Jang2020}
J.~Jang and H.-N. Lee, ``Profitable double-spending attacks,'' {\em Applied
  Sciences}, vol.~10, p.~8477, nov 2020.

\end{thebibliography}
\end{frame}

\section{Link to blockchain mining}
\begin{frame}{Dual risk model}
\begin{columns}
\begin{column}{0.5\textwidth}
\scriptsize
A blockchain miner with hashpower share $p\in(0,1)$ that
\begin{itemize}
  \item owns $u\geq0$ at the beginning 
  \item spend $c = \pi_W\cdot W\cdot p$ per time unit
  \item finds block at a rate $p \lambda$, where $\lambda$ is the arrival rate of blocks
\end{itemize}
The miner's surplus is given by 
$$
R_t = u - c\cdot t + N_t\cdot b,\text{ (Dual risk model)}
$$
where 
\begin{itemize}
  \item $(N_t)_{t\geq0}$ is a Poisson process with intensity $p\cdot\lambda$
  \item $b$ is the block finding reward (6.25 BTC) \url{bitcoinhalf.com}
\end{itemize}
\end{column}
\begin{column}{0.5\textwidth}
\begin{tikzpicture}
  %Origin and axis
  \coordinate (O) at (0,0);
  \draw[->] (-0.5,0) -- (5.5,0) coordinate[label = {below:\scriptsize$t$}] (xmax);
  \draw[->] (0,-0.5) -- (0,4) coordinate[label = {right:\scriptsize$R_t$}] (ymax);
   %Initial reserves
  \draw (0,3) node[black,left] {\scriptsize$u$} node{};
 % % %Length of the honest chain
  \draw[thick, tublue,-] (0,3) -- (2,1) node[pos=0.5, above] {};
  \draw[thick, dashed, tublue] (2,1) -- (2,2) node[pos=0.5, above left] {\scriptsize\color{black}$b$};
  \draw[thick, tublue] (2,2) -- (3.5,0.5) node[pos=0.5, above] {};
  \draw[thick, dashed, tublue] (3.5,0.5) -- (3.5, 1.5) node[pos=0.5, above left] {\scriptsize\color{black}$b$};
  \draw[thick, tublue] (3.5,1.5) -- (5, 0) node[pos=0.5, above] {};

  %Block finding Times 
  \draw (2,0) node[black,below] {\scriptsize$T_1$} node{ \color{black}$\bullet$};
  \draw (3.5,0) node[black,below] {\scriptsize$T_2$} node{ \color{black}$\bullet$};
  \draw (5,0) node[black,below] {\scriptsize$\tau_u$} node{ \color{black}$\bullet$};
\end{tikzpicture}
\end{column}
\end{columns}
\end{frame}
\begin{frame}{Expected profit given not ruin}
\scriptsize

\begin{tcolorbox}[enhanced,drop shadow, title=Fact]
The steady operational cost compensated by infrequent capital gains makes mining a risky business.
\end{tcolorbox}
Define the ruin time 
$$
\tau_u  = \inf\{t\geq0\text{ ; }R_t \leq0\}
$$
\begin{itemize}
  \item Risk measure 
  $$
  \psi(u,t) = \mathbb{P}(\tau_u \leq t)
  $$
  \item Profitability measure
  $$
  V(u,t) = \mathbb{E}(R_t\mathbb{I}_{\tau_u > t})
  $$
\end{itemize} 
\end{frame}
\begin{frame}{Miner's dilemna} 
\scriptsize
Use $\psi$ and $V$ to compare solo mining to
\begin{itemize}
  \item Joining a mining pool
\tiny
  \begin{thebibliography}{1}

\bibitem{rosenfeld2011analysis}
M.~Rosenfeld, ``Analysis of bitcoin pooled mining reward systems,'' 2011.

\bibitem{albrecher2021blockchain}
H.~Albrecher, D.~Finger, and P.-O. Goffard, ``Blockchain mining in pools:
  Analyzing the trade-off between profitability and ruin,'' 2021.


\end{thebibliography}
  \item \scriptsize Deviating from the protocol (selfish mining)

  \tiny
  \begin{thebibliography}{1}
  \bibitem{Eyal2014}
I.~Eyal and E.~G. Sirer, ``Majority is not enough: Bitcoin mining is
  vulnerable,'' in {\em Financial Cryptography and Data Security},
  pp.~436--454, Springer Berlin Heidelberg, 2014.

\bibitem{albrecher:hal-02649025}
H.~Albrecher and P.-O. Goffard, ``{On the profitability of selfish blockchain
  mining under consideration of ruin},'' {\em To appear in Operations
  Research}, May 2021.
\newblock \url{https://arxiv.org/abs/2010.12577}.
\end{thebibliography}
\end{itemize}
Tractable formulas follow from 
$$
\widehat{\psi}(u,t)= \mathbb{E}[\psi(u,T)]\text{ and }\widehat{V}(u,t)= \mathbb{E}[V(u,T)],
$$
where $T\sim\text{Exp}(t)$.
\end{frame}

\begin{frame}{When mining solo}
\scriptsize
\begin{tcolorbox}[enhanced,drop shadow, title=Theorem (profit and ruin when mining solo)]
For any $u\geq0$, we have
\begin{equation*}
\widehat{\psi}(u,t) = e^{\rho^\ast u},
\end{equation*}
and 
\begin{equation*}
\widehat{V}(u,t) = u+(p\lambda b-c)t\left(1-e^{\rho^\ast u }\right),
\end{equation*}
where $\rho^\ast$ is the negative solution of the equation
\begin{equation}\label{eq:equation_rho}
-c\rho + p\lambda(e^{b\rho}-1) = 1/t.
\end{equation}
\end{tcolorbox}
\begin{tcolorbox}[enhanced,drop shadow, title=Lambert function]
The solution $\rho^\ast$ of \eqref{eq:equation_rho} is given by 
\begin{equation*}
  \rho^{\ast}=-\frac{p \lambda t+1}{ct}
  -\frac{1}{b} \,{\rm W} \left[-\frac{p\lambda
    \,b}{c}\,{e^{-b\,\left(\frac{p \lambda t+1}{ct}\right)}}
  \right],
  \end{equation*}
  where $W(.)$ denotes the Lambert function.
\end{tcolorbox}
\end{frame}
\begin{frame}[allowframebreaks]{Sketch of the proof}
\scriptsize
Recall that the time horizon is random $T\sim\text{Exp}(t)$, we condition upon what happen during the time interval $(0,h)$, with $h<u/c$ so that ruin does not occur before $h$. Three possibilities
\begin{itemize}
  \item[(i)] $T>h$ and there is no block discovery during $(0,h)$
  \item[(ii)] $T<h$ and there is no block discovery during $(0,T)$
  \item[(iii)] There is a block discovery before time $T$ and in the interval $(0,h)$
\end{itemize}
For the expected profit $\widehat{V}(u,t)$, we get 
\begin{eqnarray*}
  \widehat{V}(u,t)& =&e^{-h(1/t + p\lambda)}\,\widehat{V}(u-ch,t)+\int\limits_0^h\frac1t\, e^{-s(1/t + p\lambda)}\,(u-cs)ds\\
  &+&\int\limits_0^h p\lambda\, e^{-s(1/t + p\lambda)}\,\widehat{V}(u-cs+b,t)ds.
  \end{eqnarray*}
Now we take the derivative with respect to $h$ and set $h=0$ to obtain
\begin{equation}\label{eq:ODE}
c\widehat{V}'(u,t) + \left(\frac{1}{t} +  p\lambda\right)\widehat{V}(u,t) - p\lambda \widehat{V}(u+b,t) - \frac{u}{t} =0,
\end{equation}
with boundary condition 
$$\widehat{V}(0,t) = 0 \text{ and such that } 0\leq \widehat{V}(u,t)\leq u-ct+p\lambda t \text{ for }u>0.
$$  
Equation \eqref{eq:ODE} is an advanced functional differential equation, consider solutions of the form 
\begin{equation}\label{eq:potential_solution}
\widehat{V}(u,t) = Ae^{\rho u }+Bu + C,\text{ }u \ge 0, 
\end{equation}
where $A, B,C$ and $\rho$ are constants to be determined. Substituting \eqref{eq:potential_solution} in \eqref{eq:ODE} together with the boundary condition yields the system of equations 
\begin{equation*}
\begin{cases}
0&=ct\rho + \left(1+p\lambda t\right)-p\lambda te^{\rho b}, \\
0&= B\left(1+tp\lambda\right)-p\lambda tB - 1,\\
0&=Bct+C(1+tp\lambda) - p\lambda t Bb-p\lambda tC, \\
0&=A+C.
\end{cases}
\end{equation*}
We then have $A = -t(p\lambda b - c)$, $B = 1$, $C = t(p\lambda b - c)$ and $\rho$ is solution of 
$$
c\rho + \left(1+p\lambda t\right)-p\lambda te^{\rho b} = 0,
$$
which admits one negative and one positive solution. As $A<0$, we have to choose the negative solution $\rho^\ast<0$ in order to ensure $\widehat{V}(u,t)>0$. Substituting $A,B,C$ and $\rho^{\ast}$ in \eqref{eq:potential_solution}  yields the result.\\

Similarly, the ruin probability is solution to 
\begin{equation*}\label{psii}
c\widehat{\psi}'(u,t)+(p \lambda+1/t)\,\widehat{\psi}(u,t)-p \lambda\,\widehat{\psi}(u+b,t)=0
\end{equation*}
with initial condition $\widehat{\psi}(0,t)=1$ and boundary condition $\lim_{u\to\infty}\widehat{\psi}(u,t)=0$.
\end{frame}
\begin{frame}{How does a mining pool works?}
\scriptsize
A set of miners $I\subset\{1,\ldots, n\}$ join forces and gather a share 
$$
p_I = \sum_{i\in I }p_i,
$$
of the total mining power. 
\begin{itemize}
  \item a pool manager coordinates
  \item miners prove their contribution by submiting \textit{shares}  
\end{itemize}
\begin{tcolorbox}[enhanced,drop shadow, title=Definition (share)]
A \textit{share} is a partial solution to the cryptopuzzle
\end{tcolorbox}
The pool manager sets
\begin{itemize} 
  \item the redistribution system
  \item the relative difficulty $q\in(0,1)$ of finding a \textit{share} compared to finding a block
  \item the pool participation fee $f$
  \end{itemize} 
\end{frame}
\begin{frame}{Redistribution systems}
\scriptsize
Miners should be retributed in proportion to their contribution to the mining effort. 
\begin{tcolorbox}[enhanced,drop shadow, title=Proportional reward system]
A \textit{round} is the time elapsed between two block discoveries
\begin{itemize} 
  \item $s_i$ the number of shares submitted by miner $i\in I$ during the \textit{round}
  \item Each miner receives at the end of the \textit{round}
  $$
  (1-f)\cdot b\cdot\frac{s_i}{\sum_{i\in I}s_i},
  $$
  where $f$ represents the pool manager's cut.
  \item The system is fair as $\frac{s_i}{\sum_{i\in I}s_i}$ should converges toward $\frac{p_i}{\sum_{i\in I}p_i}$
\end{itemize}

\end{tcolorbox}
\end{frame}
\begin{frame}{What's wrong with going proportional?}
\scriptsize
\begin{tcolorbox}[enhanced,drop shadow, title=Fact]
The proportional reward system is fair but it is not incentive compatible 
\end{tcolorbox}
\tiny
\begin{thebibliography}{1}

\bibitem{Schrijvers2017}
O.~Schrijvers, J.~Bonneau, D.~Boneh, and T.~Roughgarden, ``Incentive
  compatibility of bitcoin mining pool reward functions,'' in {\em Financial
  Cryptography and Data Security}, pp.~477--498, Springer Berlin Heidelberg,
  2017.

\end{thebibliography}
\scriptsize
Why?
\begin{itemize}
  \item The durations of a round is random 
  \begin{itemize}
    \scriptsize
    \item[$\hookrightarrow$] A \textit{share} is worth less in longer round $\Rightarrow$ \textit{pool hoping} \tiny
    \begin{thebibliography}{1}

\bibitem{rosenfeld2011analysis}
M.~Rosenfeld, ``Analysis of bitcoin pooled mining reward systems,'' 2011.

\end{thebibliography}
\scriptsize
    \item[$\hookrightarrow$] Apply a discounting factor on the value of a \textit{share} \tiny
    \begin{thebibliography}{1}
\bibitem{slush}
slush pool, ``Reward system specifications,'' 2021.
\end{thebibliography}
  \end{itemize}
  \item A miner may postpone the communication of a solution
  \begin{itemize}
    \scriptsize
    \item[$\hookrightarrow$] She may awaits for her number of \textit{share} to match her actual hashpower 
  \end{itemize} 
  \item Absolutely no risk tranfer from miners to manager
    \begin{itemize}
    \scriptsize
    \item[$\hookrightarrow$] $f$ should be low 
  \end{itemize} 
\end{itemize}
\end{frame}
\begin{frame}{Pay-per-Share (PPS) reward system}
\scriptsize
The pool manager gives out
$$
w = (1-f)\cdot q \cdot b 
$$ 
for each submitted share and keep the block reward.
\vspace{1cm}
\begin{columns}
\begin{column}{0.5\textwidth}
Miner's viewpoint
$$
R_t^i = u_i-ct + M_t^i w,\text{ }t\geq0.
$$
where 
\begin{itemize}
   \item $(M_t^i)_{t\geq0}$ is a Poisson process with intensity $p_i \mu= p_i\lambda / q$
   \item $\mu$ corresponds to the rate at which the network find \textit{shares}
\end{itemize}
\end{column}
\begin{column}{0.5\textwidth}
Manager's viewpoint
$$
R_t^I = u_I - M_t^I w+N_t^I b,\text{ }t\geq0.
$$
where 
\begin{itemize}
   \item $(M_t^I)_{t\geq0}$ is a Poisson process with intensity $p_I\mu =p_I\lambda / q$
   \item $(N_t^I)_{t\geq0}$ is a Poisson process with intensity $p_I\lambda$
\end{itemize}
\end{column}
\end{columns}
\end{frame}
\begin{frame}{Pool manager at risk?}
\scriptsize
For mathematical convenience we replace the deterministic reward by stochastic ones, the wealth of the pool manager becomes
$$
R_t= u - \sum_{i=1}^{M_t} W_i +\sum_{j=1}^{N_t} B_j,\text{ }t\geq0.
$$
where 
\begin{itemize}
  \item $(M_t)_{t\geq0}$ and $(N_t)_{t\geq0}$ are Poisson processes with intensity $\mu^\ast=\mu- \lambda$ and $\lambda$
  \item $(W_i)_{i\geq0}$ and $(B_j)_{j\geq0}$ are two independent sequence of \iid exponentially distributed random variables with mean $w$ and $b^\ast = b-w$
\end{itemize}
\begin{tcolorbox}[enhanced,drop shadow, title=Poisson process superposition]
A block discovery triggers automatically a payment to the miner, specifying 
\begin{itemize}
\item the intensity of $M_t$ as $\mu^\ast=\mu- \lambda$ 
\item the size of the upward jumps as $b^\ast = b-w$ 
\end{itemize}
allows us to isolate the downward jumps and make the Poisson processes independent. 
\end{tcolorbox}
\end{frame}
\begin{frame}{Pool manager at risk?}
\scriptsize
\begin{tcolorbox}[enhanced,drop shadow, title=Theorem (Profit and ruin of the pool manager)]
The ruin probability is given by 
\begin{equation*}\label{psiexpe}
    \widehat{\psi}(u,t) = (1-Rw)  e^{-R u},\;u\ge 0,
\end{equation*}
and the expected profit is given by
\begin{equation*}\label{Vcombexpe}
    \w{V}(u,t) = (1 - Rw)[w-t(\lambda b^\ast-\mu^\ast w)] e^{-R u}+u+t(\lambda b^\ast-\mu^\ast w),
\end{equation*}
where $R$ is the (unique) solution with positive real part of 
\begin{equation*} \label{VLunde}
    -(t^{-1}+\lambda+\mu^\ast)+\lambda(1+b^\ast r)^{-1}+\mu^\ast(1-wr)^{-1}=0.
\end{equation*}
\end{tcolorbox}
\tiny
  \begin{thebibliography}{1}

\bibitem{albrecher2021blockchain}
H.~Albrecher, D.~Finger, and P.-O. Goffard, ``Blockchain mining in pools:
  Analyzing the trade-off between profitability and ruin,'' 2021.


\end{thebibliography}
\end{frame}
\begin{frame}[allowframebreaks]{Sketch of the Proof}
\scriptsize
We condition upon what happen during the time interval $(0,h)$. Three possibilities
\begin{itemize}
  \item[(i)] $T>h$ and no jump during $(0,h)$
  \item[(ii)] $T<h$ and no jump over $(0,T)$
  \item[(iv)] An upward jump in the interval $(0,h)$
  \item[(iii)] A downward jump in the interval $(0,h)$
\end{itemize}
  \begin{eqnarray*}\label{neu0}
      \w{V}(u,t)&=& e^{-(\frac{1}{t}+\lambda+\mu^\ast)h}\w{V}(u,t) + \frac{1}{t}\int_0^h e^{-{s}/{t}}e^{-(\lambda +\mu^\ast) s} u\,ds\\
      & +& \lambda\int_0^he^{-\lambda s} e^{-({1}/{t}+\mu^\ast) s} \int_0^\infty\w{V}(u+x,t)\,dF_{B}(x)\,ds\\
      &  +&\mu^\ast \int_0^he^{-\mu^\ast s} e^{-({1}/{t}+\lambda) s}\int_0^u \w{V}(u-y,t) \,dF_W(y)\,ds.
  \end{eqnarray*}
  Differentiating with respect to $h$ and letting $h\rightarrow 0$ yields the following integral equation
  \begin{equation} \label{inteq}
    \lambda\int_0^\infty\w{V}(u+x,t)\,dF_{B}(x)-(\lambda+\mu^\ast+{1}/{t})\w{V}(u,t)+\mu^\ast\int_0^u \w{V}(u-y,t) \,dF_W(y)+{u}/{t}=0,\quad u\ge 0,
  \end{equation}
  with boundary conditions $\w{V}(u,t)=0$ for all $u<0$ and $0\leq\w{V}(u,t)\leq u+(\lambda b^\ast - \mu^\ast w)$. Let us plug in the ansatz
  $$
  Ce^{-ru}+d_1u+d_0
  $$
  \begin{itemize}
    \item Comparing the terms in $e^{-r u}$ gives and equation for $r$
    $$
    -(t^{-1}+\lambda+\mu^\ast)+\lambda(1+b^\ast r)^{-1}+\mu^\ast(1-wr)^{-1}=0
    $$
    of which only the positive solution $R>0$ is valid due to the boundary conditions.
    \item Comparing the terms in $u$ yields $d_1 = 1$
    \item Comparing the terms in $1$ yields
    $$
    d_0 = t(\lambda b^\ast-\mu^\ast w)
    $$
    \item Comparing the terms in $e^{-u/w}$
    $$
    C = (1 - Rw)[w-t(\lambda b^\ast-\mu^\ast w)]
    $$
  \end{itemize}
\end{frame}
\begin{frame}{Problem caused by mining}
\begin{itemize}
  \item $R\&D$ arm race, consuming a lot of energy and generating large amounts of e-waste
\tiny
\begin{thebibliography}{1}

\bibitem{bertucci2020mean}
C.~Bertucci, L.~Bertucci, J.-M. Lasry, and P.-L. Lions, ``Mean field game
  approach to bitcoin mining,'' 2020.

\bibitem{Alsabah2018}
H.~Alsabah and A.~Capponi, ``Bitcoin mining arms race: R{\&}d with
  spillovers,'' {\em {SSRN} Electronic Journal}, 2018.

\end{thebibliography}
\end{itemize}
\begin{itemize}
  \item \normalsize A threat on centralization?
\tiny
\begin{thebibliography}{1}

\bibitem{Cong2020}
L.~W. Cong, Z.~He, and J.~Li, ``Decentralized mining in centralized pools,''
  {\em The Review of Financial Studies}, vol.~34, pp.~1191--1235, apr 2020.

\bibitem{li2019mean}
Z.~Li, A.~M. Reppen, and R.~Sircar, ``A mean field games model for
  cryptocurrency mining,'' 2019.

\end{thebibliography}
\end{itemize}
\end{frame}

\begin{frame}{Blockwithholding strategies}
\begin{center}
\begin{tikzpicture}[-, >=stealth', auto, semithick, node distance=1cm]

% \tikzstyle{block} = [rectangle, draw, fill=blue!20,
%     text width=5em, text centered, rounded corners]
\tikzstyle{unconfirmed block}=[rectangle, fill=white, draw=black,dashed, thick,text=black,scale=1]
\tikzstyle{phantom block}=[rectangle, fill=white,draw=white,thick,text=black,scale=1]
\tikzstyle{confirmed block}=[rectangle, fill=black,draw=black,thick,text=black,scale=1]
\tikzstyle{bad block unconfirmed}=[rectangle, fill=white,draw=tublue,very thick, text=black,scale=1]
\tikzstyle{bad block confirmed}=[rectangle, fill=tublue, text=black,scale=1]
\node[confirmed block]    (1){};
\node[phantom block]    (2)[right of=1] {};
% \node[phantom block]    (3)[below of=1] {};
\pause
\node[bad block unconfirmed] (4)[below of=2] {};
\draw[-,dashed,thick,tublue] (1) |- node[near start,above] {} (4);
% \path
% (1) edge[dashed, tublue, thick, left] node{} (3);
% (3) edge[dashed, tublue, thick, left] node{} (4);
\pause
\node[bad block unconfirmed]    (5)[right of=4] {};
\node[bad block unconfirmed]    (6)[right of=5] {};
\path
(4) edge[dashed, tublue, thick, left]     node{}     (5)
(5) edge[dashed, tublue, thick, left]     node{}     (6);
\pause
\node[confirmed block] (7)[right of =1]{};
\path
(1) edge[-, black, thick, left]     node{}     (7);
\pause
\node[bad block confirmed]    (8)[below of=2] {};
\draw[-,thick,tublue] (1) |- node[near start,above] {} (8);
\pause
\node[confirmed block] (9)[right of =7]{};
\path
(7) edge[-, black, thick, left]     node{}     (9);
\pause
\node[bad block confirmed]    (10)[right of=4] {};
\node[bad block confirmed]    (11)[right of=5] {};
\node[right of=11] {\scriptsize$3\cdot b$ (LCR) };
\node[right of=9] {\scriptsize$0$};
\path
(4) edge[-, tublue, thick, left]     node{}     (10)
(4) edge[-, tublue, thick, left]     node{}     (11);
\end{tikzpicture}
\end{center}
\pause
\scriptsize
Why?
\pause
\begin{itemize}
  \item Relative revenue greater than their fair share
  \item Honest miners waste resources
  \begin{itemize}
    \scriptsize
    \item[$\hookrightarrow$] Honest miner might quit making malicious miners more powerful
  \end{itemize}
  \item Slow down the pace of block arrivals
  \begin{itemize}
    \scriptsize
    \item[$\hookrightarrow$] Downward adjustment of the cryptopuzzle difficulty
  \end{itemize}
\end{itemize}
\begin{tcolorbox}[enhanced,drop shadow, title=Difficulty adjustments]
The cryptopuzzle difficulty is adjusted every $2,016$ blocks to ensure $1$ block every ten minutes on average.
\end{tcolorbox}
\tiny
\begin{thebibliography}{1}
  \bibitem{Eyal2014}
I.~Eyal and E.~G. Sirer, ``Majority is not enough: Bitcoin mining is
  vulnerable,'' in {\em Financial Cryptography and Data Security},
  pp.~436--454, Springer Berlin Heidelberg, 2014.
\end{thebibliography}
\end{frame}
\begin{frame}{Tied selfish mining}
\begin{columns}
\begin{column}{0.5\textwidth}
\begin{center}
\begin{tikzpicture}[-, >=stealth', auto, semithick, node distance=1cm]

\tikzstyle{unconfirmed block}=[rectangle, fill=white,draw=black, thick,text=black,scale=1]
\tikzstyle{phantom block}=[rectangle, fill=white,draw=white,thick,text=black,scale=1]
\tikzstyle{confirmed block}=[rectangle, fill=black,draw=black,thick,text=black,scale=1]
\tikzstyle{bad block unconfirmed}=[rectangle, fill=white,draw=tublue,very thick, text=black,scale=1]
\tikzstyle{bad block confirmed}=[rectangle, fill=tublue, text=black,scale=1]
\node[confirmed block]    (1){};
\node[phantom block]    (2)[right of=1] {};
% \node[phantom block]    (3)[below of=1] {};
\node[bad block unconfirmed] (3)[below of=2] {};
\draw[-,dashed,thick,tublue] (1) |- node[near start,above] {} (3);
\pause 
\node[confirmed block] (4)[right of=1]{};
\path
(1) edge[-, black, thick, left]   node{}(4);
\pause
\node[bad block confirmed] (5)[below of=2] {};
\draw[-,thick,tublue] (1) |- node[near start,above] {} (5);
\pause
\node[unconfirmed block] (6)[right of=4] {};

\end{tikzpicture}
\end{center}
\end{column}
\begin{column}{0.5\textwidth}
\begin{tikzpicture}[-, >=stealth', auto, semithick, node distance=1cm]
\tikzstyle{unconfirmed block}=[rectangle, fill=white,draw=black, thick,text=black,scale=1]
\tikzstyle{phantom block}=[rectangle, fill=white,draw=white,thick,text=black,scale=1]
\tikzstyle{confirmed block}=[rectangle, fill=black,draw=black,thick,text=black,scale=1]
\tikzstyle{bad block unconfirmed}=[rectangle, fill=white,draw=tublue,thick, text=black,scale=1]
\tikzstyle{bad block confirmed}=[rectangle, fill=tublue,draw=tublue,thick, text=black,scale=1]
\node[confirmed block]    (1){};
\node[phantom block]    (2)[right of=1] {};
\node[confirmed block] (4)[right of=1]{};
\path
(1) edge[-, black, thick, left]   node{}(4);
\node[bad block confirmed] (5)[below of=2] {};
\draw[-,thick,tublue] (1) |- node[near start,above] {} (5);
\node[confirmed block] (6)[right of=4] {};
\path
(2) edge[-, black, thick, left]   node{}(6);
\node[right of=6] {\scriptsize$2\cdot b$  };
\node[right of=5] {\scriptsize$0$};
\end{tikzpicture}
\pause
\newline
\newline
\begin{tikzpicture}[-, >=stealth', auto, semithick, node distance=1cm]
\tikzstyle{unconfirmed block}=[rectangle, fill=white,draw=black, thick,text=black,scale=1]
\tikzstyle{phantom block}=[rectangle, fill=white,draw=white,thick,text=black,scale=1]
\tikzstyle{confirmed block}=[rectangle, fill=black,draw=black,thick,text=black,scale=1]
\tikzstyle{bad block unconfirmed}=[rectangle, fill=white,draw=tublue,thick, text=black,scale=1]
\tikzstyle{bad block confirmed}=[rectangle, fill=tublue,draw=tublue,thick, text=black,scale=1]
\node[confirmed block]    (1){};
\node[phantom block]    (2)[right of=1] {};
\node[confirmed block] (4)[right of=1]{};
\path
(1) edge[-, black, thick, left]   node{}(4);
\node[bad block confirmed] (5)[below of=2] {};
\draw[-,thick,tublue] (1) |- node[near start,above] {} (5);
\node[confirmed block] (6)[right of=5] {};
\path
(5) edge[-, tublue, thick, left]   node{}(6);
\node[right of=4] {\scriptsize$0$  };
\node[right of=6] {\scriptsize$b$ $\&$ $b$};
\end{tikzpicture}
\pause
\newline
\newline
\begin{tikzpicture}[-, >=stealth', auto, semithick, node distance=1cm]
\tikzstyle{unconfirmed block}=[rectangle, fill=white,draw=black, thick,text=black,scale=1]
\tikzstyle{phantom block}=[rectangle, fill=white,draw=white,thick,text=black,scale=1]
\tikzstyle{confirmed block}=[rectangle, fill=black,draw=black,thick,text=black,scale=1]
\tikzstyle{bad block unconfirmed}=[rectangle, fill=white,draw=tublue,thick, text=black,scale=1]
\tikzstyle{bad block confirmed}=[rectangle, fill=tublue,draw=tublue,thick, text=black,scale=1]
\node[confirmed block]    (1){};
\node[phantom block]    (2)[right of=1] {};
\node[confirmed block] (4)[right of=1]{};
\path
(1) edge[-, black, thick, left]   node{}(4);
\node[bad block confirmed] (5)[below of=2] {};
\draw[-,thick,tublue] (1) |- node[near start,above] {} (5);
\node[bad block confirmed] (6)[right of=5] {};
\path
(5) edge[-, tublue, thick, left]   node{}(6);
\node[right of=4] {\scriptsize$0$  };
\node[right of=6] {\scriptsize$2\cdot b$};
\end{tikzpicture}

\end{column}
% snzj
% \pause
\end{columns}
\pause
\scriptsize
\begin{tcolorbox}[enhanced,drop shadow, title=Fact]
The outcome of the fork when the honest miners find the subsequent block depends on the connectivity parameter $q\in(0,1)$ of the malicious nodes.
\end{tcolorbox}
\end{frame}
\begin{frame}{Markov model}
\scriptsize
\begin{itemize}
\item Let $(Z_n)_{n\geq0}$ be the number of blocks in the selfish miner's buffer with state space $\{0, 0^\ast, 1\}$. 

\item The transition graph is given by
\begin{center}
\begin{tikzpicture}[->, >=stealth', auto, semithick, node distance=2cm]
\tikzstyle{every state}=[fill=white,draw=black,thick,text=black,scale=0.8]
\node[state]    (1)                     {\scriptsize$0$};
\node[state]    (2)[right of=1]         {\scriptsize$1$};
\node[state]    (3)[above of=2]         {\scriptsize$0^{\ast}$};
\path
(1) edge[loop left]     node{\scriptsize$1-p$}        (1)
    edge[bend left]     node{\scriptsize$p$}          (2)
(2) edge[bend left]      node{\scriptsize$p$}         (1)
    edge[bend right]      node{\scriptsize$1-p$}      (3)
(3) edge[bend right, above]      node{\scriptsize$1$}         (1);
\end{tikzpicture}
\end{center}
\item The reward collecting process is also Markovian as
\begin{equation*}
f\left[Z_{n-1}, \xi_n,\zeta_{n}\right] = \begin{cases}
0,&\text{ if } Z_{n-1} =0,\\
0,&\text{ if } Z_{n-1} =1\text{ }\&\text{ }\xi_{n} =0,\\
2,&\text{ if } Z_{n-1} =1\text{ }\&\text{ }\xi_{n} =1,\\
0,&\text{ if } Z_{n-1} =0^\ast \text{ }\&\text{ }\xi_{n} = 0\text{ }\&\text{ }\zeta_{n} = 0,\\
1,&\text{ if } Z_{n-1} =0^\ast \text{ }\&\text{ }\xi_{n} = 0\text{ }\&\text{ }\zeta_{n} = 1,\\
2,&\text{ if }Z_{n-1} = 0^\ast \text{ }\&\text{ }\xi_{n} = 1.\\
\end{cases}
\end{equation*}
where $\xi_n\sim\text{Ber}(p)$ and $\xi_n\sim\text{Ber}(q)$
\end{itemize}
\end{frame}
\begin{frame}{Wealth of a selfish miner}
\scriptsize
The wealth process of a selfish miner is given by 
$$
R_t = u - c\cdot t +b\cdot\sum_{n=1}^{N_t} f(Z_k,\xi_n,\zeta_n ).
$$
where $N_t$ is a Poisson process with intensity $\lambda$ that corresponds to the arrival of blocks in the blockchain.
\begin{itemize}
  \item The net profit condition reads as 
  $$
  \frac{b}{t}\mathbb{E}\left[\sum_{n=1}^{N_t} f(Z_k,\xi_n,\zeta_n )\right] > c
  $$
  \item Once stationarity is reached we have 
  $$
\P(Z_\infty = 0)=\frac{1}{1+2p-p^2},\text{ }\P(Z_\infty = 1)=\frac{p}{1+2p-p^2},\text{ and }\P(Z_\infty = 0^{\ast})=\frac{p(1-p)}{1+2p-p^2},
$$
and the net profit condition correspondingly reads
\begin{equation*}
b\lambda\frac{qp(1-p)^2 + 4p^2-2p^3}{1+2p-p^2} - c>0.
\end{equation*}
\end{itemize}
\end{frame}
\begin{frame}{Expected profit of a selfish miner}
\scriptsize
Define 
\begin{equation*}
\w{\psi}_z(u,t)\equiv \E(\psi_z(u,T)) = \mathbb{E}\left(\psi(u,T)\Big \rvert Z_0 = z\right) \text{ and }\w{V}_z(u,t)\equiv \E(V_z(u,T)) = \mathbb{E}\left(R_T\mathbb{I}_{\tau_u>T}\Big \rvert Z_0 = z\right)
\end{equation*}
\begin{tcolorbox}[enhanced,drop shadow, title=Theorem]
For any $u\ge 0$, the ruin probability and expected profit of a selfish miner are given by
\begin{equation*}\label{ruin_selfish_app}
\w{\psi}_0(u,t) =
{\it C_1}\,{{\rm e}^{\rho_1\,u}}+{{\rm e}^{\rho_2\,u}} \left[ {\it C_2}\,
\cos \left( \rho_3\,u \right) +{\it C_3}\,\sin \left( \rho_3\,u \right)\right],
\end{equation*}
and 
\begin{equation*}
\w{V}_0(u,t)={\it A_1}\,{{\rm e}^{\rho_1\,u}}+{{\rm e}^{\rho_2\,u}} \left[ {\it A_2}\,
\cos \left( \rho_3\,u \right) +{\it A_3}\,\sin \left( \rho_3\,u \right)\right] +u+C,\label{thisone}
\end{equation*}
\end{tcolorbox}
\tiny
  \begin{thebibliography}{1}

\bibitem{albrecher:hal-02649025}
H.~Albrecher and P.-O. Goffard, ``{On the profitability of selfish blockchain
  mining under consideration of ruin},'' {\em To appear in Operations
  Research}, May 2021.
\newblock \url{https://arxiv.org/abs/2010.12577}.

\bibitem{grunspan2019profitability}
C.~Grunspan and R.~Pérez-Marco, ``On profitability of selfish mining,'' 2019.
\end{thebibliography}
\end{frame}
\begin{frame}{Conclusion and perspectives}
\begin{itemize}
  \item Selfish mining entails a waste of resources for everybody
  \item Including a difficulty adjustment can make it strategic
  \item Possible extensions
  \begin{enumerate}
  \item Consider stochastic jumps
  \item Consider the actual selfish mining strategy
  \item Consider generalization of the selfish mining strategy
  \tiny 
  \begin{thebibliography}{1}

\bibitem{Sapirshtein2017}
A.~Sapirshtein, Y.~Sompolinsky, and A.~Zohar, ``Optimal selfish mining
  strategies in bitcoin,'' in {\em Financial Cryptography and Data Security},
  pp.~515--532, Springer Berlin Heidelberg, 2017.

\end{thebibliography}
\begin{itemize}
  \item[$\hookrightarrow$] Stochastic control problem!
\end{itemize}
\item \small Better understanding of the connectivity parameter $q$
\tiny 
\begin{thebibliography}{1}

\bibitem{GoHoKrTa16}
J.~G{\"o}bel, H.~P. Keeler, A.~E. Krzesinski, and P.~G. Taylor, ``Bitcoin
  blockchain dynamics: The selfish-mine strategy in the presence of propagation
  delay,'' {\em Performance Evaluation}, vol.~104, pp.~23--41, 2016.

\end{thebibliography}
\begin{itemize}
  \item[$\hookrightarrow$] Graph theory!
\end{itemize}
\end{enumerate}
\end{itemize}
\end{frame}
% \begin{frame}
% \bibliography{../../blockastics}
% \bibliographystyle{ieeetr}
% \end{frame}

\end{document}
