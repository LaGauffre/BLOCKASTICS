\documentclass{beamer}
\usepackage[utf8]{inputenc}
\usepackage[T1]{fontenc}
% \usepackage{amscd, amsfonts, amsmath, amssymb, amstext, amsthm, caption, epsfig, fancyhdr, float, graphicx, latexsym, mathtools, multicol, multirow, algorithm, chngcntr}
\usepackage[english, french]{babel}
\usepackage{booktabs}

\usepackage{amsmath,amssymb}
\usepackage{graphicx}
\usepackage{caption}
\usepackage{subfig}
\usepackage{xspace}
\usepackage{fourier}

\usepackage{algorithm}
% \usepackage{algorithmic}
\usepackage{algorithmicx}
\usepackage{listings}
\usepackage{algcompatible}
\usepackage{algpseudocode} 

\usepackage{tikz}
\usetikzlibrary{shapes,arrows}
\usepackage{tkz-graph}
\usetikzlibrary{automata,arrows,positioning,calc}
\usetikzlibrary{positioning}
\usetikzlibrary{fit}
\usetikzlibrary{backgrounds}
\usetikzlibrary{calc}
\usetikzlibrary{shapes}
\usetikzlibrary{mindmap}
\usetikzlibrary{decorations.text}
\usetikzlibrary{snakes}

% \theoremstyle{definition} % insert bellow all blocks you want in normal text
% \newtheorem{definition}{Definition}



% tikzmark command, for shading over items
\newcommand{\tikzmark}[1]{\tikz[overlay,remember picture] \node (#1) {};}
% Define block styles
\tikzstyle{decision} = [diamond, draw, fill=blue!20,
    text width=4.5em, text badly centered, node distance=3cm, inner sep=0pt]
\tikzstyle{block} = [rectangle, draw, fill=blue!20,
    text width=5em, text centered, rounded corners]
\tikzstyle{line} = [draw]
\tikzstyle{cloud} = [draw, ellipse,fill=red!20, node distance=3cm,
    minimum height=2em]

\usepackage[most]{tcolorbox}

\setbeamertemplate{blocks}[rounded][shadow=true] % use rounded blocks with standard beamer shadow


% Distributions.
\newcommand*{\UnifDist}{\mathsf{Unif}}
\newcommand*{\ExpDist}{\mathsf{Exp}}
\newcommand*{\DepExpDist}{\mathsf{DepExp}}
\newcommand*{\GammaDist}{\mathsf{Gamma}}
\newcommand*{\LognormalDist}{\mathsf{LogNorm}}
\newcommand*{\WeibullDist}{\mathsf{Weib}}
\newcommand*{\ParetoDist}{\mathsf{Par}}
\newcommand*{\NormalDist}{\mathsf{Norm}}

\newcommand*{\GeometricDist}{\mathsf{Geom}}
\newcommand*{\NegBinomialDist}{\mathsf{NegBin}}
\newcommand*{\PoissonDist}{\mathsf{Poisson}}
\newcommand*{\BivariatePoissonDist}{\mathsf{BPoisson}}
\newcommand*{\CyclicalPoissonDist}{\mathsf{CPoisson}}

\newcommand*{\iid}{\textbf{iid}\@\xspace}
\newcommand*{\pdf}{\textbf{pdf}\@\xspace}
\newcommand*{\cdf}{\textbf{cdf}\@\xspace}
\newcommand*{\pmf}{\textbf{pmf}\@\xspace}
\newcommand*{\abc}{{\textbf{abc}}\@\xspace}
\newcommand*{\smc}{\textbf{smc}\@\xspace}
\newcommand*{\mcmc}{\textbf{mcmc}\@\xspace}
\newcommand*{\ess}{\textbf{ess}\@\xspace}
\newcommand*{\mle}{\textbf{mle}\@\xspace}
\newcommand*{\bic}{\textbf{bic}\@\xspace}
\newcommand*{\kde}{\textbf{kde}\@\xspace}
\newcommand*{\glm}{\textbf{glm}\@\xspace}
\newcommand*{\xol}{\textbf{xol}\@\xspace}
\newcommand*{\cpu}{\textbf{cpu}\@\xspace}
\newcommand*{\gpu}{\textbf{gpu}\@\xspace}
\newcommand*{\arm}{\textbf{arm}\@\xspace}

\def \si {\sigma}
\def \la {\lambda}
\def \al {\alpha}
% \def\e*{\end{eqnarray*}}
\def \di{\displaystyle}

\def \E{\mathbb E}
\def \N{\mathbb N}
\def \Z{\mathbb Z}
\def \NZ{\mathbb{N}_0}
\def \I{\mathbb I}
\def \w{\widehat}
\def \P {\mathbb P}
\def \V{\mathbb V}


\newcommand{\CL}{\mathbb{C}}
\newcommand{\RL}{\mathbb{R}}
\newcommand{\nat}{{\mathbb N}}
\newcommand{\Laplace}{\mathscr{L}}
\newcommand{\e}{\mathrm{e}}
\newcommand{\ve}{\bm{\mathrm{e}}} % vector e

\renewcommand{\L}{\mathcal{L}} % e.g. L^2 loss.

\newcommand{\ih}{\mathrm{i}}
\newcommand{\oh}{{\mathrm{o}}}
\newcommand{\Oh}{{\mathcal{O}}}
\newcommand{\Exp}{\mathbb{E}}

\newcommand{\Norm}{\mathcal{N}}
\newcommand{\LN}{\mathcal{LN}}
\newcommand{\SLN}{\mathcal{SLN}}

\renewcommand{\Pr}{\mathbb{P}}
\newcommand{\Ind}{\mathbb I}
\newcommand\bfsigma{\bm{\sigma}}
\newcommand\bfSigma{\bm{\Sigma}}
\newcommand\bfLambda{\bm{\Lambda}}
\newcommand{\stimes}{{\times}}
\def \limsup{\underset{n\rightarrow+\infty}{\overline{\lim}}}
\def \liminf{\underset{n\rightarrow+\infty}{\underline{\lim}}}

\hypersetup{
    colorlinks=true,
    linkcolor=tublue,
    filecolor=magenta,      
    urlcolor=cyan,
    pdftitle={Overleaf Example},
    pdfpagemode=FullScreen,
    }




% vertical separator macro
\newcommand{\vsep}{
  \column{0.0\textwidth}
    \begin{tikzpicture}
      \draw[very thick,black!10] (0,0) -- (0,7.3);
    \end{tikzpicture}
}
\newcommand\blfootnote[1]{%
  \begingroup
  \renewcommand\thefootnote{}\footnote{#1}%
  \addtocounter{footnote}{-1}%
  \endgroup
}

% More space between lines in align
% \setlength{\mathindent}{0pt}

% Beamer theme
\usetheme{ZMBZFMK}
\usefonttheme[onlysmall]{structurebold}
\mode<presentation>
\setbeamercovered{transparent=10}

% align spacing
\setlength{\jot}{0pt}

\setbeamertemplate{navigation symbols}{}%remove navigation symbols

\title[BLOCKASTICS]{Stochastic Models for blockchain analysis}
% \subtitle{Consensus protocols}
\author{Pierre-O. Goffard}
\institute[ISFA]{Institut de Science Financières et d'Assurances\\
 \texttt{pierre-olivier.goffard@univ-lyon1.fr}
}
\date{\today}
% \titlegraphic{\includegraphics[width=2.5cm]{../../Figures/bfs_logo.png}} 

\begin{document}
\begin{frame}
  \titlepage
\end{frame}
\section{Introduction}
\begin{frame}{Blockchain}
A decentralized data ledger made of blocks maintained by achieving consensus in a P2P network.
\begin{columns}
\begin{column}{0.5\textwidth}
% \small

\begin{itemize}
  \item Decentralized
  \item Public/private
  \item Permissionned/permissionless
  \item Immutable
  \item Incentive compatible
\end{itemize}
\end{column}
\begin{column}{0.5\textwidth}
\begin{center}
\begin{tikzpicture}[-, >=stealth', auto, semithick, node distance=01cm]
\tikzstyle{every edge}=[snake=expanding waves,segment length=1mm,segment angle=10, draw]

\tikzstyle{full node}=[circle, fill=tublue,draw=tublue,thick,text=black,scale=0.8]
\tikzstyle{light node}=[circle, fill=white,draw=tublue,thick,text=black,scale=0.8]
\node[full node]    (1)                     {};
\node[full node]    (2)[above right of=1]         {};
\node[full node]    (3)[above left of=1]         {};
\node[full node]    (4)[below of=1]         {};
\node[full node]    (5)[right of=4]         {};
\node[full node]    (6)[below of=4]         {};
\node[light node]    (7)[left of=1]         {};
\node[light node]    (8)[right of=2]         {};
\node[light node]    (9)[left of=4]         {};
\node[light node]    (10)[above right of=5]         {};
\node[light node]    (11)[ right of=5]         {};
\node[light node]    (12)[ below right of=5]         {};
% \node[light node]    (4)[above of=2]         {};
\path

(1) edge node{} (2)
    edge node{} (3)
    edge node{} (7)
    ;
\path
(5) edge node{} (10)
    edge node{} (11)
    edge node{} (12)
    ;
    \path
(4) edge node{} (5)
    edge node{} (1)
    edge node{} (9)
    edge node{} (6)
    ;
    \path
(2) edge node{} (8)   
    ;
\end{tikzpicture}
\end{center}
\end{column}
\end{columns}


\end{frame}
\begin{frame}{Consensus protocol}
\begin{tcolorbox}[enhanced,drop shadow, title=Definition]
    Algorithm to allows the full nodes to agree on a common data history
\end{tcolorbox}
It must rely on the scarce resources of the network
\begin{itemize}
  \item bandwidth
  \item computational power
  \item storage (disk space)
\end{itemize}
\end{frame}
\begin{frame}{Types of consensus protocols}
\begin{enumerate}
  \item Voting based 
  \tiny 
\begin{thebibliography}{1}

\bibitem{lamport1982the}
L.~Lamport, R.~Shostak, and M.~Pease, ``The byzantine generals problem,'' {\em
  ACM Transactions on Programming Languages and Systems}, pp.~382--401, July
  1982.

\end{thebibliography}
\normalsize\medskip 
  \item Leader based
  \begin{itemize}
    \item Proof-of-Work (computational power)
    \item Proof-of-Capacity and Proof-of-Spacetime (storage)
    \item Proof-of-Interaction (bandwidth) 
%     \tiny
%     \begin{thebibliography}{1}

% \bibitem{Abegg2021}
% J.-P. Abegg, Q.~Bramas, and T.~Noël, ``Blockchain using
%   proof-of-interaction,'' in {\em Networked Systems}, pp.~129--143, Springer
%   International Publishing, 2021.

% \end{thebibliography}
% \normalsize
    \item Proof-of-Stake (tokens)
  \end{itemize}
\end{enumerate}
\end{frame}
\begin{frame}{Conflict resolution in blockchain}
\begin{tcolorbox}[enhanced,drop shadow, title=Fork]
    A fork arises when there is a disagreement between the nodes resulting in several branches in the blockchain.
\end{tcolorbox}
\begin{tcolorbox}[enhanced,drop shadow, title=LCR]
    The \textit{Longest Chain Rule} states that if there exist several branches of the blockchain then the longest should be trusted.
\end{tcolorbox}
In practice 
\begin{itemize}
  \item A branch can be considered legitimate if it is $k\in\mathbb{N}$ blocks ahead of its pursuers.
  \item Fork can be avoided when
  $$
  \text{block appending time}> \text{ propagation delay} 
  $$
\end{itemize}
\end{frame}
\begin{frame}{Blockastics project}
% \begin{tcolorbox}[enhanced,drop shadow, title=Algorithme de consensus]
% Mécanisme premettant aux noeuds du réseau de s'accorder sur un historique de données commun.
% \end{tcolorbox}

% \vspace{0.3cm}
Stochastic models to assess
\medskip
\begin{enumerate}
  \item Efficiency (Queueing models)
  \begin{itemize}
    \item Average number of transactions processed per time units
  \end{itemize}
  \medskip
  \item Decentralization (Stochastic process with reinforcement)
  \begin{itemize}
    \item Distribution of the decision power accross the nodes
  \end{itemize}
  \medskip
  \item Security (Risk theory)
  \begin{itemize}
    \item Resistance to attacks
  \end{itemize}
\end{enumerate}
\blfootnote{\href{https://pierre-olivier.goffard.me/BLOCKASTICS/}{https://pierre-olivier.goffard.me/BLOCKASTICS/}}
\end{frame}

\section{Examples of consensus protocol}
\begin{frame}{What's inside a block?}
A block consists of 
\begin{itemize}
\item a header 
\item a list of "transactions" that represents the information recorded through the blockchain. 
\end{itemize}
The header usually includes 
\begin{itemize}
\item the date and time of creation of the block, 
\item the block height which is the index inside the blockchain, 
\item the hash of the block 
\item the hash of the previous block. 
\end{itemize}
\begin{tcolorbox}[enhanced,drop shadow, title=Question]
What is the hash of a block?
\end{tcolorbox}
\end{frame}

\begin{frame}{Cryptographic Hash function}
\small
A function that maps data of arbitratry size (message) to a bit array of fixed size (hash value)
$$
h:\{0,1\}^\ast\mapsto \{0,1\}^d. 
$$
A good hash function is
\begin{itemize}
\item deterministic
\item quick to compute
\item One way
\begin{itemize}
  \scriptsize
\item[$\hookrightarrow$] For a given hash value $\overline{h}$ it is hard to find a message $m$ such that 
$$
h(m) = \overline{h}
$$
\end{itemize}
\item Colision resistant 
\begin{itemize}
\item[$\hookrightarrow$] Impossible to find $m_1$ and $m_2$ such that 
$$
h(m_1) = h(m_2)
$$
\end{itemize}
\item Chaotic
$$m_1\approx m_2\Rightarrow  h(m_1) \neq h(m_2)$$
\end{itemize}
\end{frame}
\begin{frame}{SHA-256}
The SHA-256 function which converts any message into a hash value of $256$ bits.
\begin{tcolorbox}[enhanced,drop shadow, title=Example]
The hexadecimal digest of the message
$$
\texttt{Blockastics is fantastic}
$$
is 
\footnotesize
$$
\texttt{60a147c28568dc925c347bce20c910ef90f3774e2501ac63344f3411b6a6bf79}
$$
\end{tcolorbox}
\end{frame}
\begin{frame}{Mining a block}
\begin{figure}[!ht]
    \includegraphics[width = \textwidth]{../../Figures/block_not_mined.png}
    \captionsetup{width=0.8\textwidth}
    \centering
    \caption{A block that has not been mined yet.}
    \label{fig:block_not_mined}
\end{figure}
\end{frame}
\begin{frame}{Mining a block}
The maximum value for a 256 bits number is
$$
T_\text{max} = 2^{256}-1 \approx 1.16e^{77}.
$$
Mining consists in drawing at random a nonce 
$$
\text{Nonce} \sim \text{Unif}(\{0,\ldots, 2^{32}-1\}),
$$
until 
$$
h(\text{Nonce}|\text{Block info})<T,
$$
where $T$ is referred to as the target.
\begin{tcolorbox}[enhanced,drop shadow, title=Difficulty of the cryptopuzzle]
$$
D = \frac{T_{\max}}{T}.
$$
\end{tcolorbox}

\end{frame}
\begin{frame}{Mining a block}
If we set the difficulty to $D = 2^4$ then the hexadecimal digest must start with at least $1$ leading $0$
\begin{figure}[!ht]
    \includegraphics[width = \textwidth]{../../Figures/block_mined.png}
    \captionsetup{width=0.8\textwidth}
    \centering
    \caption{A mined block with a hash value having on leading zero.}
    \label{fig:block_mined}
\end{figure}
The number of trial is geometrically distributed
\begin{itemize}
\item Exponential inter-block times
\item Lenght of the blockchain = Poisson process
\end{itemize}
\end{frame}
\begin{frame}{Bitcoin protocol}
\begin{itemize}
  \item One block every 10 minutes on average
  \item Depends on the hashrate of the network
  \item Difficulty adjustment every 2,016 blocks ($\approx$ two weeks)
  \item Reward halving every 210,000 blocks
\end{itemize}
Check out \url{https://www.bitcoinblockhalf.com/}
\end{frame}
\begin{frame}{Mining equipments}
How it started
\begin{itemize}
  \item CPU, GPU
\end{itemize}
How it is going
\begin{itemize}
  \item Application Specific Integrated Chip (ASIC)
  \begin{itemize}
  \item Increase of the network electricity consumption \url{https://digiconomist.net/bitcoin-energy-consumption}
  \item E-Waste
  \item Centralization issue \url{https://www.bitmain.com/}
  \begin{itemize}
    \item Mining pool ranking at \url{https://btc.com/}
    \item Mining equipment profitability at \url{https://v3.antpool.com/minerIncomeRank}
  \end{itemize}
  \end{itemize}
\end{itemize}
\end{frame}

\begin{frame}{Proof of Stake}
PoW is slow and ressource consuming. Let $\{1,\ldots, N\}$ be a set of miners and $\{\pi_1,\ldots, \pi_N\}$ be their share of cryptocoins.
\begin{tcolorbox}[enhanced,drop shadow, title=PoS]
\begin{enumerate}
\item Node $i\in \{1,\ldots, N\}$ is selected with probability $\pi_i$ to append the next block
\end{enumerate}
\end{tcolorbox}
\vspace{0.3cm}
Nodes are chosen according to what they own.
\begin{itemize}
  \item Nothing at stake problem
  \item Rich gets richer ? 
  \item \url{https://www.peercoin.net/}
\end{itemize}
\footnotesize{
\begin{thebibliography}{1}
\bibitem{Saleh2020}
F.~Saleh, ``Blockchain without waste: Proof-of-stake,'' {\em The Review of
  Financial Studies}, vol.~34, pp.~1156--1190, jul 2020.
\end{thebibliography}}
\end{frame}
% \begin{frame}{Using storage}
% \begin{itemize}
%   \item Proof-Of-Capacity: 
%   \begin{enumerate}
%     \item Store solution to the PoW cryptopuzzle in a large file
%     \item Read through the cache for an acceptable solution
%   \end{enumerate}
%   \item Proof-of-Spacetime
%     \begin{enumerate}
%     \item Deliver proof of storing some data during some time
%     \item Probability of being selected proportional to the amount of data stored times duration of the storage
%     \item \url{https://filecoin.io/}
%   \end{enumerate}
% \end{itemize}
% \end{frame}
\begin{frame}{Using bandwidth}
Proof-of-Interaction 
\begin{itemize}
  \item The node receives a list of node they must get in touch with 
  \item The first one who is able to complete the task gets a reward and share it with the responding nodes
\end{itemize}
\footnotesize{
\begin{thebibliography}{1}

\bibitem{Abegg2021}
J.-P. Abegg, Q.~Bramas, and T.~Noël, ``Blockchain using
  proof-of-interaction,'' in {\em Networked Systems}, pp.~129--143, Springer
  International Publishing, 2021.

\end{thebibliography}}
\medskip\normalsize
For an up-to-date list of consensus protocol\\
\begin{center}
\url{https://tokens-economy.gitbook.io/consensus/}
\end{center}
\end{frame}
\section{Stochastic Models:}

\subsection{Security of PoW blockchain}
\begin{frame}{Double spending attack}
\scriptsize
\begin{enumerate}
\item Mary transfers 10 BTCs to John
\item The transaction is recorded in the public branch of the blockchain and John ships the good.
\item Mary transfers to herself the exact same BTCs
\item The malicious transaction is recorded into a private branch of the blockchain
\begin{itemize}
  \scriptsize
\item Mary has friends among the miners to help her out
\item The two chains are copycat up to the one transaction
\end{itemize}
\end{enumerate}
\begin{tcolorbox}[enhanced,drop shadow, title=Fact (Bitcoin has only one rule)]
The longest chain is to be trusted
\end{tcolorbox}
\end{frame}
\begin{frame}{Double spending in practice}
\scriptsize
Vendor are advised to wait for $\alpha\in\mathbb{N}$ of confirmations so that the honest chain is ahead of the dishonest one.
\begin{center}
\begin{tikzpicture}[-, >=stealth', auto, semithick, node distance=1cm]
% \tikzstyle{block} = [rectangle, draw, fill=blue!20,
%     text width=5em, text centered, rounded corners]
\tikzstyle{block}=[rectangle, fill=black,draw=black,thick,text=black,scale=0.6]
\tikzstyle{block}=[rectangle, fill=white,draw=black,thick,text=black,scale=0.8]
\tikzstyle{confirmed block}=[rectangle, fill=white,draw=blue,thick,text=black,scale=0.8]
\tikzstyle{bad block}=[rectangle, fill=white,draw=red,thick,text=black,scale=0.8]
\node[block]    (1)                     {\tiny $\text{M}\rightarrow \text{J}$};
\node[block]    (2)[right of=1]                     {};
\node[block]    (3)[right of=2]                     {};
\node[block]    (4)[right of=3]                     {};
\node[confirmed block]    (5)[right of=4]                     {};

\node[bad block]    (6)[below of=1]         {\tiny $\text{M}\rightarrow \text{M}$};
\node[block]    (7)[right of=6]         {};
\node[block]    (8)[right of=7]         {};
\path
(1) edge[ left]     node{}     (2)
(2) edge[ left]     node{}     (3)
(3) edge[ left]     node{}     (4)
(4) edge[ left]     node{}     (5)
(6) edge[ left]     node{}     (7)
(7) edge[ left]     node{}     (8);

\end{tikzpicture}
\end{center}
In the example, vendor awaits $\alpha = 4$ confirmations, the honest chain is ahead of the dishonest one by $z = 2$ blocks.
\begin{tcolorbox}[enhanced,drop shadow, title=Fact (PoW is resistant to double spending)]
\begin{itemize}
\item Attacker does not own the majority of computing power 
\item Suitable $\alpha$ 
\end{itemize}
Double spending is unlikely to succeed.
\end{tcolorbox}
\tiny
\begin{thebibliography}{1}
\bibitem{Na08}
S.~Nakamoto, ``Bitcoin: A peer-to-peer electronic cash system.'' Available at
  \href{https://bitcoin.org/bitcoin.pdf}{https://bitcoin.org/bitcoin.pdf},
  2008.
  
\end{thebibliography}

\end{frame}
\begin{frame}{Mathematical set up}
\scriptsize
Assume that
\begin{itemize}
\item $R_0=z\geq1$ (the honest chain is z blocks ahead)
\item at each time unit a block is created
\begin{itemize}
  \scriptsize
\item[$\hookrightarrow$] in the honest chain with probability $p$
\item[$\hookrightarrow$] in the dishonest chain with probability $q=1-p$
\end{itemize}
\end{itemize}
The process $(R_n)_{n\geq0}$ is a random walk on $\mathbb{Z}$ with
$$R_n=z+Y_1+\ldots+Y_n,$$
where $Y_1,\ldots,Y_n$ are the \textbf{i.i.d.} steps of the random walk. 

\end{frame}
\begin{frame}{Double spending rate of success}
\scriptsize
Double spending occurs at time
$$
\tau_z=\inf\{n\in \mathbb{N}\text{ ; }R_n=0\}.
$$
\begin{tikzpicture}
  %Origin and axis
  \coordinate (O) at (0,0);
  \draw[->] (-1,0) -- (9,0) coordinate[label = {below:$n$}] (xmax);
  \draw[->] (0,-0.5) -- (0,3) coordinate[label = {left:$Z_n$}] (ymax);
  %Lower linear boundary

 
  %Stochastic process trajectory
  
  \draw (0,0) node[tublue,left] {} node{};
  \draw[very thick,tublue,-] (0,1) -- (1,1) node[pos=0.5, above] {} ;
  \draw[very thick,dashed,tublue] (1,1) -- (1,1.5) node[pos=0.5, right] {};
  \draw[very thick,tublue,-] (1,1.5) -- (2,1.5) node[pos=0.5, above] {};
  \draw[very thick,dashed,tublue] (2,1.5) -- (2,2) node[pos=0.5, right] {};
  \draw[very thick,tublue,-] (2,2) -- (3,2) node[pos=0.5, above] {};
  \draw[very thick,dashed,tublue] (3,2) -- (3,1.5) node[pos=0.5, right] {};
  \draw[very thick,tublue,-] (3,1.5) -- (4,1.5)node[pos=0.5, above] {};
  \draw[very thick,dashed,tublue] (4,1.5) -- (4,1) node[pos=0.5, right] {};  
  \draw[very thick,tublue,-] (4,1) -- (5,1) node[pos=0.5, above] {};
  \draw[very thick,dashed,tublue] (5,1) -- (5,0.5) node[pos=0.5, right] {};  
  \draw[very thick,tublue,-] (5,0.5) -- (6,0.5) node[pos=0.5, above] {};
  \draw[very thick,dashed,tublue,-] (6,0.5) -- (6,1) node[pos=0.5, above] {};
   \draw[very thick,tublue,-] (6,1) -- (7,1) node[pos=0.5, above] {};
    \draw[very thick,dashed,tublue,-] (7,1) -- (7,0.5) node[pos=0.5, above] {};
     \draw[very thick,tublue,-] (7,0.5) -- (8,0.5) node[pos=0.5, above] {};
     \draw[very thick,dashed,tublue,-] (8,0.5) -- (8,0) node[pos=0.5, above] {};
  %Jump Times
  \draw (1,0) node[black,below] {$1$} node{ \color{black}$\bullet$};
  \draw (2,0) node[black,below] {$2$} node{ \color{black}$\bullet$};
  \draw (3,0) node[black,below] {$3$} node{ \color{black}$\bullet$};
  \draw (4,0) node[black,below] {$4$} node{ \color{black}$\bullet$};
  \draw (5,0) node[black,below] {$5$} node{ \color{black}$\bullet$};
  \draw (6,0) node[black,below] {$6$} node{ \color{black}$\bullet$};
  \draw (7,0) node[black,below] {$7$} node{ \color{black}$\bullet$};
  \draw (8,0) node[black,below] {$8$} node{ \color{black}$\bullet$};
  %Level of the counting process
   \draw (0,0) node[black,below left] {$0$} node{ \color{black}$\bullet$};
   \draw (0,0.5) node[black,left] {$1$} node{ \color{black}$\bullet$};
   \draw (0,1) node[black,left] {$z=2$} node{ \color{black}$\bullet$};
   \draw (0,1.5) node[black,left] {$3$} node{ \color{black}$\bullet$};
   \draw (0,2) node[black,left] {$4$} node{ \color{black}$\bullet$};
   \draw (0,2.5) node[black,left] {$5$} node{ \color{black}$\bullet$};

  % %Aggregated Capital gains
%  \draw (0,1.5) node[blue,below right] {$\mu_1$} node{ \color{blue}$-$};
%  \draw (0,2.25) node[blue,left] {$\mu_2$} node{ \color{blue}$-$};
%  \draw (0,3.75) node[blue,left] {$\mu_3$} node{ \color{blue}$-$};
  %Ruin time = First-crossing time time
%  \draw (5,0) node[black,above right] {$\tau_u$} node{ \color{black}$\times$};
%  \draw[dotted,black] (0,3.28) -- (5,3.28);
%  \draw[dotted,black] (5,0) -- (5,3.28);
\end{tikzpicture}
\begin{tcolorbox}[enhanced,drop shadow, title=Double spending theorem]
If $p>q$ then the double-spending probability is given by
$$
\phi(z) = \mathbb{P}(\tau_z<\infty)=\left(\frac{q}{p}\right)^{z}.
$$
\end{tcolorbox}

\end{frame}

\begin{frame}[allowframebreaks]{Proof of the double spending theorem }
\scriptsize Analogy with the gambler's ruin problem. Using a first step analysis, we have 
\begin{equation}\label{eq:difference_equation}
\phi(z) = p\phi(z+1)+(1-p)\phi(z-1),\text{ }z\geq1.
\end{equation}
We also have the boundary conditions
\begin{equation}\label{eq:boundary_conditions_double_spending}
\phi(0) = 1\text{ and }\underset{z\rightarrow +\infty}{\lim}\phi(z) = 0
\end{equation}
Equation \eqref{eq:difference_equation} is a linear difference equation of order $2$ associated to the following characteristic equation
$$
px^2 - x + 1-p = 0
$$
which has two roots on the real line with 
$$
r_1 = 1, \text{ and }r_2 = \frac{1-p}{p}.
$$
The solution of \eqref{eq:difference_equation} is given by 
$$
\phi(z)=A+B\left(\frac{1-p}{p}\right)^z,
$$
where $A$ and $B$ are constant. Using the boudary conditions \eqref{eq:boundary_conditions_double_spending}, we deduce that
$$
\phi(z) = \left(\frac{1-p}{p}\right)^z
$$
as announced.
\end{frame}
\begin{frame}{Refinements of the double spending problem}
\scriptsize
The number of blocks $M$ found by the attacker until the honest miners find $\alpha$ blocks is a negative binomial random variable with \pmf
$$
\mathbb{P}(M = m) = \binom{\alpha+m-1}{m}p^\alpha q^m,\text{ }m\geq0.
$$
The number of block that the honest chain is ahead of the dishonest one is given by 
$$
Z= (\alpha-M)_+.
$$
Applying the law of total probability yields the probability of successful double spending with
$$
\mathbb{P}(\text{Double Spending}) = \mathbb{P}(M\geq \alpha) + \sum_{m = 0}^{\alpha - 1}\binom{\alpha+m-1}{m}q^{\alpha} p^{m}.
$$ 
\tiny


\begin{thebibliography}{1}
  \bibitem{rosenfeld2014analysis}
M.~Rosenfeld, ``Analysis of hashrate-based double spending,'' {\em arXiv
  preprint arXiv:1402.2009}, 2014.
  \bibitem{GRUNSPAN2018}
C.~Grunspan and R.~Perez-Marco, ``Double spend race,'' {\em
  International Journal of Theoretical and Applied Finance}, vol.~21,
  p.~1850053, dec 2018.
\end{thebibliography}

\end{frame}
\begin{frame}{Refinements of the double spending problem}
\scriptsize
Let the length of honest and dishonest chain be driven by counting processes
\begin{itemize}
\item Honest chain $\Rightarrow$ $z+N_t\text{ , }t\geq0$, where $z\geq1$.
\item Malicious chain $\Rightarrow$ $M_t\text{ , }t\geq0$
\item Study the distribution of the first-\textit{rendez-vous} time
$$
\tau_z=\inf\{t\geq0\text{ , } M_t=z+N_t\}.
$$
\end{itemize}
If $N_t\sim\text{Pois}(\lambda t)$ and $M_t\sim\text{Pois}(\mu t)$ such that $\lambda>\mu$ then 
$$
\phi(z) = \left(\frac{\mu}{\lambda}\right)^z,\text{ }z\geq 0.
$$
\tiny
\begin{thebibliography}{1}

\bibitem{Goffard2019}
P.-O. Goffard, ``Fraud risk assessment within blockchain transactions,'' {\em
  Advances in Applied Probability}, vol.~51, pp.~443--467, jun 2019.
\newblock \url{https://hal.archives-ouvertes.fr/hal-01716687v2}.

\bibitem{Bowden2020}
R.~Bowden, H.~P. Keeler, A.~E. Krzesinski, and P.~G. Taylor, ``Modeling and
  analysis of block arrival times in the bitcoin blockchain,'' {\em Stochastic
  Models}, vol.~36, pp.~602--637, jul 2020.
\end{thebibliography}

\end{frame}
\begin{frame}{Perspectives}


\begin{itemize}
\item Include network delay 
\tiny
\begin{thebibliography}{1}

\bibitem{Dembo2020}
A.~Dembo, S.~Kannan, E.~N. Tas, D.~Tse, P.~Viswanath, X.~Wang, and O.~Zeitouni,
  ``Everything is a race and nakamoto always wins,'' in {\em Proceedings of the
  2020 {ACM} {SIGSAC} Conference on Computer and Communications Security},
  {ACM}, oct 2020.
\end{thebibliography}
\normalsize\medskip
\item Double spending in block-DAGS
\tiny
\begin{thebibliography}{1}
\bibitem{Anceaume_2018}
E.~Anceaume, A.~Guellier, R.~Ludinard, and B.~Sericola, ``Sycomore: A
  permissionless distributed ledger that self-adapts to transactions demand,''
  in {\em 2018 {IEEE} 17th International Symposium on Network Computing and
  Applications ({NCA})}, {IEEE}, nov 2018.

\end{thebibliography}
\end{itemize}
\end{frame}
\subsection{Decentralization in PoS blockchain}
\begin{frame}{Proof of Stake protocol}
\scriptsize
PoS is the most popular alternative to PoW.
\begin{itemize}
  \item A block validator is selected according to the number of native coins she owns
  \item Update the blockchain and receive a reward or do nothing  
\end{itemize}
Two problems 
\begin{itemize}
  \item[\warning] Nothing at stake $\Rightarrow$ Consensus postponed
  \item[\warning] Rich gets richer $\Rightarrow$ Risk of centralization
\end{itemize}

\end{frame}
% \begin{frame}{Nothing-at-Stake}
% \scriptsize
% If given the opportunity a node will always append a new block
% \begin{itemize}
%   \item Everlasting fork if any
% \end{itemize}
% Perpetuating disagreement prevent users to exchange which lower the coin value.
% \begin{tcolorbox}[enhanced,drop shadow, title=Theorem]
% To get consensus faster and almost surely 
% \begin{itemize}
%   \item Set a minimum stake to outweight the benefit of the reward
%   \item Set up a modest reward schedule $\sum_{t = 1}^\infty \text{r}_t<\infty$
% \end{itemize}
% \end{tcolorbox}
% \tiny
% \begin{thebibliography}{1}

% \bibitem{Saleh2020}
% F.~Saleh, ``Blockchain without waste: Proof-of-stake,'' {\em The Review of
%   Financial Studies}, vol.~34, pp.~1156--1190, jul 2020.
% \end{thebibliography}
% \end{frame}
\begin{frame}{Risk of centralization ?}
\scriptsize
\begin{columns}
\begin{column}{0.5\textwidth}
Block appending process
\begin{itemize}
  \item Draw a coin at random
  \item The owner of the coin append a block and collect the reward
  \item The block appender is more likely to get selected during the next round
\end{itemize}
\end{column}
\begin{column}{0.5\textwidth}
Similar to Polya's urn \includegraphics[scale=0.1]{../../Figures/poly_urn.png}
\begin{itemize}
  \item Consider an urn of $N$ balls of color in $E=\{1,\ldots, p\}$
  \item Draw a ball of color $x\in E$
  \item Replace the ball together with $r$ balls of color $x$ 
\end{itemize}
$p$ is the number of peers and $r$ is the size of the block reward. 
\end{column}
\end{columns}
\begin{tcolorbox}[enhanced,drop shadow, title=Theorem]
The proportion of coins owned by each peer is stable on average over the long run
\end{tcolorbox}

\tiny
\begin{thebibliography}{1}

\bibitem{Rosu2021}
I.~Ro{\c{s}}u and F.~Saleh, ``Evolution of shares in a proof-of-stake
  cryptocurrency,'' {\em Management Science}, vol.~67, pp.~661--672, feb 2021.
  \end{thebibliography}
\end{frame}
\begin{frame}{Proof}
\scriptsize
Consider the balls of some color $x\in E$, and denote by 
\begin{itemize}
\item $N_x$ the number of balls of color $x$ initially in the urn
\item $Y_n$ the number of balls of color $x$ in the urn after $n$ draws
\item $Z_n$ the corresponding proportion of balls of color $x$.
\end{itemize}   
We show that $(Z_n)_{n\geq0}$ is a $\mathcal{F}_n-$Martingale where $\mathcal{F}_n=\sigma(Y_1,\ldots, Y_n)$. We have 
\begin{eqnarray*}
\mathbb{E}(Z_{n+1}|\mathcal{F}_n) = Z_n\frac{Y_n+r}{N+r(n+1)} +(1-Z_n)\frac{Y_n}{N+r(n+1)} = Z_n
\end{eqnarray*}
It follows that 
$$
\mathbb{E}(Z_n) = \mathbb{E}(Z_0) = \frac{N_x}{N}, \text{ for }n\geq0.
$$
hence the stability. Furthermore, because $|Z_n|<1$, then $\underset{n\rightarrow\infty}{\lim} Z_n = Z_\infty$ exists and it holds that $\mathbb{E}(Z_\infty) = \mathbb{E}(Z_0)$.\\
\end{frame}
\begin{frame}{What is the limiting distributions of the shares?}
\scriptsize
\begin{tcolorbox}[enhanced,drop shadow, title=Dirichlet distribution]
A random vector $(Z_1,\ldots, Z_p)$ has a Dirichlet distribution $\text{Dir}(\alpha_1,\ldots, \alpha_p)$ with \pdf
$$
f(z_1,\ldots, z_p;\alpha_1,\ldots, \alpha_p) = \frac{1}{B(\alpha)}\prod_{i=1}^p z_i^{\alpha_i-1}, 
$$
for $\alpha_1,\ldots, \alpha_p>0$, $0< z_1,\ldots, z_p <1$ and $\sum_{i=1}^pz_i=1$, where 
$$
B(\alpha) = \frac{\prod_{i = 1}^p \Gamma(\alpha_i)}{\Gamma(\sum_{i=1}^p \alpha_i)}.
$$
\end{tcolorbox}
\begin{tcolorbox}[enhanced,drop shadow, title=Theorem (Convergence toward a Dirichlet distribution)]
Suppose that $r=1$ and let $X_n$ be the color of the ball drawn at the $n^{th}$ round then 
$$
\{\mathbb{P}(X_\infty = x), x\in E\} \sim \text{Dir}(\{N_x\text{, }x\in E\}).
$$
\end{tcolorbox}
\end{frame}
% \begin{frame}[allowframebreaks]{Proof}
% \scriptsize
% We have that 
% \begin{equation}\label{eq:polya_sequence_1}
% \mathbb{P}(X_1=x) = \frac{N_x}{N}
% \end{equation}
% and 
% \begin{equation}\label{eq:polya_sequence_2}
% \mathbb{P}(X_{n+1}=x) = \frac{N_x + \sum_{i=1}^n\delta_{X_i}(x)}{N+n} = \frac{N_x + \lambda_n(x)}{N+n} = m_n(x)
% \end{equation}
% where $\delta_{X_i}$ denotes the Dirac measure at $X_i$.\\
% A sequence that satisfies \eqref{eq:polya_sequence_1} and \eqref{eq:polya_sequence_2} is said to be a Polya sequence with parameter $N_x\text{, }x\in E$.

% \begin{lemma}
% There is an equivalence between the two following statements
% \begin{itemize}
% \item[(i)] $X_1,X_2,\ldots,$ is a Polya sequence
% \item[(ii)] $\mu^{\ast}\sim \text{Dir}(N_x,x\in E)$ and $X_1,X_2,\ldots$ given $\mu^\ast$ are \iid as $\mu^\ast$
% \end{itemize}
% \end{lemma}
% Consider the event $A_n = \{X_1 = x_1,\ldots, X_n = x_n\}$. Induction on $n$ allows us to show that (i) is equivalent to 
% \begin{equation}\label{eq:P_A_polya_i}
% \mathbb{P}(A_n) = \frac{\prod_{x\in E} N_x^{[\lambda_n(x)]}}{N^{[n]}},
% \end{equation}
% where $\lambda_n(x)$ is the number of $i$'s in $1,\ldots, n$ for which $x_i = x$ and $a^{[k]} = a(a+1)\ldots(a+k-1)$.  Now assume that $(ii)$ holds true, then 
% $$
% \mathbb{P}(A_n|\mu^\ast) = \prod_{x\in E}\mu^\ast(x)^{\lambda_n(x)},
% $$
% recall that $\mu^\ast$ is a random vector, indexed on $E$, We denote by $\mu^\ast(x)$ the component associated with $x\in E$. The law of total probability then yields
% \begin{equation}\label{eq:P_A_polya_ii}
% \mathbb{P}(A_n) = \E\left[\prod_{x\in E}\mu^\ast(x)^{\lambda_n(x)}\right],
% \end{equation}
% which is the same as \eqref{eq:P_A_polya_i}. Applying the lemma together with the law of large number yields 
% $$
% n^{-1}\sum_{i=1}^n\delta_{X_i}(x) \rightarrow \mu^{\ast}(x)\text{ as } n\rightarrow\infty.
% $$
% and then $m_n(x)\rightarrow\mu^{\ast}(x)$.
% \tiny
% \begin{thebibliography}{1}
% \bibitem{Blackwell1973}
% D.~Blackwell and J.~B. MacQueen, ``Ferguson distributions via polya urn
%   schemes,'' {\em The Annals of Statistics}, vol.~1, mar 1973.
%   \end{thebibliography}
% \end{frame}
% \begin{frame}{Measuring decentrality}
% \scriptsize
% \begin{tcolorbox}[enhanced,drop shadow, title=Fact]
% The most desirable situation corresponds to all the peers being equally likely to be selected. 
% \end{tcolorbox}
% Decentrality maybe measure by Shannon's entropy
% $$
% H(\mu^\ast) = -\mathbb{E}\left\{\sum_x \mu^\ast(x)\ln[\mu^\ast(x)]\right\} = -\sum_x\frac{N}{N_x}\left[\psi(N_x+1)-\psi(N+1)\right],
% $$
% where $\psi(x) = \frac{\text{d}}{\text{d}x}\ln[\Gamma(x)]$ is the digamma function, to be compared to $\ln(p)$
% \tiny
% \begin{thebibliography}{1}

% \bibitem{Gochhayat2020}
% S.~P. Gochhayat, S.~Shetty, R.~Mukkamala, P.~Foytik, G.~A. Kamhoua, and
%   L.~Njilla, ``Measuring decentrality in blockchain based systems,'' {\em
%   {IEEE} Access}, vol.~8, pp.~178372--178390, 2020.

% \end{thebibliography}

% \end{frame}
\begin{frame}{Extensions and perspectives}
\begin{itemize}
  \item How to include more peers along the way?
  \item What if the peers are not simply buy and hold investors?
  \item Find ways to monitor decentralization and take action if necessary
\end{itemize}
\tiny
\begin{thebibliography}{1}

\bibitem{Rosu2021}
I.~Ro{\c{s}}u and F.~Saleh, ``Evolution of shares in a proof-of-stake
  cryptocurrency,'' {\em Management Science}, vol.~67, pp.~661--672, feb 2021.
  \end{thebibliography} 

\end{frame}
\subsection{Blockchain efficiency}
\begin{frame}{Efficiency}
Efficiency is characterized by 
\begin{itemize}
  \item Throughputs: Number of transaction being processed per time unit
  \item Latency: Average transaction confirmation time
\end{itemize}
We focus on a PoW equipped blockchain and study the above quantities using a queueing model.
\end{frame}
\begin{frame}{Queue settings}
\begin{center}
\begin{tikzpicture}[-, >=stealth', auto, semithick, node distance=1cm]

\tikzstyle{phantom block}=[rectangle, fill=white,draw=white, thick,text=black,scale=2]
\tikzstyle{block}=[rectangle, fill=white,draw=black,thick,text=black,scale=4]
\tikzstyle{Intensity}=[circle, fill=white,draw=tublue,very thick, text=black,scale=1.2]
\tikzstyle{transaction pending}=[circle, fill=white,draw=tublue,very thick, text=black,scale=1]
\tikzstyle{transaction considered}=[circle, fill=tublue, text=black, scale=1]
\node[Intensity]    (1){$\lambda$};
\node[phantom block]  (2)[right of=1] {};
\node[transaction pending] (3)[right of=2] {};
\node[transaction pending] (4)[above of=3] {};
\node[transaction pending] (5)[above of=4] {};
\node[transaction pending] (6)[above of=5] {};
\node[transaction pending] (7)[below of=3] {};
\node[transaction pending] (8)[below of=7] {};
\path
(1) edge[->,bend left]     node{} (4)
    edge[->, bend right]     node{}          (7);
\pause
\node[Intensity]  (14)[above of=6]{$\mu$};
% \path
% (1) edge[->,bend left]     node{$\text{Exp}(\lambda)$}        (4)
%     edge[->, bend right]     node{}          (7);
\pause
\node[transaction considered] (3)[right of=2] {};
\node[transaction considered] (4)[above of=3] {};
\node[transaction considered] (5)[above of=4] {};
\node[transaction considered] (6)[above of=5] {};

\pause
\node[phantom block]  (10)[right of=3] {};
\node[block]  (11)[right of=10] {};
\path
(6) edge[->]   node{} (11)
(5) edge[->]   node{} (11)
(4) edge[->]   node{} (11)
(3) edge[->]   node{} (11)
;
\node[Intensity]  (12)[above of=11] {$\mu$};
\node[phantom block]  (13)[above of=12] {};

\path
(11) edge[-]     node{}        (12);
\path
(12) edge[->]     node{}        (13);

\end{tikzpicture}
\end{center}
\end{frame}
\begin{frame}{Queueing setting}
\begin{itemize}
\item Poisson arrival with rate $\lambda>0$ for the transactions
\item Poisson arrival with rate $\mu>0$ for the blocks 
\item Block size $b\in\mathbb{N}^\ast$
$\Rightarrow $Batch service
\item[\warning] The server is always busy
\end{itemize}
This is somekind of $M/M^b/1$ queue.
\tiny
\begin{thebibliography}{1}
\bibitem{Kawase2017}
Y.~Kawase and S.~Kasahara, ``Transaction-confirmation time for bitcoin:
  A~queueing analytical approach to~blockchain~mechanism,'' in {\em Queueing
  Theory and Network Applications}, pp.~75--88, Springer International
  Publishing, 2017.
\bibitem{Bailey1954}
N.~T.~J. Bailey, ``On queueing processes with bulk service,'' {\em Journal of
  the Royal Statistical Society: Series B (Methodological)}, vol.~16,
  pp.~80--87, jan 1954.

\bibitem{Cox1955}
D.~R. Cox, ``The analysis of non-markovian stochastic processes by the
  inclusion of supplementary variables,'' {\em Mathematical Proceedings of the
  Cambridge Philosophical Society}, vol.~51, pp.~433--441, jul 1955.

\end{thebibliography}
\end{frame}

\begin{frame}{Queue length distribution}
\scriptsize
The queueuing process eventually reaches stationarity if 
\begin{equation}\label{eq:stationarity_cond}
\mu\cdot b > \lambda.
\end{equation}
We denote by $N^q$ the length of the queue upon stationarity. 
\begin{tcolorbox}[enhanced,drop shadow, title=The blockchain efficiency theorem]
Assume that \eqref{eq:stationarity_cond} holds then $N^q$ is geometrically distributed 
$$
\mathbb{P}(N^q = n) = (1-p)\cdot p^n,
$$
where $p = 1/z^\ast$ and $z^\ast$ is the only root of 
$$
-\frac{\lambda}{\mu}z^{b+1}+z^b\left(\frac{\lambda}{\mu}+1\right) - 1,
$$
such that $|z^\ast$|>1.  
\end{tcolorbox}
\end{frame}
\begin{frame}[allowframebreaks]{Proof of the efficiency theorem}
\scriptsize
Let $N^q_t$ be the number of transactions in the queue at time $t\geq0$ and $X_t$ the time elapsed since the last block was found. Further define
\[
P_{n}(x,t)\text{d}x  =\mathbb{P}[N_t^q = n, X_t \in(x, x + \text{dx})] 
\]
If $\lambda < \mu\cdot b$ holds then the process admits a limiting distribution given by 
\[
\underset{t\rightarrow\infty}{\lim}P_{n}(x,t) = P_{n}(x).
\]
We aim at finding the distribution of the queue length upon stationarity
\begin{equation}\label{eq:alpha_n}
\mathbb{P}(N^q=n):=\alpha_n =\int_{0}^\infty P_{n}(x)\text{d}x.
\end{equation}
Consider the possible transitions over a small time lapse \text{h} during which no block is being generated. Over this time interval, either 
\begin{itemize}
  \item no transactions arrives
  \item one transaction arrives
\end{itemize}
We have for $n\geq1$
\[
P_{n}(x+h) = e^{-\mu h}\left[e^{-\lambda h}P_{n}(x)+\lambda h e^{-\lambda h}P_{n-1}(x)\right]
\]
Differentiating with respect to $h$ and letting $h\rightarrow0$ leads to 
\begin{equation}\label{eq:diff_eq_n_geq_1}
P_{n}'(x) = -(\lambda+\mu)P_{n}(x)+\lambda P_{n-1}(x),\text{ }n \geq1.
\end{equation}
Similarly for $n = 0$, we have 
\begin{equation}\label{eq:diff_eq_n_eq_0}
P_{0}'(x) = -(\lambda+\mu)P_{0}(x).
\end{equation}
We denote by 
$$
\xi(x)\text{d}x =\mathbb{P}(x\leq X< x+\text{d}x|X\geq x)= \mu\text{d}x
$$
the hazard function of the block arrival time (constant as it is exponentially distributed). The system of differential equations \eqref{eq:diff_eq_n_geq_1}, \eqref{eq:diff_eq_n_eq_0} admits boundary conditions at $x = 0$ with 
\begin{equation}\label{eq:boundary_cond_1}
\begin{cases}
P_{n}(0) = \int_0^{+\infty} P_{n+b}(x)\xi(x)\text{d}x = \mu\alpha_{n+b},&n \geq1,\\
P_{0}(0) = \mu\sum_{n=0}^{b}\alpha_n,&n = 0,\ldots,b\\
\end{cases}
\end{equation}
Define the probability generating function of $N^q$ at some elapsed service time $x\geq 0$ as 
$$
G(z;x) = \sum_{n=0}^\infty P_{n}(x)z^n.
$$
By differentiating with respect to $x$, we get (using \eqref{eq:diff_eq_n_geq_1} and \eqref{eq:diff_eq_n_eq_0})
$$
\frac{\partial}{\partial x}G(z;x) = -\left[\lambda(1-z)+\mu\right]G(z;x)
$$
and therefore
$$
G(z;x) = G(z;0)\exp\left\{-\left[\lambda(1-z)+\mu\right]x\right\}
$$
We get the probability generating function of $N^q$ by integrating over $x$ as 
\begin{equation}\label{eq:G_z_solve_ODE}
G(z) = \frac{G(z;0)}{\lambda(1-z)+\mu}
\end{equation}
Using the boundary conditions \eqref{eq:boundary_cond_1}, we write 
\begin{eqnarray}
G(z;0) &= &\sum_{n = 0}^\infty P_{n}(0)z^n \nonumber\\
&= &P_{0}(0)+\sum_{n=1}^{+\infty}P_{n}(0)z^n\nonumber\\
&=& \mu\sum_{n = 0}^{b}\alpha_n  + \mu\sum_{n=1}^{+\infty}\alpha_{n+b} z^n\nonumber\\
&=& \mu\sum_{n = 0}^{b}\alpha_n + \mu z^{-b}\left[G(z)-\sum_{n = 0}^{b}\alpha_n z^n\right]\label{eq:G_z_0}
\end{eqnarray}
Replacing the left hand side of \eqref{eq:G_z_0} by \eqref{eq:G_z_solve_ODE}, multiplying on both side by $z^b$ and rearranging yields 
\begin{equation}\label{eq:G_z_as_rational_function}
\frac{G(z)}{M(z)}[z^b - M(z)] =\sum_{n=0}^{b-1}\alpha_n(z^b - z^n), 
\end{equation}
where $M(z) = \mu/(\lambda(1-z)+\mu)$. Using Rouche's theorem, we find that both side of the equation shares $b$ zeros inside the circle $\mathcal{C} = \{z\in\mathbb{C}\text{ ; }|z| <1+\epsilon\}$ for some epsilon. 
\begin{tcolorbox}[enhanced,drop shadow, title=Rouche's theorem]
Let $\mathcal{C}\in \mathbb{C}$ and $f$ and $g$ two holomorphic functions on $\mathcal{C}$. Let $\partial\mathcal{C}$ be the contour of $\partial\mathcal{C}$. If 
$$
|f(z)-g(z)|<|g(z)|\text{, }\forall z\in\partial\mathcal{C}
$$ 
then $Z_f-P_f = Z_g-P_g$, where $Z_f$, $P_f$, $Z_g$, and $P_g$ are the number of zeros and poles of $f$ and $g$ respectively.

\end{tcolorbox}
We have $\partial\mathcal{C} =\{z\in\mathbb{C}\text{ ; }|z| =1+\epsilon\}$. The left hand side can be rewritten as
$$
G(z)\left[-\frac{\lambda}{\mu}z^{b+1} + \left(1 + \frac{\lambda}{\mu}\right)z^b -1\right].
$$
Define $f(z) = -\frac{\lambda}{\mu}z^{b+1} + \left(1 + \frac{\lambda}{\mu}\right)z^b -1$ and $g(z)=\left(1 + \frac{\lambda}{\mu}\right)z^b$. We have 
$$
|f(z) - g(z)| = |-\frac{\lambda}{\mu}z^{b+1}-1|< \frac{\lambda}{\mu}(1+\epsilon)^{b+1}+1\rightarrow \frac{\lambda}{\mu}+1,\text{ as }\epsilon \rightarrow 0. 
$$
and 
$$
|g(z)| = \left(1 + \frac{\lambda}{\mu}\right)(1+\epsilon)^b\rightarrow \frac{\lambda}{\mu}+1,\text{ as }\epsilon \rightarrow 0. 
$$
Regarding the right hand side, define $f(z) = \sum_{n=0}^{b-1}\alpha_n(z^b - z^n)$ and $g(z) =\sum_{n=0}^{b-1}\alpha_nz^b $. We have 
$$
|f(z) - g(z)| < |\sum_{n=0}^{b-1}\alpha_n z^n| < \sum_{n=0}^{b-1}\alpha_n (1+\epsilon)^n\rightarrow  \sum_{n=0}^{b-1}\alpha_n,\text{ as }\epsilon \rightarrow 0.
$$
and 
$$
|g(z)| = (1+\epsilon)^b\sum_{n=0}^{b-1}\alpha_n\rightarrow \sum_{n=0}^{b-1}\alpha_n,\text{ as }\epsilon \rightarrow 0.
$$
One of them is $1$, and we denote by $z_k$, $k = 1,\ldots, b-1$ the remaining $b-1$ zeros. Given the polynomial form of the right hand side of \eqref{eq:G_z_as_rational_function}, the fundamental theorem of algebra indicates that the number of zero is $b$. Given the left hand side 
$$
G(z)\left[-\frac{\lambda}{\mu}z^{b+1} + \left(1 + \frac{\lambda}{\mu}\right)z^b -1\right].
$$
we deduce that there is one zeros outside $\mathcal{C}$, we can further show that it is a real number $z^\ast$. Multiplying both side of \eqref{eq:G_z_as_rational_function} by $(z-1)\prod_{k =1}^{b-1}(z-z_k)$, and using $G(1)=1$ yields
$$
G(z) = \frac{1-z^\ast}{z-z^{\ast}}.
$$
$N^q$ is then a geometric random variable with parameter $p = \frac{1}{z^\ast}.$
\end{frame}
\begin{frame}{Latency and throughputs}
\scriptsize
\begin{tcolorbox}[enhanced,drop shadow, title=Little's law]
Consider a stationary queueing system and denote by 
\begin{itemize}
  \item $1/\lambda$ the mean of the unit inter-arrival times
  \item $L$ be the mean number of units in the system
  \item $W$ be the mean time spent by units in the system
\end{itemize}
We have
$$
L = \lambda \cdot W
$$
\end{tcolorbox}
\tiny
\begin{thebibliography}{1}

\bibitem{Little1961}
J.~D.~C. Little, ``A proof for the queuing formula:{L}= $\lambda${W},'' {\em
  Operations Research}, vol.~9, pp.~383--387, jun 1961.

\end{thebibliography}
\scriptsize
\begin{itemize}
  \item Latency is the confirmation time of a transaction 
    $$
    \text{Latency} = \frac{p}{(1-p)\lambda} + \frac{1}{\mu}
    $$
  \item Throughput is the number of transaction confirmed per time unit
  $$
    \text{Throughput} = \mu\mathbb{E}(N^q\mathbb{I}_{N^q\leq b}+b\mathbb{I}_{N^q> b}) = \mu\sum_{n = 0}^bn(1-p)p^n + bp^{b+1}.
  $$
\end{itemize}
\end{frame}
\begin{frame}{Perspective}
\begin{enumerate}
  \item  Include some priority consideration to account for the transaction fees 
  \tiny 
  \begin{thebibliography}{1}

\bibitem{Kawase2020}
Y.~Kawase, , and S.~Kasahara, ``Priority queueing analysis of
  transaction-confirmation time for bitcoin,'' {\em Journal of Industrial {\&}
  Management Optimization}, vol.~16, no.~3, pp.~1077--1098, 2020.

\end{thebibliography}

  \item \normalsize Go beyond the Poisson process framework
  \tiny
  \begin{thebibliography}{1}

\bibitem{Li2018}
Q.-L. Li, J.-Y. Ma, and Y.-X. Chang, ``Blockchain queue theory,'' in {\em
  Computational Data and Social Networks}, pp.~25--40, Springer International
  Publishing, 2018.

\bibitem{Li2019}
Q.-L. Li, J.-Y. Ma, Y.-X. Chang, F.-Q. Ma, and H.-B. Yu, ``Markov processes in
  blockchain systems,'' {\em Computational Social Networks}, vol.~6, jul 2019.

\end{thebibliography}
\end{enumerate}

\end{frame}
\begin{frame}
\bibliography{../../blockastics}
\bibliographystyle{ieeetr}
\end{frame}

\appendix

\begin{frame}{Two generals problem}
Two nodes who must agree are communicating through an unreliable link.
\begin{itemize}
  \item Analogy with two generals besieging a city
\end{itemize}
The generals exchange messages through enemy territory
\begin{itemize}
\item[G1]$$\texttt{"I will attack tomorrow at dawn, if you confirm"}$$
\item[G2] $$\texttt{"I will follow your lead, if you confirm"}$$
\end{itemize}
\begin{figure}[ht!]
 \begin{center}
\begin{tikzpicture}[->, >=stealth', auto, semithick, node distance=3cm]
\tikzstyle{every state}=[fill=white,draw=black,thick,text=black,scale=0.8]
\node[state]    (1)                     {$G_1$};
\node[state]    (2)[right of=1]   {$G_2$};
\path
(1) edge[bend left]     node{Attack}     (2)
(2) edge[bend left]     node{Confirmation}      (1);
\end{tikzpicture}
\end{center}
\caption{Message and confirmation loop}
\label{fig:message_loop}
\end{figure}
\end{frame}
\begin{frame}{Byzantine General problem}
    $n$ generals must agree on a common battle plan, to either 
    \begin{itemize}
    \item Attack (A) 
    \item Retreat (R)
  \end{itemize}
\begin{tcolorbox}[enhanced,drop shadow, title=Problem]
There are $m<n$ traitors among the generals
\end{tcolorbox}
\begin{enumerate}
\item message $m(i,j)$ is sent to general $j$ by general $i$ 
\item Consensus is reached as general $j$ applies 
$$
f(\{m(i,j);\text{ }i = 1,\ldots,n\}) = \begin{cases}
A,& \text{if }\sum_{i = 1}^n\mathbb{I}_{m(i,j) =A} >n/2,\\
R, &\text{else}.
\end{cases}
$$
\end{enumerate}
\end{frame}
\begin{frame}[plain]
\begin{figure}[!ht]
 \begin{center}
 \subfloat[No traitor]{
\begin{tikzpicture}[->, >=stealth', auto, semithick, node distance=2cm]
\tikzstyle{every state}=[fill=white,draw=black,thick,text=black,scale=0.8]
\node[state]    (1)                     {$G_1$};
\node[state]    (2)[below of=1]   {$G_2$};
\node[state]    (3)[below of=2]   {$G_3$};
\node[state]    (4)[below of=3]   {$G_4$};
\node[state]    (5)[below of=4]   {$G_5$};
\node[draw] (6) [right of=1]   { \footnotesize$f({A,R,R,A,A}) = A$};
\node[draw] (7) [right of=2]   { \footnotesize$f({A,R,R,A,A}) = A$};
\node[draw] (8) [right of=3]   { \footnotesize$f({A,R,R,A,A}) = A$};
\node[draw] (9) [right of=
4]   { \footnotesize$f({A,R,R,A,A}) = A$};
\node[draw] (10) [right of=5]   { \footnotesize$f({A,R,R,A,A}) = A$};
\path
(1) edge[bend left]     node{ A}     (2)
(2) edge[bend left]     node{ R}      (1)
(2) edge[bend left]     node{ R}      (3)
(3) edge[bend left]     node{ R}      (2)
(3) edge[bend left]     node{ R}      (4)
(4) edge[bend left]     node{ A}      (3)
(4) edge[bend left]     node{ A}      (5)
(5) edge[bend left]     node{ A}      (4)
;
\path
(1) edge     (6)
(2) edge     (7)
(3) edge     (8)
(4) edge     (9)
(5) edge     (10)
;
\end{tikzpicture}
\label{fig:no_traitor}}
\hskip2em
 \subfloat[One traitor]{
\begin{tikzpicture}[->, >=stealth', auto, semithick, node distance=2cm]
\tikzstyle{every state}=[fill=white,draw=black,thick,text=black,scale=0.8]
\node[state]    (1)                     {$G_1$};
\node[state]    (2)[below of=1]   {$G_2$};
\node[state]    (3)[below of=2]   {$G_3$};
\node[state]    (4)[below of=3]   {\color{red}$G_4$};
\node[state]    (5)[below of=4]   {$G_5$};
\node[draw] (6) [right of=1]   { \footnotesize$f({A,R,R,R,A}) = R$};
\node[draw] (7) [right of=2]   { \footnotesize$f({A,R,R,R,A}) = R$};
\node[draw] (8) [right of=3]   { \footnotesize$f({A,R,R,R,A}) = R$};
\node[draw] (9) [right of=
4]   { \footnotesize$f({A,R,R,?,A}) = ?$};
\node[draw] (10) [right of=5]   { \footnotesize$f({A,R,R,A,A}) = A$};
\path
(1) edge[bend left]     node{ A}     (2)
(2) edge[bend left]     node{ R}      (1)
(2) edge[bend left]     node{ R}      (3)
(3) edge[bend left]     node{ R}      (2)
(3) edge[bend left]     node{ R}      (4)
(4) edge[bend left]     node{ \color{red}R}      (3)
(4) edge[bend left]     node{ \color{red}A}      (5)
(5) edge[bend left]     node{ A}      (4)
;
\path
(1) edge     (6)
(2) edge     (7)
(3) edge     (8)
(4) edge     (9)
(5) edge     (10)
;
\end{tikzpicture}
\label{fig:_one_traitor}}
\end{center}
\caption{Majority vote with or without a traitor}
\label{fig:majority_vote}
\end{figure}
\end{frame}
\begin{frame}{Commanders and Lieutenants}
One general is the commander while the others are the lieutenants
\begin{tcolorbox}[enhanced,drop shadow, title=Objective]
Design an algorithm so that the following conditions are met:
\begin{itemize}
  \item[C1] All the loyal lieutenants obey the same order
  \item[C2] If the commanding general is loyal, then  every loyal lieutenants obey the order he sends
\end{itemize}
\end{tcolorbox}
\begin{tcolorbox}[enhanced,drop shadow, title=Byzantine Fault Tolerance Theorem (Lamport et al.)]
There are no solution to the Byzantine General problem for $n<3m+1$ generals, where $m$ is the number of traitors.
\end{tcolorbox}
\end{frame}
\begin{frame}[plain]
\begin{figure}[!ht]
 \begin{center}
 \subfloat[Commander is loyal]{
\begin{tikzpicture}[->, >=stealth', auto, semithick, node distance=4cm]
\tikzstyle{every state}=[fill=white,draw=black,thick,text=black,scale=0.8]
\node[state]    (1)               {C};
\node[state]    (2)[below left of=1]   {L1};
\node[state]    (3)[below right of=1]   {\begin{color}{red}L2 \end{color}};
\path
(1) edge[bend left]     node{A}     (3)
(1) edge[bend right, above]     node{A}      (2)
(2) edge[bend left]     node{A}      (3)
(3) edge[bend left]     node{R}      (2)
% (3) edge[bend left]     node{R}      (4)
% (4) edge[bend left]     node{A}      (3)
% (4) edge[bend left]     node{A}      (5)
% (5) edge[bend left]     node{A}      (4)
;
\end{tikzpicture}
\label{fig:commander_loyal}}
\hskip2em
\subfloat[Commander is a traitor]{
\begin{tikzpicture}[->, >=stealth', auto, semithick, node distance=4cm]
\tikzstyle{every state}=[fill=white,draw=black,thick,text=black,scale=0.8]
\node[state]    (1)               {\begin{color}{red}C \end{color}};
\node[state]    (2)[below left of=1]   {L1};
\node[state]    (3)[below right of=1]   {L2};
\path
(1) edge[bend left, above]     node{A}     (3)
(1) edge[bend right, above]     node{R}      (2)
(2) edge[bend left]     node{R}      (3)
(3) edge[bend left]     node{A}      (2)
% (3) edge[bend left]     node{R}      (4)
% (4) edge[bend left]     node{A}      (3)
% (4) edge[bend left]     node{A}      (5)
% (5) edge[bend left]     node{A}      (4)
;
\end{tikzpicture}
\label{fig:commander_traitor}}
\end{center}
\caption{Majority vote with or without a traitor}
\label{fig:majority_vote}
\end{figure}
\end{frame}
\begin{frame}[plain]
\begin{algorithm}[H]
\caption{The Oral message algorithm $\text{OM}(m)$}\label{alg:om}
\begin{algorithmic}
\If{$m=0$};
\For{$i =1 \to n-1$} 
\State Commander sends $v_i = v$ to lieutenant $i$ 
\State Lieutenant $i$ set their value to $v$
\EndFor
\EndIf
\If{$m>0$};
\For{$i =1 \to n-1$} 
\State Commander sends $v_i$ to lieutenant $i$ 
\State Lieutenant $i$ uses OM(m-1) to communicate $v_i$ to the $n-2$ lieutenants
\EndFor
\For{$i =1 \to n-1$} 
\State Lieutenant $i$ set their value to $f(v_1, \ldots, v_{n-1})$
\EndFor
\EndIf
\end{algorithmic}
\end{algorithm}
\end{frame}
\begin{frame}{$n = 4$ and $m = 1$: Step 1}
\begin{figure}[!ht]
 \begin{center}
 \subfloat[Commander is loyal]{
\begin{tikzpicture}[->, >=stealth', auto, semithick, node distance=4cm]
\tikzstyle{every state}=[fill=white,draw=black,thick,text=black,scale=0.8]
\node[state]    (1)               {C};
\node[state]    (2)[below left of=1]   {L1};
\node[state]    (3)[below of=1]   {L2};
\node[state]    (4)[below right of=1]   {\begin{color}{red}L3 \end{color}};
\path
(1) edge[bend right]     node{A}     (2)
(1) edge[bend left]     node{A}      (4)
(1) edge     node{A}      (3)
% (2) edge[bend right, below]     node{A}      (3)
% (3) edge[bend right, above]     node{A}      (2)
% (4) edge[bend right, above]     node{R}      (3)
% (3) edge[bend right, below]     node{A}      (4)
;
\end{tikzpicture}
\label{fig:commander_loyal_om}}
\hskip2em
\subfloat[Commander is a traitor]{
\begin{tikzpicture}[->, >=stealth', auto, semithick, node distance=4cm]
\tikzstyle{every state}=[fill=white,draw=black,thick,text=black,scale=0.8]
\node[state]    (1)               {\begin{color}{red}C \end{color}};
\node[state]    (2)[below left of=1]   {L1};
\node[state]    (3)[below of=1]   {L2};
\node[state]    (4)[below right of=1]   {L3};
\path
(1) edge[bend right]     node{A}     (2)
(1) edge[bend left]     node{R}      (4)
(1) edge     node{R}      (3)
% (2) edge[bend right, below]     node{A}      (3)
% (3) edge[bend right, above]     node{R}      (2)
% (4) edge[bend right, above]     node{R}      (3)
% (3) edge[bend right, below]     node{R}      (4)
;
\end{tikzpicture}
\label{fig:commander_traitor_om}}
\end{center}
\caption{Illustration of the \text{OM}(m) algorithm in the case where $n = 4$ and $m=1$.}
\label{fig:OM_algorithm_illustration}
\end{figure}
\end{frame}
\begin{frame}{$n = 4$ and $m = 1$: Step 2}
\begin{figure}[!ht]
 \begin{center}
 \subfloat[Commander is loyal]{
\begin{tikzpicture}[->, >=stealth', auto, semithick, node distance=4cm]
\tikzstyle{every state}=[fill=white,draw=black,thick,text=black,scale=0.8]
\node[state]    (1)               {C};
\node[state]    (2)[below left of=1]   {L1};
\node[state]    (3)[below of=1]   {L2};
\node[state]    (4)[below right of=1]   {\begin{color}{red}L3 \end{color}};
\path
(1) edge[bend right]     node{A}     (2)
(1) edge[bend left]     node{A}      (4)
(1) edge     node{A}      (3)
(2) edge[bend right, below]     node{A}      (3)
(3) edge[bend right, above]     node{A}      (2)
(4) edge[bend right, above]     node{R}      (3)
(3) edge[bend right, below]     node{A}      (4)
;
\end{tikzpicture}
\label{fig:commander_loyal_om}}
\hskip2em
\subfloat[Commander is a traitor]{
\begin{tikzpicture}[->, >=stealth', auto, semithick, node distance=4cm]
\tikzstyle{every state}=[fill=white,draw=black,thick,text=black,scale=0.8]
\node[state]    (1)               {\begin{color}{red}C \end{color}};
\node[state]    (2)[below left of=1]   {L1};
\node[state]    (3)[below of=1]   {L2};
\node[state]    (4)[below right of=1]   {L3};
\path
(1) edge[bend right]     node{A}     (2)
(1) edge[bend left]     node{R}      (4)
(1) edge     node{R}      (3)
(2) edge[bend right, below]     node{A}      (3)
(3) edge[bend right, above]     node{R}      (2)
(4) edge[bend right, above]     node{R}      (3)
(3) edge[bend right, below]     node{R}      (4)
;
\end{tikzpicture}
\label{fig:commander_traitor_om}}
\end{center}
\caption{Illustration of the \text{OM}(m) algorithm in the case where $n = 4$ and $m=1$.}
\label{fig:OM_algorithm_illustration}
\end{figure}
\end{frame}
\begin{frame}{$n = 4$ and $m = 1$: Step 3}
\begin{figure}[!ht]
 \begin{center}
 \subfloat[Commander is loyal, C1 and C2]{
\begin{tikzpicture}[->, >=stealth', auto, semithick, node distance=4cm]
\tikzstyle{every state}=[fill=white,draw=black,thick,text=black,scale=0.78]
\node[state]    (1)               {C};
\node[state]    (2)[below left of=1]   {L1};
\node[state]    (3)[below of=1]   {L2};
\node[state]    (4)[below right of=1]   {\begin{color}{red}L3 \end{color}};

\path
(1) edge[bend right]     node{A}     (2)
(1) edge[bend left]     node{A}      (4)
(1) edge     node{A}      (3)
(2) edge[bend right, below]     node{A}      (3)
(3) edge[bend right, above]     node{A}      (2)
(4) edge[bend right, above]     node{R}      (3)
(3) edge[bend right, below]     node{A}      (4)
(2) edge[loop below] node{\tiny f(A,A,R)=A } (2)
(3) edge[loop below] node{\tiny f(A,A,R)=A } (3)
;
\end{tikzpicture}
\label{fig:commander_loyal_om}}
\hskip1em
\subfloat[Commander is a traitor, C1]{
\begin{tikzpicture}[->, >=stealth', auto, semithick, node distance=4cm]
\tikzstyle{every state}=[fill=white,draw=black,thick,text=black,scale=0.78]
\node[state]    (1)               {\begin{color}{red}C \end{color}};
\node[state]    (2)[below left of=1]   {L1};
\node[state]    (3)[below of=1]   {L2};
\node[state]    (4)[below right of=1]   {L3};
\path
(1) edge[bend right]     node{A}     (2)
(1) edge[bend left]     node{R}      (4)
(1) edge     node{R}      (3)
(2) edge[bend right, below]     node{A}      (3)
(3) edge[bend right, above]     node{R}      (2)
(4) edge[bend right, above]     node{R}      (3)
(3) edge[bend right, below]     node{R}      (4)
(2) edge[loop below] node{\tiny f(A,R,R)=R } (2)
(3) edge[loop below] node{\tiny f(A,R,R)=R } (3)
;
;
\end{tikzpicture}
\label{fig:commander_traitor_om}}
\end{center}
\caption{Illustration of the \text{OM}(m) algorithm in the case where $n = 4$ and $m=1$.}
\label{fig:OM_algorithm_illustration}
\end{figure}
\end{frame}
\begin{frame}{The problem with majority vote}
The OM algorithm requires to send $n^{m+1}$
\begin{itemize}
  \item[\danger] Communication overhead
  \item[\danger] Denial of service
\end{itemize}
\begin{tcolorbox}[enhanced,drop shadow, title=Solution]
Leader based protocols!
\end{tcolorbox}
\end{frame}
\begin{frame}{Proof-of-Work}
\begin{tcolorbox}[enhanced,drop shadow, title=Objective]
    Elect a leader based on computational effort to append the next block.
\end{tcolorbox}
\end{frame}

\end{document}
