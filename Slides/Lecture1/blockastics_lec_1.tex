\documentclass{beamer}
\usepackage[utf8]{inputenc}
\usepackage[T1]{fontenc}
% \usepackage{amscd, amsfonts, amsmath, amssymb, amstext, amsthm, caption, epsfig, fancyhdr, float, graphicx, latexsym, mathtools, multicol, multirow, algorithm, chngcntr}
\usepackage[english, french]{babel}
\usepackage{booktabs}

\usepackage{amsmath,amssymb}
\usepackage{graphicx}
\usepackage{caption}
\usepackage{subfig}
\usepackage{xspace}
\usepackage{fourier}

\usepackage{tikz}
\usetikzlibrary{positioning}
\usetikzlibrary{fit}
\usetikzlibrary{backgrounds}
\usetikzlibrary{calc}
\usetikzlibrary{shapes}
\usetikzlibrary{mindmap}
\usetikzlibrary{decorations.text}


% tikzmark command, for shading over items
\newcommand{\tikzmark}[1]{\tikz[overlay,remember picture] \node (#1) {};}
% Define block styles
\tikzstyle{decision} = [diamond, draw, fill=blue!20,
    text width=4.5em, text badly centered, node distance=3cm, inner sep=0pt]
\tikzstyle{block} = [rectangle, draw, fill=blue!20,
    text width=5em, text centered, rounded corners]
\tikzstyle{line} = [draw]
\tikzstyle{cloud} = [draw, ellipse,fill=red!20, node distance=3cm,
    minimum height=2em]

\usepackage[most]{tcolorbox}

\setbeamertemplate{blocks}[rounded][shadow=true] % use rounded blocks with standard beamer shadow


% Distributions.
\newcommand*{\UnifDist}{\mathsf{Unif}}
\newcommand*{\ExpDist}{\mathsf{Exp}}
\newcommand*{\DepExpDist}{\mathsf{DepExp}}
\newcommand*{\GammaDist}{\mathsf{Gamma}}
\newcommand*{\LognormalDist}{\mathsf{LogNorm}}
\newcommand*{\WeibullDist}{\mathsf{Weib}}
\newcommand*{\ParetoDist}{\mathsf{Par}}
\newcommand*{\NormalDist}{\mathsf{Norm}}

\newcommand*{\GeometricDist}{\mathsf{Geom}}
\newcommand*{\NegBinomialDist}{\mathsf{NegBin}}
\newcommand*{\PoissonDist}{\mathsf{Poisson}}
\newcommand*{\BivariatePoissonDist}{\mathsf{BPoisson}}
\newcommand*{\CyclicalPoissonDist}{\mathsf{CPoisson}}

\newcommand*{\iid}{\textbf{iid}\@\xspace}
\newcommand*{\pdf}{\textbf{pdf}\@\xspace}
\newcommand*{\cdf}{\textbf{cdf}\@\xspace}
\newcommand*{\pmf}{\textbf{pmf}\@\xspace}
\newcommand*{\abc}{{\textbf{abc}}\@\xspace}
\newcommand*{\smc}{\textbf{smc}\@\xspace}
\newcommand*{\mcmc}{\textbf{mcmc}\@\xspace}
\newcommand*{\ess}{\textbf{ess}\@\xspace}
\newcommand*{\mle}{\textbf{mle}\@\xspace}
\newcommand*{\bic}{\textbf{bic}\@\xspace}
\newcommand*{\kde}{\textbf{kde}\@\xspace}
\newcommand*{\glm}{\textbf{glm}\@\xspace}
\newcommand*{\xol}{\textbf{xol}\@\xspace}
\newcommand*{\cpu}{\textbf{cpu}\@\xspace}
\newcommand*{\gpu}{\textbf{gpu}\@\xspace}
\newcommand*{\arm}{\textbf{arm}\@\xspace}

\def \si {\sigma}
\def \la {\lambda}
\def \al {\alpha}
% \def\e*{\end{eqnarray*}}
\def \di{\displaystyle}

\def \E{\mathbb E}
\def \N{\mathbb N}
\def \Z{\mathbb Z}
\def \NZ{\mathbb{N}_0}
\def \I{\mathbb I}
\def \w{\widehat}
\def \P {\mathbb P}
\def \V{\mathbb V}


\newcommand{\CL}{\mathbb{C}}
\newcommand{\RL}{\mathbb{R}}
\newcommand{\nat}{{\mathbb N}}
\newcommand{\Laplace}{\mathscr{L}}
\newcommand{\e}{\mathrm{e}}
\newcommand{\ve}{\bm{\mathrm{e}}} % vector e

\renewcommand{\L}{\mathcal{L}} % e.g. L^2 loss.

\newcommand{\ih}{\mathrm{i}}
\newcommand{\oh}{{\mathrm{o}}}
\newcommand{\Oh}{{\mathcal{O}}}
\newcommand{\Exp}{\mathbb{E}}

\newcommand{\Norm}{\mathcal{N}}
\newcommand{\LN}{\mathcal{LN}}
\newcommand{\SLN}{\mathcal{SLN}}

\renewcommand{\Pr}{\mathbb{P}}
\newcommand{\Ind}{\mathbb I}
\newcommand\bfsigma{\bm{\sigma}}
\newcommand\bfSigma{\bm{\Sigma}}
\newcommand\bfLambda{\bm{\Lambda}}
\newcommand{\stimes}{{\times}}
\def \limsup{\underset{n\rightarrow+\infty}{\overline{\lim}}}
\def \liminf{\underset{n\rightarrow+\infty}{\underline{\lim}}}




% vertical separator macro
\newcommand{\vsep}{
  \column{0.0\textwidth}
    \begin{tikzpicture}
      \draw[very thick,black!10] (0,0) -- (0,7.3);
    \end{tikzpicture}
}
\newcommand\blfootnote[1]{%
  \begingroup
  \renewcommand\thefootnote{}\footnote{#1}%
  \addtocounter{footnote}{-1}%
  \endgroup
}

% More space between lines in align
% \setlength{\mathindent}{0pt}

% Beamer theme
\usetheme{ZMBZFMK}
\usefonttheme[onlysmall]{structurebold}
\mode<presentation>
\setbeamercovered{transparent=10}

% align spacing
\setlength{\jot}{0pt}

\setbeamertemplate{navigation symbols}{}%remove navigation symbols

\title[BLOCKASTICS I]{Stochastic Models for blockchain analysis}
\subtitle{Introduction}
\author{Pierre-O. Goffard}
\institute[ISFA]{Institut de Science Financières et d'Assurances\\
 \texttt{pierre-olivier.goffard@univ-lyon1.fr}}
\date{\today}
\titlegraphic{\includegraphics[width=2.5cm]{../../Figures/bfs_logo.png}} 

\begin{document}
\begin{frame}
  \titlepage
\end{frame}
\begin{frame}{Blockchain}
A data ledger made of a sequence of blocks maintained by a achieving consensus in a Peer-To-Peer network.
\begin{itemize}
  \item Decentralized
  \item Public/private
  \item Permissionned/permissionless
  \item Immutable
  \item Incentive compatible
  \begin{itemize}
    \item The nodes are compensated for their hard work in cryptocurrencies
  \end{itemize}
\end{itemize}

\vspace{0.2cm}
We will focus on public blockchain and their associated consensus protocol.
\end{frame}
\begin{frame}{Blocks}
A block contains
\begin{itemize}
  \item block height/ID
  \item Time stamp
  \item hash of the block
  \item hash of the previous block
  \item Set of transactions (data stored in the blockchain)
\end{itemize}
\begin{center}
\includegraphics[width=0.9\textwidth]{../../Figures/genesis_block.png}
\end{center}
\end{frame}
\begin{frame}{Cryptographic Hash function}
\small
A function that maps data of arbitratry size (message) to a bit array of fixed size (hash value)
$$
h:\{0,1\}^\ast\mapsto \{0,1\}^d. 
$$
A good hash function is
\begin{itemize}
\item deterministic
\item quick to compute
\item One way
\begin{itemize}
  \scriptsize
\item[$\hookrightarrow$] For a given hash value $\overline{h}$ it is hard to find a message $m$ such that 
$$
h(m) = \overline{h}
$$
\end{itemize}
\item Colision resistant 
\begin{itemize}
\item[$\hookrightarrow$] Impossible to find $m_1$ and $m_2$ such that 
$$
h(m_1) = h(m_2)
$$
\end{itemize}
\item Chaotic
$$m_1\approx m_2\Rightarrow  h(m_1) \neq h(m_2)$$
\end{itemize}

\end{frame}
\begin{frame}{Consensus protocols}
The mechanism to make all the nodes agree on a common data history.\\
\vspace{0.3cm}
The three dimensions of blockchain systems analysis
\begin{enumerate}
  \item Efficiency (Queueing theory)
  \begin{itemize}
    \item Throughputs
    \item Transaction confirmation time
  \end{itemize}
  \item Decentralization (Entropy)
  \begin{itemize}
    \item Fair distribution of the accounting right
  \end{itemize}
  \item Security (Insurance Risk Theory)
  \begin{itemize}
    \item Resistance to attacks
  \end{itemize}
\end{enumerate}
\footnotesize
\begin{thebibliography}{1}
\bibitem{Fu2020}
X.~Fu, H.~Wang, and P.~Shi, ``A survey of blockchain consensus algorithms:
  mechanism, design and applications,'' {\em Science China Information
  Sciences}, vol.~64, nov 2020.
\end{thebibliography}

\end{frame}
\begin{frame}{Proof of Work}
The nodes compete to solve a cryptographic problem by brute force search.
\begin{tcolorbox}[enhanced,drop shadow, title=PoW]
\begin{enumerate}
    \item Draw a random number (nonce)
    \[
    X\sim\{1,\ldots, 2^{32}\}.
    \]
    \item While $X > L$, where $L$ is the target then try again  
\end{enumerate}
\end{tcolorbox}
\vspace{0.3cm}
Nodes are chosen according to their computing power
{\footnotesize
\begin{thebibliography}{1}
\bibitem{Na08}
S.~Nakamoto, ``Bitcoin: A peer-to-peer electronic cash system.'' Available at
  \href{https://bitcoin.org/bitcoin.pdf}{https://bitcoin.org/bitcoin.pdf},
  2008.
\end{thebibliography}  
}
\end{frame}
\begin{frame}{Proof of Stake}
PoW is slow and ressource consuming. Let $\{1,\ldots, N\}$ be a set of miner and $\{\pi_1,\ldots, \pi_N\}$ be their share of cryptocoins.
\begin{tcolorbox}[enhanced,drop shadow, title=PoS]
Node $i\in \{1,\ldots, N\}$ is selected with probability $\pi_i$ to append the next block
\end{tcolorbox}
\vspace{0.3cm}
Nodes are chosen according to what they own.
\begin{itemize}
  \item Nothing at stake problem
  \item Rich gets richer ? (To be discussed later on)
\end{itemize}
\footnotesize{
\begin{thebibliography}{1}

\bibitem{Saleh2020}
F.~Saleh, ``Blockchain without waste: Proof-of-stake,'' {\em The Review of
  Financial Studies}, vol.~34, pp.~1156--1190, jul 2020.

\end{thebibliography}}

\end{frame}
\begin{frame}{Applications of blockchain: Cryptocurrency}
\begin{columns}
\begin{column}{0.5\textwidth}
   
{\footnotesize
\begin{thebibliography}{1}
\bibitem{Na08}
S.~Nakamoto, ``Bitcoin: A peer-to-peer electronic cash system.'' Available at
  \href{https://bitcoin.org/bitcoin.pdf}{https://bitcoin.org/bitcoin.pdf},
  2008.
\end{thebibliography}  
}
\end{column}
\begin{column}{0.5\textwidth}  %%<--- here
    \begin{center}
     \includegraphics[width=0.5\textwidth]{../../Figures/bitcoin-6284869_1920.png}
     \end{center}
\end{column}
\end{columns}

\begin{itemize}
  \item Transaction anonymity
  \item Banking and reliable currency in certain regions of the world
  \item Money Transfer worldwide (at low fare)
  \item No need for a thrusted third party
\end{itemize}



\end{frame}
\begin{frame}{Decentralized application}
The network provide ressources such as
\begin{itemize}
  \item storage
  \item computing power
\end{itemize}
through a smart contract on the ethereum blockchain. 
\vspace{0.3cm}
\begin{tcolorbox}[enhanced,drop shadow, title=GOLEM \url{https://www.golem.network/}]
    Build a network of idle computers to do paralell computing. 
\end{tcolorbox}
Utility tokens are used to access the service and provision the network ressources.
\begin{tcolorbox}[enhanced,drop shadow, title=Equation of Exchange]
    \[MV = PQ\] 
\end{tcolorbox}
\end{frame}
\begin{frame}{Decentralized finance}
DEFI creates new financial architecture
\begin{columns}
\begin{column}{0.5\textwidth}
\begin{itemize}
\item[+] Non custodial
\item[+] Anonymous
\item[+] Permisionless
\item[+] openly auditable
\end{itemize}
\end{column}
\begin{column}{0.5\textwidth} 
\begin{itemize}
\item[-] Unregulated
\item[-] Tax evasion
\item[-] Fraud
\item[-] Money laundering
\end{itemize} 
\end{column}
\end{columns}
\vspace{0.5cm}
Extends the Bitcoin promises to more complex financial operations
\begin{itemize}
  \item Collateralized lending
  \item Decentralized Exchange Platform
  \item Tokenized assets
  \item Fundraising vehicle (ICO, STO, ...)
\end{itemize}
\vspace{0.3cm}
\scriptsize
\begin{thebibliography}{1}

\bibitem{werner2021sok}
S.~M. Werner, D.~Perez, L.~Gudgeon, A.~Klages-Mundt, D.~Harz, and W.~J.
  Knottenbelt, ``Sok: Decentralized finance (defi),'' 2021.

\end{thebibliography}

\end{frame}
\begin{frame}{Decentralized insurance}
\begin{tcolorbox}[enhanced,drop shadow, title=Parametric insurance]
    Compensation if a measurable quantity reaches a threshold 
\end{tcolorbox}
\begin{itemize}
  \item Example: Flight delay insurance
  \begin{itemize}
    \item \url{https://etherscan.io/address/0xdc3d8fc2c41781b0259175bdc19516f7da11cba7}
  \end{itemize}
  \item Use smart contract and off-chain data through oracles
  \item Transparent and automatic
\end{itemize}
Another example is P2P insurance.
\end{frame}


\begin{frame}
\bibliography{../../blockastics}
\bibliographystyle{ieeetr}
\end{frame}

\end{document}
