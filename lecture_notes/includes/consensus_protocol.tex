% !TEX root = ../main_lecture_notes.tex
\chapter{Consensus protocol}\label{chap:consensus}

The problem of reaching consensus within a Peer-to-peer network is a very old problem in computer science. An obvious solution is to rely on a majority vote. This is the solution proposed by SHostak and his co-author who, in passing, make a famous analogy with Byzantine generals trying to agree on a common battle plan. Here the battle plan corresponds to adding a new block with a set of transactions deemed valid and therefore agreeing on a common data history. A voting system inside a large network involves a colossal number of messages exchanged leading to the consumption of all the bandwidth, the failure of certain nodes by denial of service and delays in the synchronization of the network. Castro and Liskov's Practical Byzantine Fault Tolerance (PBFT) algorithm is the gold standard for practical implementation of a voting system within a peer-to-peer network. Despite these recent advances, such a system is not suitable for a network that can grow indefinitely. Bitcoin solved this scaling problem by proposing a system based on the election of a leader making unilateral decisions. The Proof of Work protocol appoints a leader based on its computing resources. Each node competes to solve a puzzle with a brute force search algorithm. The first node who is able to propose a solution append the next block. The search for a solution, referred to as mining, is associated with an operational cost borne by the nodes which is compensated by a reward expressed in the native blockchain cryptocurrency. The surge in cryptocurrency prices has led to a rush in block mining, leading to a major spike in the electricity consumption and electronic waste generation of blockchain networks. The blockchain network consumes as much electricity as countries the size of Thailand at the time of the writing. The need for a more restricted consensus protocol therefore becomes crucial. These remain based on the election of a leader but rely on other network resources. The Proof-of-Stake protocol sample the nodes with
The proof of storage (no arm race, no electronic waste and useful like the file coin project)
\section{Voting system}\label{sec:voting}
\section{Leader system}\label{sec:leader}
\begin{itemize}
\item Public blockchain, 
\item operational cost, 
\item reward, 
\item incentive compatible
\item Uses the scarce ressource of the network
\begin{itemize}
	\item Computational power (CPU, GPU)
	\item Bandwidth
	\item Storage space
	\item Crypto coins
\end{itemize}
\end{itemize}
\subsection{Proof-of-Work}\label{ssec:pow}
\subsection{Proof-of-Stake}\label{ssec:pos}

