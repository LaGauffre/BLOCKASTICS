% !TEX root = ../main_lecture_notes.tex
\chapter{Decenralized Finance and Data Analysis}\label{chap:defi}

\section{Decentralized Finance}

Let us first list the distinctions between traditional finance (TradFi) and decentralized finance (DeFi). TradFi is often characterized by high access barriers, requiring specific criteria such as bank accounts, whereas DeFi eliminates these barriers, allowing universal participation. TradFi operates on centralized systems where banks serve as the primary record keepers, exposing them to cyber risks, whereas DeFi utilizes a decentralized ledger system, enhancing security and reducing such vulnerabilities. Moreover, TradFi is plagued by high transaction costs, including fees for account maintenance and wire transfers, and relies on intermediaries for transactions; in contrast, DeFi minimizes these costs by using smart contracts, although users must still handle gas fees. Transactions in TradFi, especially cross-border ones, can be slow, taking days to settle, while DeFi offers near-instant settlement. Furthermore, TradFi lacks transparency and can be difficult to audit, whereas DeFi provides a publicly available ledger and open-source code, ensuring greater transparency and auditability. Additionally, TradFi operations are restricted by geographical and regulatory constraints, and are often limited to business hours, while DeFi offers 24/7 global accessibility without censorship or restrictions by central authorities. TradFi's challenges in providing fractional ownership for assets like real estate and art are addressed in DeFi through the tokenization of real-world assets. Lastly, TradFi often relies on outdated IT solutions, which stifles innovation and interoperability, whereas DeFi fosters rapid innovation and enables seamless interoperability across platforms and projects. These discussion can be found for instance in the bok of \citet{Lipton2021}.
\subsection{Maximum Extractable value}
\subsection{Automated Market Makers}
\section{Data and statistics}
\begin{itemize}
\item Price of cryptoassets coming from various sources
\begin{itemize}
\item Aggregation of data (Gobet)
\end{itemize}
\item Cope with high level of volatility
\begin{itemize}
  \item Stochastic volatility model (inference via ABC)
  \item Hidden Markov model (inference via Gibbs)
\end{itemize}
\end{itemize}
\subsection{Stochastic volatility model}
\subsubsection{Definition}
\subsubsection{ABC}
\subsubsection{Application to stochastic volatility}
\subsection{Detecting trend in (decentralized) financial data}
\subsubsection{Hidden Markov models}
\subsubsection{An example: Markov Modulated Geometric Brownian Motions}
\subsubsection{Inference and filtering using a Gibbs sampler}